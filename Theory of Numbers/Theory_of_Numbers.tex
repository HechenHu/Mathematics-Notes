%%%%%%%%%%%%%%%%%%%%%%%%%%%%%%%%%%%%%%%%%%%%%%%%%%%
%% LaTeX book template                           %%
%% Author:  Amber Jain (http://amberj.devio.us/) %%
%% License: ISC license                          %%
%%%%%%%%%%%%%%%%%%%%%%%%%%%%%%%%%%%%%%%%%%%%%%%%%%%

\documentclass[a4paper,11pt]{book}
\usepackage[T1]{fontenc}
\usepackage[utf8]{inputenc}
\usepackage{lmodern}
%%%%%%%%%%%%%%%%%%%%%%%%%%%%%%%%%%%%%%%%%%%%%%%%%%%%%%%%%
% Source: http://en.wikibooks.org/wiki/LaTeX/Hyperlinks %
%%%%%%%%%%%%%%%%%%%%%%%%%%%%%%%%%%%%%%%%%%%%%%%%%%%%%%%%%
\usepackage{/Users/HechenHu/Development/NoteTaking/Mathematics-Notes/Customized}
%%%%%%%%%%%%%%%%%%%%%%%%%%%%%%%%%%%%%%%%%%%%%%%%
% Chapter quote at the start of chapter        %
% Source: http://tex.stackexchange.com/a/53380 %
%%%%%%%%%%%%%%%%%%%%%%%%%%%%%%%%%%%%%%%%%%%%%%%%

\makeatletter

\renewcommand{\@chapapp}{}% Not necessary...

\newenvironment{chapquote}[2][2em]
{\setlength{\@tempdima}{#1}%
	\def\chapquote@author{#2}%
	\parshape 1 \@tempdima \dimexpr\textwidth-2\@tempdima\relax%
	\itshape}
{\par\normalfont\hfill--\ \chapquote@author\hspace*{\@tempdima}\par\bigskip}
\makeatother

%%%%%%%%%%%%%%%%%%%%%%%%%%%%%%%%%%%%%%%%%%%%%%%%%%%
% First page of book which contains 'stuff' like: %
%  - Book title, subtitle                         %
%  - Book author name                             %
%%%%%%%%%%%%%%%%%%%%%%%%%%%%%%%%%%%%%%%%%%%%%%%%%%%

% Book's title and subtitle
\title{\Huge \textbf{Theory of Numbers}}
% Author
\author{\textsc{Hechen Hu}}

\begin{document}
	\frontmatter
	\maketitle
	%%%%%%%%%%%%%%%%%%%%%%%%%%%%%%%%%%%%%%%%%%%%%%%%%%%%%%%%%%%%%%%%%%%%%%%%
	% Auto-generated table of contents, list of figures and list of tables %
	%%%%%%%%%%%%%%%%%%%%%%%%%%%%%%%%%%%%%%%%%%%%%%%%%%%%%%%%%%%%%%%%%%%%%%%%
	
	\tableofcontents
	
	\mainmatter
	
	\chapter{Divisibility, the Fundamental Theorem of Number Theory}
	\section{Divisibility}
	\begin{Definition}
	The divisors of a number that are less than the number itself is called its \textit{parts}. If a number is the sum of its parts, it's called a \textit{perfect number}(e.g. $ 6 $, $ 28 $, and $ 496 $). If two numbers are the sum of the other one's parts, they are called \textit{amicable}(e.g. $ 220 $ and $ 284 $).
\end{Definition}
\begin{Theorem}[Remainder Theorem]
	For all numbers $ a $ and $ b \neq 0 $, there is an integer $ c $ and a number $ d $ such that
	\begin{equation}
		a=bc+d\quad \text{ and }\quad 0 \leqslant d < |b| \nonumber
	\end{equation}
	and only one such $ c $ and $ d $ exist. We say that $ a $ divided by $ b $ has \textit{quotient} $ c $ with \textit{remainder} $ d $.
\end{Theorem}
\begin{Proposition}
	For all numbers $ a $ and $ b\neq 0 $, there is an integer $ c^\prime $ and a number $ d^\prime $ such that
	\begin{equation}
		a = b c^\prime + d^\prime\quad \text{ and }\quad -\frac{|b|}{2}< d^\prime \leqslant \frac{|b|}{2} \nonumber
	\end{equation}
	and only one such $ c^\prime $ and $ d^\prime $.
\end{Proposition}
\begin{Theorem}[Four Number Theorem]
	If $ a $ and $ c $ are numbers and $ b $ and $ d $ are integers such that
	\begin{equation}
		ab=cd \nonumber
	\end{equation} 
	then there exists a positive number $ r $ and positive integers $ s $, $ t $, and $ u $ such that the following equalities hold:
	\begin{equation}
		a=rs,\quad b=tu,\quad c=rt, \quad d=su \nonumber
	\end{equation}
	If, in addition, $ a $ and $ c $ are integers, then $ r $ may be taken to be an integer.
\end{Theorem}
\begin{Definition}
	An integer $ a $ is a \textit{divisor} of an integer $ b $ if there exists a number $ c $ such that
	\begin{equation}
		b=ac \nonumber
	\end{equation}
	In this case we also say that $ b $ is \textit{divisible} by $ a $ and denoted $ a | b $. Otherwise, it is denoted $ a\nmid b$. Among the divisors of $ a $, $ 1 $, $ -1 $, $ a $, and $ -a $ is called its \textit{trivial divisors}. Other positive divisors smaller than $ a $ are called its \textit{proper divisors}. $ 1 $ and $ -1 $ is called \textit{units}.
\end{Definition}
\begin{Definition}
	Two numbers that do not have a common divisor other than the units are called \textit{relatively prime}.
\end{Definition}
\begin{Example}
	for any number $ a $
	\begin{itemize}
		\item $ a|0 $; 
		\item $ 0 $ is only a divisor of $ 0 $.
		\item If $ a |b$ and $ b|c $, then $ a|c $.
	\end{itemize}
\end{Example}
Division is reflexive and transitive. In general it is not symmetric.\newpara
If $ b_i $ are integers such that $ a|b_i $, and $ c_i $ are arbitrary integers ($ i=1,2,\cdots,k $), then $ a | \sum_{i=1}^{k}b_i c_i $.
\begin{Definition}
	A number $ a $ and $ -a $ is said to be \textit{associates} of each other. Theorems relating to divisibility apply to the classes of associated numbers.
\end{Definition}
\begin{Example}
	If $ a|b $, then $ ca | cb $, and if $ c \neq 0 $, then the first relation follows from the second.
\end{Example}
\begin{Lemma}[Euclid's Lemma]
	If a number divides the product of two numbers and is relatively prime to one of the factors, then it must divide the other factor.
\end{Lemma}



\section{Prime Numbers}
\begin{Definition}
	If a number only has the trivial ones as its divisors, it's called \textit{prime}. If a number is not prime and not unit, it's called a \textit{composite number}.
\end{Definition}
\begin{Theorem}
	Every  number larger than one has a prime divisor.
\end{Theorem}
\begin{Theorem}
	There are infinitely many prime numbers.
\end{Theorem}
\begin{Theorem}
	Every number different from $ 0 $ and not a unit can be decomposed into the product of finitely many primes.
\end{Theorem}
\begin{Definition}
	For certain number, if it divide a product of numbers, it also divide one of the factors. Numbers of this type that are different from $ 0 $ and the units have the \textit{prime property}.
\end{Definition}
\begin{Theorem}
	The prime numbers are precisely those with the prime property.
\end{Theorem}
\begin{Theorem}[Fundamental Theorem of Arithmetic]
	The prime factorization of a nonzero number that is not a unit is unique up to the order and signs of the factors.
\end{Theorem}
\begin{Proposition}
	If $ a_1,\cdots,a_j $; $ b_1,\cdots,b_k $ are integers such that
	\begin{equation}
		a_1 a_2\cdots a_j = b_1 b_2 \cdots b_k \nonumber
	\end{equation}
	then there exists integer $ t_{uv} $ ($ 1 \leqslant u \leqslant j,1 \leqslant v \leqslant k $) such that
	\begin{equation}
		a_u = \prod_{v=1}^{k}t_{uv},\qquad b_v = \prod_{u=1}^{j}t_{uv}\nonumber
	\end{equation}
\end{Proposition}
	
	
	
	\chapter{Congruences}
	
	\chapter{Rational and Irrational Numbers. Approximation of Numbers by Rational Numbers (Diophantine Approximation)}
	
	\chapter{Geometric Methods in Number Theory}
	
	\chapter{Properties of Prime Numbers}
	
	\chapter{Sequences of Integers}
	
	\chapter{Diophantine Problems}
	
	\chapter{Arithmetic Functions}
	
	
	
	
	
	
	
	
	
	
	
	
	
	
	
	
	
	
	
	
\end{document}