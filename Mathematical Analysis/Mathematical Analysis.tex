%%%%%%%%%%%%%%%%%%%%%%%%%%%%%%%%%%%%%%%%%%%%%%%%%%%
%% LaTeX book template                           %%
%% Author:  Amber Jain (http://amberj.devio.us/) %%
%% License: ISC license                          %%
%%%%%%%%%%%%%%%%%%%%%%%%%%%%%%%%%%%%%%%%%%%%%%%%%%%

\documentclass[a4paper,11pt]{book}
\usepackage[T1]{fontenc}
\usepackage[utf8]{inputenc}
\usepackage{lmodern}
%%%%%%%%%%%%%%%%%%%%%%%%%%%%%%%%%%%%%%%%%%%%%%%%%%%%%%%%%
% Source: http://en.wikibooks.org/wiki/LaTeX/Hyperlinks %
%%%%%%%%%%%%%%%%%%%%%%%%%%%%%%%%%%%%%%%%%%%%%%%%%%%%%%%%%
\usepackage{hyperref}
\usepackage{graphicx}
\usepackage[english]{babel}
\usepackage{amsmath}
\usepackage{amsfonts}
\usepackage{amsthm}
\usepackage{mathtools}
\usepackage{amssymb}

%%%%%%%%%%%%%%%%%%%%%%%%%%%%%%%%%%%%%%%%%%%%%%%%
% Chapter quote at the start of chapter        %
% Source: http://tex.stackexchange.com/a/53380 %
%%%%%%%%%%%%%%%%%%%%%%%%%%%%%%%%%%%%%%%%%%%%%%%%
\makeatletter

\renewcommand{\@chapapp}{}% Not necessary...
\newcommand{\Real}{\mathbb{R}}
\newcommand{\Natural}{\mathbb{N}}
\newcommand{\Zahlen}{\mathbb{Z}}
\newcommand{\Quoziente}{\mathbb{Q}}
\newcommand{\newpara}{\\ \newline}

\newenvironment{chapquote}[2][2em]
  {\setlength{\@tempdima}{#1}%
   \def\chapquote@author{#2}%
   \parshape 1 \@tempdima \dimexpr\textwidth-2\@tempdima\relax%
   \itshape}
  {\par\normalfont\hfill--\ \chapquote@author\hspace*{\@tempdima}\par\bigskip}
\makeatother

\newtheorem{Theorem}{Theorem}[section]%Theorem
\newtheorem*{Corollary}{Corollary} %Corollary
\newtheorem*{Lemma}{Lemma}%Lemma
\newtheorem*{Statement}{Statement} % Statement
\theoremstyle{definition}
\newtheorem*{Definition}{Definition} %Definition
\newtheorem*{Proposition}{Proposition}%Proposition

%%%%%%%%%%%%%%%%%%%%%%%%%%%%%%%%%%%%%%%%%%%%%%%%%%%
% First page of book which contains 'stuff' like: %
%  - Book title, subtitle                         %
%  - Book author name                             %
%%%%%%%%%%%%%%%%%%%%%%%%%%%%%%%%%%%%%%%%%%%%%%%%%%%

% Book's title and subtitle
\title{\Huge \textbf{Mathematical Analysis}}
% Author
\author{\textsc{Hechen Hu}}


\begin{document}

\frontmatter
\maketitle
%%%%%%%%%%%%%%%%%%%%%%%%%%%%%%%%%%%%%%%%%%%%%%%%%%%%%%%%%%%%%%%%%%%%%%%%
% Auto-generated table of contents, list of figures and list of tables %
%%%%%%%%%%%%%%%%%%%%%%%%%%%%%%%%%%%%%%%%%%%%%%%%%%%%%%%%%%%%%%%%%%%%%%%%
\tableofcontents

\mainmatter

%%%%%%%%%%%%%%%%
% NEW CHAPTER! %
%%%%%%%%%%%%%%%%
\chapter{General Mathematical Concepts and Notation}

\section{Mathematical Symbols and their meanings}
\begin{tabular}{| l | l |}
	\hline
	Notation &Meaning\\ \hline
	$L\Rightarrow P$ &L implies P \\ \hline
	$L\Leftrightarrow P$ &L is equivalent to P \\ \hline
	$((L\Rightarrow P)\land(\lnot P))\Rightarrow(\lnot L)$ &If P follows from L and P is false, then L is false \\ \hline
	$\lnot((L\Leftrightarrow G)\lor(P\Leftrightarrow G)$ &G is not equivalent either to L or to P \\ \hline
	$A \coloneqq B$ &The Definition of A is B (equality by definition)\\ \hline
	$\square$ &End of proof \\
	\hline
\end{tabular}

\section{Sets and Operations on Them}

\subsection{Naive Set Theory}
1$^0$.\quad A set may consist of any distinguishable objects($x\in A\Rightarrow\exists!x\in A$) \\
2$^0$.\quad A set is unambiguously determined by the collection of objects that comprise it. \\
3$^0$.\quad Any property defines the set of objects having that property($A = \{x|P(x)\}\Rightarrow P(A)$). \\ \\
However, this will lead to Russell's Paradox: \\
Let's have$P(M) \coloneqq M\notin M$ \\
Consider the class $K = \{M|P(M)\}$. If so $K$ is not a set, since whether $P(K)$ is true or false, contradiction arises.\\

\subsection{ZFC: Zermelo-Fraenkel Axioms and Axiom of Choice}
1$^0$.\quad \textbf{(Axiom of Extensionality)} Sets $ A $ and $B$ are equal iff they have the same elements. $ (A=B)\Leftrightarrow(\forall x((x\in A)\Leftrightarrow(x \in B))) $\\
2$^0$. \quad \textbf{(Axiom of Seperation)} To any set $A$ and any property P there corresponds a set $B$ whose elements are those elements of $A$, and only those, having property P(if $A$ is a set, then $B=\{x\in A|P(x)\}$ is also a set). \\
3$^0$. \quad \textbf{(Union Axiom)} For any set $ M $ whose elements are sets there exists a set $\bigcup M$, called the union of $ M $ and consisting of those elements and only those that belong to some element of $ M $ $ (x\in \bigcup M \Leftrightarrow \exists X((X\in M)\land (x\in X))) $ \\
Similarly, the intersection of the set $ M $ is defined as:
\begin{equation}
	\bigcap M  \coloneqq \{x\in \bigcup M | \forall X((X\in M)\Rightarrow (x\in X))\} \nonumber
\end{equation}
4$^0$ \quad \textbf{(Pairing Axiom)} For any sets $ X $ and $ Y $ there exists a set $ Z $ such that $ X $ and $ Y $ are its only elements. \\
5$^0$ \quad \textbf{(Power Set Axiom)} For any set $ X $ there exists a set $ P(X)$ having each subset of $ X $ as an element, and having no other elements. \\
\newline
\begin{Definition}
The \textit{successor} $X^+$ of the set $ X $ is $X^+ = X\cup \{X\}$.\\
\end{Definition} 
\begin{Definition}
An \textit{inductive} set is a set that $\varnothing$ is one of its elements and the successor of each of its elements aso belongs to it.
\end{Definition}
6$^0$ \quad \textbf{(Axiom of Infinity)} There exist inductive sets (Example: $\mathbb{N}_0$).\\ 
7$^0$ \quad \textbf{(Axiom of Replacement)} Let $ F(x,y) $ be a statement(a formula) such that for every $ x_0 \in X$ there exists a unique object $ y_0 $ such that $ F(x_0,y_0) $ is true. Then the objects $ y $ for which there exists an element $ x\in X $ such that $ F(x,y) $ is true form a set.\\
And finally, an axiom that is independent of ZF. \\
\newline
\begin{Definition}
A choice function is a function $ f $, defined on a collection $ X $ of nonempty sets, such that for every set $ A $ in $ X $, $ f(A) $ is an element of $ A $.
\end{Definition}
8$^0$ \quad \textbf{(Axiom of Choice/Zermelo's Axiom)}  For any set $ X$ of nonempty sets, there exists a choice function $ f $ defined on $ X $.$(\forall X[\varnothing \notin X \Rightarrow \exists f: X\mapsto \bigcup X \quad \forall A\in X(f(A)\in A)])$

\subsection{The \textit{Cardinality} of a Set(\textit{Cardinal Numbers})}
\begin{Definition}
The set $ X $ is said to be \textit{equipollent} to the set $ Y $ if there exists a bijective mapping of $ X $ onto $ Y $(then $ X\sim Y $).
\end{Definition}
\begin{Definition}
\textit{Cardinality} is a measure of the number of elements of the set. If $ X \sim Y $, we write $ card X = card Y $.
\end{Definition}
If $ X $ is equipollent to some subset of $ Y $, we say $ card X\leqslant card Y $, thus
\begin{equation}
	(card X \leqslant card Y)\coloneqq \exists Z\subset Y (card X = card Z) \nonumber
\end{equation}
A set is called \textit{finite} if it is not equipollent to any proper subset of itself; otherwise it is called \textit{infinite}. \\
It has the  properties below:\\
\newline
1$^0  \quad (cardX\leqslant card Y)\land(card Y \leqslant card Z)\Rightarrow(card X \leqslant card Z)$.\\
2$^0 \quad (card X \leqslant card Y)\land(card Y \leqslant card X)\Rightarrow(card X = card Y)$(The Schröder–Bernstein theorem).\\
3$^0 \quad \forall X\forall Y(card X \leqslant card Y)\lor(card Y \leqslant card X)$(Cantor's theorem).\\
\newline
We say $ card X < card Y $ if $ (card X \leqslant card Y) \land (card X \neq card Y)$.\\
let $ \varnothing $ be the empty set and $ P(X) $ the set of all subsets(thus, the power set) of the set $ X $. Then:
\begin{Theorem}
	$ card X < card P(X) $
\end{Theorem}
\begin{proof}
The assertion is obvious for the empty set, and we shall assume that $ X\neq \varnothing $.\\
Since $ P(X) $ contains all the one-element subsets of $ X $, $ cardX \leqslant cardP(X)$.\\
Suppose, contrary to the assertion, that there exists a bijective mapping $ f : X \to P(X) $. Let set $ A = \{x\in X:x\notin f(x)\}$ consisting of the elements $ x \in X $ that do not belong to the set $ f(x)\in P(X) $ assigned to them by the bijection. Because $ A \in P(X) $, there exists $ a\in X $ such that $ f(a) = A $. For the element $ a $ the relation $ a \in A $ or $ a \notin A $ is impossible by the definition of $ A $(Similar to Russell's Paradox).
\end{proof}



\subsection{Operations on Sets}
\begin{tabular}{| l | l | l |}
	\hline
	Notation &Meaning &Definition \\ \hline
	$A\subset B$ &$A$ is a subset of $B$ &$\forall x ((x\in A) \Rightarrow(x\in B))$ \\ \hline
	$A = B$ &$A$ equals to $B$ &$(A \subset B)\land(B \subset A)$ \\ \hline
	$\varnothing$ &Empty Set & $\{x|x\neq x\}$ \\ \hline
	$A \cup B$ &The union of $ A $ and $ B $ &$ \{x|x\in A \lor x\in B\} $ \\
	$ A \cap B $ &The intersection of $ A $ and $ B $ &$ \{x|x\in A \land x\in B\} $ \\ \hline
	$ A\setminus B $ &The difference between $ A $ and $ B $ &$ \{x|x\in A \land x\notin B\} $ \\ \hline
	$ C_M A $ &The complement of $A$ in M &$ \{x|x\in M \land x\notin A\} where A\subset M $ \\ \hline
	$ A \times B $ &The Cartesian Product of $ A $ and $ B $ &$ \{(x,y)|x\in A \land y\in B\} $ \\ \hline
	$ A^2 $ &$ A \times A $ & \\
		\hline
\end{tabular}
 \newline \\
In the ordered pair$ z = (x_1,x_2) $ where$ Z = X_1 \times X_2 ,z\in Z,x_1 \in X_1, x_2 \in X_2$, $ x_1 $ is called the \textit{first projection} of the pair $ z $ and denoted  pr$_1 z $ while $ x_2 $ is called the \textit{second projection} of the pair $ z $ and denoted  pr$_2 z $.






\section{Relations and Functions}
\subsection{Definitions of Functions}
\begin{Definition} 
We say that there is a \textit{function} defined on $ X $ with values in $ Y $ if, by virtue of some fule $ f $, to each element $ x\in X $ there corresponds an element $ y \in Y $.
\begin{equation}
f(X) \coloneqq \{y \in Y | \exists x((x\in X)\land(y = f(x)))\} \nonumber
\end{equation}
X is called the \textit{domain of definition} and Y is called \textit{set of values} or \textit{range} of the function.
\end{Definition}
\begin{Definition}
If $ A \subset X$ and $ f:X \to Y $ is a function. We denote by $ f|_A $ the function $ \varphi:A \to Y $ that agrees with $ f $ on $ A $. More precisely, $ f|_A(x) \coloneqq \varphi(x)$ if $ x\in A  $. The function $ f|_A $ is called the restriction of $ f $ to $ A $, and the function $ f:X \to Y $ is called an extension or a continuation of $ \varphi $ to $ X $.
\end{Definition}

We use the term \textit{domain of departure} of the function to denote any set $ X $ containing the domain of a function, and \textit{domain of arrival} to denote any subset of $ Y $ containing its range.

\subsection{Elementry Classification of Mappings}
\begin{Definition}
When a function $ f: X\to Y $ is called a mapping, the value $ f(x) \in Y $ that it assumes at the element $ x \in X$ is usually called the \textit{image} of $ x $. \\
The \textit{image} of a set $ A \subset X $ under the mapping $  f: X\to Y $ is defined as the set
\begin{equation}
f(A) \coloneqq \{y\in Y | \exists x((x\in A)\land(y = f(x))\} \nonumber
\end{equation}
consisting of the elements of $ Y $ that are images of elements of $ A $.
\end{Definition} 

The set
\begin{equation}
	f^{-1}(B)\coloneqq \{x\in X | f(x)\in B\} \nonumber
\end{equation}
consisting of the elements of $ X $ whose images belong to $ B $ is called the \textit{pre-image} (or \textit{complete pre-image}) of the set $ B \subset Y $. \\
\newline
\begin{Definition}
A mapping $  f: X\to Y  $ is said to be \\
\textit{surjective} if $ f(X) = Y $\\
\textit{injective} if $ \forall x_1,x_2 \in X $, $(f(x_1)=f(x_2))\Rightarrow(x_1=x_2)$ holds.\\
\textit{bijective} if it's both surjective and injective.	
\end{Definition}

\begin{Definition}
The inverse mapping of a bijective $ f $ is denoted as
\begin{equation}
f^{-1}:Y\to X \nonumber
\end{equation}
and defined as follows: if $ f(x) = y $, then $ f^{-1}(y) = x $. 
\end{Definition}
Note that the pre-image of a set is defined for any mapping $ f:X\to Y $, even if it is not bijective and hence has no inverse.

\subsection{Composition of Functions and Mutually Inverse Mappings}
\begin{Definition}
For two mappings $ f:X\to Y $ and $ g:Y \to Z $,
\begin{equation}
g \circ f:X \to Z,\quad (g \circ f)(x)\coloneqq g(f(x)) \nonumber
\end{equation}
is called the \textit{composition} of the mapping $ f $ and the mapping $ g $.	
\end{Definition}

If all the terms of a composition $ f_n \circ ... \circ f_1 $ are equal to the same function $ f $, we abbreviate it to $ f^n $.\\
\newline
Function composition is associative, that is,
\begin{equation}
	h \circ (g\circ f) = (h \circ g)\circ f \nonumber
\end{equation}
But in general, $ g\circ f \neq f \circ g $. \\
\newline
\begin{Definition}
The mapping $ f:X \to X $ that assigns each element in $ X $ to itself is called the \textit{identity mapping} on $ X $ and denoted $ e_X $.
\end{Definition} 
\begin{Lemma}
	$	(g\circ f = e_X) \Rightarrow$($g$ is surjective) $\land$ ($f$ is injective).
\end{Lemma}

\begin{proof}
	If $ f:X\to Y$,$ g:Y\to X $, and $ g \circ f = e_X:X\to X $, then
	\begin{equation}
		X = e_X(X) = (g\circ f)(X) = g(f(X))\subset g(Y) \nonumber
	\end{equation}
	and hence $ g $ is surjective. \\
	Further, if$ x_1\in X $ and $ x_2 \in X $, then
	\begin{align}
		(x_1\neq x_2) &\Rightarrow (e_X(x_1)\neq e_X(x_2))\Rightarrow((g\circ f)(x-1)\neq (g\circ f)(x_2))\Rightarrow \nonumber \\
							 &\Rightarrow (g(f(x_1))\neq g(f(x_2))) \Rightarrow(f(x_1)\neq f(x_2)) \nonumber
	\end{align}
	and therefore $ f $ is injective.
\end{proof}

\begin{Proposition}
The mappings $ f :X\to Y $ and $ g:Y\to X $ are bijective and mutually inverse to each other if and only if $ g \circ f = e_X $ and $ f \circ g = e_Y $.
\end{Proposition}


\subsection{Functions as Relations. The Graph of a Function}
\subsubsection{Relations}
\begin{Definition}
A Relation $ R $ is any set of ordered pairs $ (x,y) $. \\
The set $ X $ is called the \textit{domain of definition} of $ R $, and the set $ Y $ is the \textit{range of values}. 
\end{Definition}
Any set containing the domain of definition of a relation is called a \textit{domain of departure} for that relation, and \textit{domain of arrival} is a set that contains the range of values of the relation. \\
\newline
Instead writing $ (x,y)\in R $, we write $ xRy $ and say that $ x $ is \textit{connected with} $ y $ \textit{by the relation $ R $}.\\
If $ R \subset X^2 $, then we say that the relation $ R $ is defined on $ X $.

\subsubsection{Classification of Relations}
\begin{Definition}
An \textit{equivalence relation} is a relation that satisfy the following properties: \\
\newline
$ aRa $ (Reflexivity);\\
$ aRb \Rightarrow bRa $ (Symmetry); \\
$ (aRb)\land (bRc) \Rightarrow aRc $ (Transitivity).\\
\newline
An equivalence relation is denoted by the special symbol $ \sim $. $ a\sim b $ means $ a $ is \textit{equivalent} to $ b $.	
\end{Definition}

\begin{Definition}
A \textit{partial ordering} on a set $ X^2 $ is a relation $ R $ that have the following properties: \\
\newline
$ aRa $ (Reflexivity);\\
$ (aRb)\land (bRc) \Rightarrow aRc $ (Transitivity).\\
$ (aRb) \land (bRa )\Rightarrow (a=b) $ (Anti-symmetry);	
\end{Definition}

We often write $ a\preceq b $ and say that $ b $ \textit{follows} $ a $.
If the condition
\begin{equation}
	\forall a \forall b((aRb)\lor(bRa) \nonumber
\end{equation}
holds in addition to transitivity and anti-symmetry defining a partial ordering relation(this means any two elements of $ X $ is comparable), the relation $ R $ is called an \textit{ordering}, and the set $ X $ is said to be \textit{linearly ordered}.

\subsubsection{Functions and Their Graphs}
\begin{Definition}
	A relation $ R $ is said to be functional if
	\begin{equation}
	(xRy_1)\land(xRy_2)\Rightarrow (y_1=y_2) \nonumber
	\end{equation}
	and it is called a \textit{function}.\\
	$ R\subset X \times Y $ is a \textit{mapping from $ X $ into $ Y $}, or a \textit{function from $ X $ into $ Y $}.\\
\end{Definition}

\begin{Definition}
The \textit{graph} of a function $ f:X \to Y $, is the subset $ \Gamma $ of $ X \times Y $.
\begin{equation}
\Gamma \coloneqq \{(x,y)\in X \times Y | y = f(x)\} \nonumber
\end{equation}
\end{Definition} 


\chapter{The Real Numbers}
\section{The Axiom System and Some General Properties of the Set of Real Numbers}
\subsection{The Axiom System of the Real Numbers}
\subsubsection{Axioms for Addition:} An operation 
\begin{equation}
	+:\Real \times \Real \to \Real \nonumber
\end{equation}
is defined, assgining to each ordered pair $ (x,y) $ of elements $ x,y $ of $ \Real $ a certain element $ x+y \in \Real $. \\
1$_+ $\quad There exists a neutral, or identity element $ 0 $ (called zero) such that 
\begin{equation}
	x+0 = 0+x = x \nonumber
\end{equation} for every $ x \in \Real $. \\
2$ _+ $\quad For every element $ x\in \Real $ there exists an element $ -x\in \Real $ called the \textit{negative of $ x $}.
\begin{equation}
	\forall x \in \Real \Rightarrow (\exists! (-x)\in \Real)\land (x +(-x) = (-x)+x=0) \nonumber
\end{equation}
3$ _+ $\quad The operation $ + $ is associative.
\begin{equation}
	\forall x \forall y \forall z \in \Real \Rightarrow x+(y+z) = (x+y)+z \nonumber
\end{equation}
4$ _+ $\quad The operation $ + $ is commutative.
\begin{equation}
	\forall x \forall y \in \Real \Rightarrow(x+y = y+x) \nonumber
\end{equation}
\newline
\begin{Definition}
A group structure is defined on the set $ G $ -- or $ G $ is a group -- if Axioms 1$_+ $, 2$ _+ $, and 3$ _+ $ holds for an operation defined on this set. The group is called \textit{additive} group if the operation is called addition. When the operation is also commutative, that is, Axiom 4$ _+ $ holds, the group is also called a \textit{commutative group} or an \textit{Abelian group}. 
\end{Definition}
According to Axioms 1$ _+ $ -- 4$ _+ $, $ \Real $ is an additive Abelian group.\\
\subsubsection{Axioms for Multiplication:} An operation 
\begin{equation}
	\bullet:\Real \times \Real \to \Real \nonumber
\end{equation}
is defined, assigning to each ordered pair $ (x,y) $ of elements $ x,y\in \Real $ a certain element $ x \cdot y \in \Real $, called the product of $ x $ and $ y $. \\
1$ _\bullet $\quad There exists a neutral, or identity element $ 1 \in \Real\setminus 0 $ such that
\begin{equation}
	\forall x \in \Real \Rightarrow (x \cdot 1 = 1 \cdot x = x) \nonumber
\end{equation}
2$ _\bullet $\quad  For every element $ x \in \Real\setminus 0 $ there exists an element $ x^{-1}\in \Real $, called the \textit{inverse} or \textit{reciprocal} of $ x $.
\begin{equation}
	\forall x \in \Real \Rightarrow (x\cdot x^{-1} = x^{-1} \cdot x = 1) \nonumber
\end{equation}
3$ _\bullet $\quad The operation $ \bullet $ is associative.
\begin{equation}
	\forall x \forall y \forall z \in \Real \Rightarrow x\cdot(y\cdot z) = (x\cdot y)\cdot z \nonumber
\end{equation}
4$ _\bullet $\quad The operation $ \bullet $ is commutative.
\begin{equation}
	\forall x \forall y \in \Real \Rightarrow x\cdot y = y \cdot x \nonumber
\end{equation}
The set $ \Real \setminus 0 $ is a \textit{multiplicative} group. \\
\newline
Multiplication is distributive with respect to addition.
\begin{equation}
		\forall x \forall y \forall z \in \Real \Rightarrow (x+y)z = xz+yz \nonumber
\end{equation}
\begin{Definition}
If two operations satisfying the Axioms of Addition and Multiplication are defined on a set $ G $, then $ G $ is called a \textit{field}.
\end{Definition}
\subsubsection{Order Axioms}
Between elements of $ \Real $ there is a relation $ \leqslant $ defined, and :\\
0$ _{\leqslant} $\quad $ \forall x \in\Real(x\leqslant x) $ (Reflexivity) \nonumber \\
1$ _{\leqslant} $\quad $ (x \leqslant y)\land (y \leqslant x) \Rightarrow (x=y) $(Anti-Symmetry)\nonumber \\
2$ _{\leqslant} $\quad $ (x \leqslant y) \land (y \leqslant z) \Rightarrow (x \leqslant z) $ (Transitivity)\nonumber \\
3$ _{\leqslant} $\quad $ \forall x \in \Real \forall y \in \Real(x\leqslant y)\lor(y\leqslant x) $\nonumber \\
Thus, the relation $ \leqslant $ is an ordering, and $ \Real $ is linearly ordered.
\subsubsection{The Connection between Addition and Order on $ \Real $}
\begin{equation}
	\forall x \forall y \forall z \in \Real \Rightarrow ((x\leqslant y)\Rightarrow (x +z \leqslant y + z) \nonumber
\end{equation}
\subsubsection{The Connection between Multiplication and Order on $ \Real $}
\begin{equation}
\forall x \forall y \in \Real \Rightarrow ((0\leqslant x)\land (0 \leqslant y)\Rightarrow(0 \leqslant x \cdot y)) \nonumber
\end{equation}
\subsubsection{The Axiom of Completeness(Continuity)}
If $ X $ and $ Y $ are nonempty subsets of $ \Real $ having the property that $ x \leqslant y $ for every $ x \in X $ and every $ y \in Y $, then there exists $ c\in \Real $ such that $ x\leqslant c \leqslant y $ for all $ x \in X $ and $ y \in Y $.
\begin{equation}
	(\forall X \forall Y\subset \Real)\land(X,Y \neq \varnothing)\land(\forall x\in X \forall y \in Y \Rightarrow x \leqslant y)\Rightarrow (\exists c\in \Real \forall x\in X \forall y \in Y (x \leqslant c \leqslant y)) \nonumber
\end{equation}
\newline
\textbf{Any set on which these axioms hold can be considered a \textit{model} of the real numbers.} \\
\newline
\begin{Definition}
An axiom system is said to be \textit{categorical} if it determines an unique mathematical object.
\end{Definition}

\begin{Definition}
If there are two models of independent number systems $ \Real_A $ and $ \Real _B $ that satisfying all the axioms, then a bijective correspondence can be established between these two systems, say $ f:\Real_A \to \Real_B $, preserving the arithmetic operations and the order, that is, 
\begin{align}
f(x+y) &= f(x) + f(y)\nonumber \\
f(x\cdot y) &= f(x) \cdot f(y) \nonumber \\
x\leqslant y &\Leftrightarrow f(x) \leqslant f(y) \nonumber 
\end{align}
and we can say that $ \Real_A $ and $ \Real _B $ are \textit{isomorphic} and the mapping $ f $ is called an \textit{isomorphism}.
\end{Definition} 

\begin{Theorem}
	The Axiom System of The Real Numbers is categorical.
\end{Theorem}

\subsection{Some General Algebraic Properties of Real Numbers}
\subsubsection{Consequences of the Addition Axioms}
1$ ^0 $\quad There is only one zero in the set of real numbers.\\
2$ ^0 $\quad Each element of the set of real numbers has a unique negative.\\
3$ ^0 $\quad In $ \Real $ the equation
\begin{equation}
	a + x = b \nonumber
\end{equation}
has the unique solution 
\begin{equation}
	x = b+(-a) \nonumber
\end{equation}

\subsubsection{Consequences of the Multiplication Axioms}
1$ ^0 $\quad There is only one multiplicative unit in the real numbers.\\
2$ ^0 $\quad For each $ x \neq 0 $ there is only one reciprocal $ x^{-1} $.\\
3$ ^0 $\quad For $ a\in \Real\setminus 0 $, the equation $ a\cdot x = b $ has the unique solution $ x = b\cdot a^{-1} $.

\subsubsection{Consequences of the Axiom Connecting Addition and Multiplication}
1$ ^0 $\quad For any $ x\in \Real $ 
\begin{equation}
	x\cdot 0 = 0 \cdot x = 0 \nonumber
\end{equation}
2$ ^0 $\quad $ (x\cdot y = 0)\Rightarrow(x=0)\lor(y=0) $. \\
3$ ^0 $\quad For any $x \in \Real  $
\begin{equation}
	-x = (-1)\cdot x \nonumber
\end{equation}
4$ ^0 $\quad For any $ x\in\Real $
\begin{equation}
	(-1)\cdot(-x)=x \nonumber
\end{equation}
5$ ^0 $\quad For any $ x\in \Real $
\begin{equation}
	(-x)\cdot(-x) = x\cdot x \nonumber
\end{equation}

\subsubsection{Consequences of the Order Axioms}
1$ ^0 $\quad For any $ x $ and $ y $ in $ \Real $ precisely one of the following relations holds:
\begin{equation}
	x < y, \quad x = y,\quad x>y \nonumber
\end{equation}
2$ ^0 $\quad For any $ x,y,z\in\Real $
\begin{align}
	(x<y)\land(y\leqslant z) &\Rightarrow (x<z) \nonumber \\	
	(x\leqslant y)\land(y< z) &\Rightarrow (x<z) \nonumber
\end{align}

\subsubsection{Consequences of the Axiom Connecting Order with Addition and Multiplication}
1$ ^0 $\quad For any $ x,y,z,w\in \Real $
\begin{align}
	(x<y)&\Rightarrow(x+z)<(y+z)\nonumber \\
	(0<x)&\Rightarrow(-x<0) \nonumber\\
	(x\leqslant y)\land(z\leqslant w)&\Rightarrow (x+z)\leqslant(y+w)\nonumber \\
	(x< y)\land(z<w)&\Rightarrow (x+z)<(y+w)\nonumber \\
\end{align}
2$ ^0 $\quad If $ x,y,z \in \Real $,then
\begin{align}
	(0<x)\land(0<y)&\Rightarrow(0<xy) \nonumber \\
	(x<0)\land(y<0)&\Rightarrow(0<xy) \nonumber \\
	(x<0)\land(0<y)&\Rightarrow(xy<0) \nonumber \\
	(x<y)\land(0<z)&\Rightarrow(xz<yz)\nonumber \\
	(x<y)\land(z<0)&\Rightarrow(yz<xz)\nonumber
\end{align}
3$ ^0 $\quad $0<1$. \\
4$ ^0 $\quad $ (0<x)\Rightarrow(0<x^{-1}) $ and $ (0<x)\land(x<y)\Rightarrow(0<y^{-1})\land(y^{-1}<x^{-1}) $.

\subsection{The Completeness Axiom and the Existence of a Least Upper(or Greatest Lower) Bound of a Set of Numbers}
\begin{Definition}
	A set $ X\subset \Real $ is said to be \textit{bounded above} (resp. \textit{bounded below}) if there exists a number $ c\in\Real $ such that $ x\leqslant c $(resp.$ c\leqslant x $) for all $ x\in X $.
\end{Definition}
\begin{Definition}[Maximal and Minimal Elements]
	\begin{align}
	(a= maxX)&\coloneqq(a\in X\land \forall x\in X(x\leqslant a))\nonumber \\
	(a= minX)&\coloneqq(a\in X\land \forall x\in X(a\leqslant x))\nonumber \\
	\end{align}
	It follows from the order Axiom 1$ _\leqslant $ that if there is a maximal (resp. minimal ) element in a set of numbers, it is the only one. \\
	However, not every set, not even every bounded set, has a maximal or minimal element(e.g. $ X = \{x\in R|0\leqslant x<1\} $).
\end{Definition}
\begin{Definition}
	The smallest number that bounds a set $ X\subset \Real $ from above is called the \textit{least upper bound}(or the \textit{eact upper bound}) of $ X $ and denoted $ sup X $("the supremum of $ X $").
	\begin{equation}
	(s=sup X)\coloneqq \forall x\in X((x\leqslant s)\land(\forall s^\prime<s\exists x^\prime \in X (s^\prime < x^\prime)))\nonumber
	\end{equation}
	Similarly, the greatest lower bound of $ X $, $ inf X $("the infimum of $ X $) can be defined as:
	\begin{equation}
	(i=inf X)\coloneqq \forall x\in X((i\leqslant x)\land(\forall i^\prime>i\exists x^\prime \in X (x^\prime < i^\prime)))\nonumber
	\end{equation}
\end{Definition}\textbf{}
\begin{Lemma}
	(The least upper bound principle) Every nonempty set of real numbers that is bounded from above has a unique least upper bound.
\end{Lemma}
\begin{proof}
	Since we already know that the minimal element of a set of numbers is unique(the relation $ \leqslant $ is Anti-Symmetric), we need only verify that the least upper bound exists.\\
	Let $ X\subset \Real $ be a given set and $ Y = \{y\in \Real|\forall x\in X (x \leqslant y)\} $. We know that $ X \neq \varnothing $ and $ Y \neq \varnothing $. Then, by the completeness axiom there exists $ c \in R $ such that $ \forall x \in X \forall y \in Y (x\leqslant c \leqslant y) $. Because $ c $ is  greater than all the elements in $ X $ and smaller than all the elements in $ Y $, we can see that $ c \in Y$ and $ c = min Y $. $ \forall c^\prime < c $ we have $ c^\prime \in X $, and according to the completeness axiom of real numbers, there exists some $ x^\prime \in X$ such that $ c^\prime \leqslant x^\prime \leqslant c $. Thus $ c $ is $ sup X $.
\end{proof}
The existence of greatest lower bound is analogous with the existence of least upper bound, so
\begin{Lemma}
	($ X $ is nonempty and bounded below)$ \Rightarrow(\exists! inf X) $.
\end{Lemma}









\section{The Most Important Classes of Real Numbers and Computational Aspects of Operations with Real Numbers}
\subsection{The Natural Numbers and the Principle of Mathematical Induction}
\subsubsection{Definition of the Set of Natural Numbers}
\begin{Definition}
	A set $ X\subset \Real $ is \textit{inductive} if for each number $ x\in X $, it also contains $ x+1 $.
\end{Definition}

\begin{Definition}
	The set of \textit{natural numbers} is the smallest inductive set(has the cardinality of $ \aleph_0 $) containing $ 1 $, that is, the intersection of all inductive sets that contain $ 1 $.
\end{Definition}

\subsubsection{The Principle of Mathematical Induction}
\begin{Definition}
	If a subset $ E $ of the set of natural numbers $ \Natural $ is such that $ 1 \in E $ and together with each number $ x\in E $, the number $ x+1 $ also belongs to $ E $, then $ E=\Natural $.
	\begin{equation}
		(E\subset \Natural)\land(1\in E)\land(x\in E \Rightarrow (x+1)\in E)\Rightarrow E = \Natural \nonumber
	\end{equation}
\end{Definition}
Some properties of the natural numbers:\\
1$ ^0 $\quad The sum and product of natural numbers are natural numbers.\\
2$ ^0 $\quad $ (n\in \Natural)\land(n\neq 1)\Rightarrow ((n-1)\in \Natural) $. \\
3$ ^0 $\quad For any $ n\in \Natural $ the set $ \{x\in \Natural | n <x\} $ contains a minimal element, namely
\begin{equation}
	min\{x\in \Natural | n <x\} = n+1 \nonumber
\end{equation}
4$ ^0 $\quad $ (n\in\Natural)\land(m \in \Natural)\land(n<m)\Rightarrow(n+1\leqslant m) $.\\
5$ ^0 $\quad The number $ (n+1)\in\Natural $ is the immediate successor of the number $ n\in \Natural $; that is, if $ n\in \Natural $, there are no natural numbers $ x $ satisfying $ n-1<x<n $. \\
6$ ^0 $\quad If $ n\in \Natural $ and $ n\neq 1 $, then $ (n-1)\in \Natural $ and $ (n-1) $ is the immediate predecessor of $ n\in \Natural $; that is , if $ n\in \Natural $, there are no natural numbers $ x $ satisfying $ n-1<x<n $.\\ 
7$ ^0 $\quad In any nonempty subset of $ \Natural $ there is a minimal element.\\
\begin{proof}
	Let $ M \subset \Natural $. \\
	Case 1: For $ 1\in M $, We'll have $ minM $ = 1, since $ \forall n \in \Natural (1\leqslant n)$. \\
	Case 2: For $ 1 \notin M $, we find a set $ E $ such that $ 1 \in E = \Natural \setminus M $. If $ n $ is $ max E $, then $ \forall e \in E \Rightarrow e \leqslant n $. However, $ (n+1) \notin E $ because $ \forall n \in \Natural\Rightarrow(n+1)>n $, and thus $ (n+1)\in M $ (If such $ n $ do not exist, then $ E $  which contains $ 1 $ is not bounded from above, and we can see that $ (n\in E) \Rightarrow ((n+1)\in E) $. By the principle of induction, $ E = \Natural $. But this is impossible, since $ \Natural \setminus E = M \neq \varnothing $). \\
	Therefore, $ (\forall x\in M)\Rightarrow (n\nleqslant x\nleqslant (n+1) ) \Rightarrow  (min M = (n+1)) $ .
\end{proof}

\subsection{Rational and Irrational Numbers}
\subsubsection{The Integers}
\begin{Definition}
	The union of the set of natural numbers, the set of negatives of natural numbers, and zero is called the set of \textit{integers} and is denoted $ \Zahlen $.
\end{Definition}
The addition and multiplication of integers do not lead outside of $ \Zahlen $. Thus, $ \Zahlen $ is an Abelian group with respect to addition, but $ \Zahlen $ nor $ \Zahlen \setminus 0 $ is a group with respect to multiplication. \\
\begin{Theorem}[The fundamental theorem of arithmetic]
	Each natural number admits a representation as a product 
	\begin{equation}
		n = p_1 \cdots p_k \nonumber
	\end{equation}
	where $ p_1,..., p_k $ are prime numbers. This representation is unique except for the order of the factors.
\end{Theorem}
\begin{Corollary}
	$ 1 $ is not a prime number, since the product representation can contain infinite numbers of $ 1 $, which makes the representation not unique.
\end{Corollary}

\subsubsection{The Rational Numbers}
\begin{Definition}
	Numbers of the form $ m \cdot n^{-1} $, where $ m,n \in \Zahlen $, are called rational numbers. The set of rational numbers is denoted as $ \Quoziente $.
\end{Definition}

\subsubsection{The Irrational Numbers}
\begin{Definition}
	The real numbers that are not rational are called \textit{irrational}. The set of irrational numbers is $ \Real \setminus \Quoziente $.
\end{Definition}
\begin{Statement}
	$ \sqrt{2} $ is irrational.
\end{Statement}

\begin{proof}
	Let $ X $ and $ Y $ be the sets of positive real numbers such that $ \forall x \in X (x^2<2) $ and $ \forall y \in Y (2<y^2) $. $ X \neq \varnothing $ and $ Y \neq \varnothing $, since $ 1 \in X $ and $ 2 \in Y $. \newpara
	Further, $ (x<y)\Leftrightarrow(x^2<y^2) $, and by the completeness axiom there exists $ s \in \Real $ such that $\forall x \in X \forall y \in Y  (x \leqslant s \leqslant y )$. The next step is to show that $ s^2=2 $. \newpara
	Case 1:$ s^2<2 $. Then we can see, for example, the number $ s+ \frac{2-s^2}{3s} $, which is larger than $ s $, would have a square less than $ 2 $. Indeed, we know that $ 1 \in X $, $ 1^2\leqslant s^2 <2 $, and $ 0< \Delta \coloneqq 2 - s^2 \leqslant 1 $. It follows that 
	\begin{equation}
		(s+\frac{\Delta}{3s})^2 = s^2 + 2\cdot \frac{\Delta}{3s} + (\frac{\Delta}{3s})^2 < s^2 + 3s\cdot \frac{\Delta}{3s} = s^2 + \Delta = 2 \nonumber
	\end{equation}
	Therefore $ (s+ \frac{\Delta}{3s})\in X $, and this contradicts to the fact that $ \forall x \in X (x\leqslant s) $. \newpara
	Similarly, we can prove that $ s^2 \ngtr 2 $ by consider the number $ s -  \frac{\Delta}{3s}$ and $ 0<\Delta \coloneqq s^2 -2 <3 $, and we have $ s^2=2 $. \newpara
	Finally, now we'll show that $ s \notin \Quoziente $. Let;s suppose the contrary that $\exists m \in \Natural \exists n \in \Natural (s=\frac{m}{n}) $ and $ m $ as well as $ n $ is a prime number. Because $ m $ and $ n $ are both prime, we can see that the only common factor between them is $ 1 $. But
	\begin{align}
		\frac{m}{n} &= \sqrt{2} \nonumber \\
		(\frac{m}{n})^2&= 2 \nonumber \\
		m^2 &= n^2 \cdot 2 \nonumber \\
	\end{align}
	Hence $ m $ is even. Now let $ m = 2k $, and $ 2 k^2 = n^2 $. Also, $ n $ has to be divisible by $ 2 $. This contradicts with the fact that the only common factors between $ m $ and $ n $ is 1.
\end{proof}
\begin{Definition}(Algebraic Numbers and Transcendental Numbers)
	A real number is called \textit{algebraic} if it is the root of an algebraic equation
	\begin{equation}
		a_0x^n + \cdots + a_{n-1}x + a_n = 0 \nonumber
	\end{equation}
	with rational coefficients. Otherwise, it is called \textit{transcendental}.
\end{Definition}

\subsection{The Principle of Archimedes}
1$ ^0 $\quad Any nonempty subset of natural numbers that is bounded from above contains a maximal element. \\
 \begin{Corollary}
 	The set of natural numbers is not bounded above.
 \end{Corollary}
2$ ^0 $\quad Any nonempty subset of the integers that is bounded from above(resp. from below) contains a maximal element(resp. minimal element). \\
3$ ^0 $\quad The set of integers is unbounded above and unbounded below. \\
\begin{Theorem}[The principle of Archimedes]
	For any fixed positive number $ h $ and any real number $ x $ there exists a unieuq integer $ k $ such that $ (k-1)h\leqslant x \leqslant kh $
\end{Theorem}
\begin{proof}
	The set $ \{n \in \Zahlen | \frac{x}{h}<n\} $ is not empty, since we can always find an integer that is greater than $ \frac{x}{h} $ for any value of $ x $ and $ h $. This set is also bounded below and contains a minimal element $ k $. We can see that $ (k-1) \leqslant \frac{x}{h} < k $. These inequalities are equivalent to the principle of Archimedes because $ h >0 $. The uniqueness of $ k $ can be derived from the uniqueness of the minimal element of a set of numbers.
\end{proof}

\begin{Corollary}
	For any positive number $ \varepsilon $ there exists a natural number $ n $ such that $ 0 < \frac{1}{n}<\varepsilon $.
\end{Corollary}
\begin{proof}
	By the principle of Archimedes there exists $ n \in \Zahlen $ such tha $ 1 < \varepsilon \cdot n $. Since $ 0<1 $ and $ 0<\varepsilon $, we have $ 0<n $. Thus $ n\in \Natural $ and $ 0 < \frac{1}{n}<\varepsilon $.
\end{proof}

\begin{Corollary}
	If the number $ x \in \Real $ is such that $ 0 \leqslant x $ and $ \forall n \in \Zahlen (x < \frac{1}{n} )$ , then $ x=0 $.
\end{Corollary}

\begin{Corollary}
	For any numbers $ a,b \in \Real $ such that $ a < b $ there is a rational number $ r \in \Quoziente $ such that $ a<r<b $.
\end{Corollary}
\begin{proof}
	According to what we had proved, we can choose $ n \in \Natural $ such that $ 0<\frac{1}{n} <(b-a) $. Then by the principle of Archimedes, there exists an integer $ m\in \Zahlen $ and $ \frac{m-1}{n} \leqslant a < \frac{m}{n} $. Hence the relationship $ b < \frac{m}{n} $ is impossible, since then we'll have $ \frac{m-1}{n} \leqslant a < b \leqslant \frac{m}{n} $. Now we substract each side by $ a $, and $ (b-a) \leqslant \frac{m}{n} -a $. Because $ \frac{m-1}{n} \leqslant a $, it is obvious that $ (b-a) \leqslant \frac{m}{n} - \frac{m-1}{n}  \Rightarrow (b-a) \leqslant \frac{1}{n}$, which contradicts with the fact that $ \frac{1}{n}< b-a $. We can choose $ r = \frac{m}{n} \in \Quoziente $ and $ a< \frac{m}{n} <b $.
\end{proof}
\begin{Corollary}
	For any number $ x \in \Real $ there exists a unique integer $ k \in \Zahlen $ such that $ k \leqslant x < k+1 $.
\end{Corollary}
\begin{proof}
	Replace $ h $ with $ 1 $ in the principle of Archimedes.
\end{proof}
The number $ k $ just mentioned is denoted $ [x] $ and is called the \textit{integer part} of x. The quantity $ \{x\}\coloneqq x - [x] $ is called the \textit{fractional part} of x. Thus $ x = [x] + \{x\} $, and $ \{x\} \geqslant 0 $.


\subsection{Miscellaneous}
\begin{Theorem}[Triangle Inequality]
	$ |a+b|\leqslant |a|+|b| $ holds for all $ a,b \in \Real $.
\end{Theorem}
\begin{proof}
	We know that $ |x| = max\{x,-x\}  $ and $ \pm x \leqslant |x| $. Thus
	\begin{align}
	a+b &\leqslant |a| + b \leqslant |a| + |b| \nonumber \\
	-a-b &\leqslant |a| - b \leqslant |a| + |b| \nonumber \\
	a-b & \leqslant |a| - b \leqslant |a| + |b| \nonumber
	\end{align}
\end{proof}
By the principle of induction, we can prove the following theorem.
\begin{Theorem}
	The inequality
	\begin{equation}
	|x_1+ \cdots x_n| \leqslant |x_1| + \cdots |x_n| \nonumber
	\end{equation}
	holds and equality holds if $ \forall n \in \Natural(x_n \leqslant 0 )\lor \forall n \in \Natural (0 \leqslant x_n) $.
\end{Theorem}




\begin{Definition}
	An open interval containing the point $ x \in \Real $ will be called a \textit{neighborhood} of this point. The interval $ (x- \delta, x+ \delta) $ is the \textit{$ \delta $-neighborhood} about $ x $.
\end{Definition}


\subsubsection{Estimation for errors in arithmetic operations}
\begin{Definition}
	If $ x $ is the exact value of a quantity and $ \tilde{x} $ is a known approximation to the quantity, the numbers
	\begin{equation}
		\Delta(\tilde{x}) \coloneqq |x-\tilde{x}|  \nonumber
	\end{equation}
	and 
	\begin{equation}
		\delta(\tilde{x}) \coloneqq \frac{\Delta(\tilde{x}) }{|\tilde{x}|}  \nonumber
	\end{equation}
	are called respectively the \textit{absolute} and \textit{relative} error of approximation by $ \tilde{x} $. The relative error is not defined when $ \tilde{x}=0 $.
\end{Definition}

\begin{Proposition}
	If
	\begin{equation}
		|x-\tilde{x}| = \Delta(\tilde{x}) ,\quad \quad |y-\tilde{y}| = \Delta(\tilde{y}) ,\nonumber
	\end{equation}
	then
	\begin{align}
		\Delta(\tilde{x}+\tilde{y}) &\coloneqq |(x+y)-(\tilde{x}+\tilde{y})| \leqslant \Delta(\tilde{x}) + \Delta(\tilde{y}), \nonumber \\
		\Delta(\tilde{x}\cdot \tilde{y}) &\coloneqq |(x\cdot y)-(\tilde{x}\cdot \tilde{y})| \leqslant |\tilde{x}|\Delta(\tilde{y}) + |\tilde{y}|\Delta(\tilde{x})+ \Delta(\tilde{x}) \cdot \Delta(\tilde{y}), \nonumber
	\end{align}
	if, inaddition
	\begin{equation}
		y \neq 0, \quad \quad \tilde{y} \neq 0, \quad \quad \delta(\tilde{y}) = \frac{\Delta(\tilde{y})}{|\tilde{y}|} < 1 \nonumber
	\end{equation}
	then
	\begin{equation}
		\Delta(\frac{\tilde{x}}{\tilde{y}}) \coloneqq |\frac{x}{y} - \frac{\tilde{x}}{\tilde{y}}| \leqslant \frac{|\tilde{x}|\Delta(\tilde{y}) + |\tilde{y}|\Delta(\tilde{x})}{\tilde{y}^2} \cdot \frac{1}{1-\delta(\tilde{y})}\nonumber
	\end{equation}
\end{Proposition}

\begin{proof}
	Let $ x = \tilde{x} + \alpha $ and $ y = \tilde{y} + \beta $. Thus, $ |\alpha| = \Delta(\tilde{x}) $ and $ |\beta| = \Delta(\tilde{y}) $Then
	\begin{align}
		\Delta(\tilde{x}+\tilde{y}) &= |(x+y)-(\tilde{x}+\tilde{y})| = |\alpha + \beta| \leqslant |\alpha| + |\beta| = \Delta(\tilde{x}) + \Delta(\tilde{y}) \nonumber \\
		\Delta(\tilde{x} \cdot \tilde{y}) & = |(x\cdot y)-(\tilde{x}\cdot \tilde{y})| = |(\tilde{x} + \alpha)(\tilde{y} + \beta)- \tilde{x} \cdot \tilde{y}|	 = \nonumber\\
		&= |\tilde{x}\beta + \tilde{y} \alpha + \alpha \beta| \leqslant |\tilde{x}||\beta| + |\tilde{y} ||\alpha |+ |\alpha ||\beta| = \nonumber\\
		&= |\tilde{x}|\Delta(\tilde{y}) +  |\tilde{y}|\Delta(\tilde{x}) + \Delta(\tilde{x}) \cdot \Delta(\tilde{y}) \nonumber \\
		\Delta(\frac{\tilde{x}}{\tilde{y}}) &= |\frac{x}{y} - \frac{\tilde{x}}{\tilde{y}}| = |\frac{x\tilde{y}-y\tilde{x}}{y\tilde{y}}| = \nonumber \\
		&=|\frac{(\tilde{x} + \alpha)\tilde{y} - (\tilde{y} + \beta)\tilde{x}}{\tilde{y}^2}| \cdot |\frac{1}{1+\beta/\tilde{y}}| \leqslant  \frac{|\tilde{x}||\beta|+|\tilde{y}||\alpha|}{\tilde{y}^2} \cdot \frac{1}{1-\delta(\tilde{y})} = \nonumber \\
		&= \frac{|\tilde{x}|\Delta(\tilde{y}) + |\tilde{y}|\Delta(\tilde{x})}{\tilde{y}^2} \cdot \frac{1}{1-\delta(\tilde{y})}\nonumber
	\end{align}
\end{proof}

These statements imply that 
\begin{align}
	\delta(\tilde{x}+\tilde{y})& \leqslant \frac{\Delta(\tilde{x}+\Delta(\tilde{y}))}{|\tilde{x}+\tilde{y}|} \nonumber \\
	\delta(\tilde{x}\cdot \tilde{y}) &\leqslant \delta(\tilde{x})+ \delta(\tilde{y}) + \delta(\tilde{y}) \cdot \delta(\tilde{y}) \nonumber \\
	\delta(\frac{\tilde{x}}{\tilde{y}})& \leqslant \frac{\delta(\tilde{x})+\delta(\tilde{y})}{1-\delta(\tilde{y})} \nonumber
\end{align}









\subsubsection{The Positional Computation System}
\begin{Lemma}
	If a number $ q>1 $ is fixed, then for every positive number $ x \in
	 \Real $ there exists a unique integer $ k \in \Zahlen $ such that
	 \begin{equation}
	 	q^{k-1} \leqslant x < q^k \nonumber
	 \end{equation}
\end{Lemma}
\begin{proof}
	We first verify that the set of numbers of the form $ q^k $, $ k \in \Natural $ is not bounded above. Suppose the contrary, we'll have a least upper bound such that there exists some $ m \in \Natural $ such that $ q^m < s $. Also, we can see that $ \frac{s}{q}<q^m $(If this is not the case, then we can have $ q^m \leqslant \frac{s}{q} \Rightarrow q^{m+1} \leqslant s$, which makes the biggest element in this set be $ q^{m+1} $, and this is impossible, since $ q^m $ must be the biggest element if the least upper bound is $ s $). Here we'll have $ \frac{s}{q}<q^m\leqslant s \Rightarrow s< q^{m+1} $, so $ s $ could not be the east upper bound of the set. \newpara
	Because $ 1 < q $, $ \forall m,n \in \Zahlen \land (m<n)\Rightarrow  q^m < q^n$. We already show that this set is not bounded from above, so $ \forall c \in \Real \Rightarrow \exists N \in \Natural(\forall n>N(c<q^n)) $(or $ c $ will be the upper bound of this set). \newpara
	Now let's set $ c = \frac{1}{\varepsilon} $ and $ M = N $, it follows that $ \forall \varepsilon>0 (\exists M \in \Natural \forall n > M(\frac{1}{q^m}< \varepsilon)) $. \newpara
	The set $ K \subset \Zahlen $ that $ K\{m|(0<x)\land(x<q^m)\} $ is bounded below(because when $ x = \frac{1}{\varepsilon} $, for $ m <M $ we will have $ q^m<x $). Therefore if the minimal element is denoted as $ k $, for this integer $ x $ it is obvious that $ q^{k-1}<x<q^k $. \newpara
	Next we have to prove the uniqueness of such integer $ k $. For $ m,n\in \Zahlen \Rightarrow ((m<n)\Rightarrow(m \leqslant n-1)$, and $ q >1 \Rightarrow(q^m \leqslant q^{n-1}) $. Thus, if $ m $ and $ n $ are both the minimal element of the set $ K $, it can be derived that $ q^{m-1}\leqslant x < q^m $ and $ q^{n-1} \leqslant x < q^n $, which imply $ q^{n-1} \leqslant x < q^m $, are incompatible if $ m \neq n $.
\end{proof}

\begin{Definition}
	The number $ p $ satisfying the Lemma($ p=k $) is called the \textit{order of} $ x $ in the base $ q $ or(when $ q $ is fixed) simply the \textit{order} of $ x $.
\end{Definition}
By the principle of Archimedes, $ \exists! \alpha_p(\alpha_p q^p \leqslant x < (\alpha_p+1)q^p) $. \\
We can use $ r_n $, a sequence of numbers, can be used to approximate some real number $ x $. Take $ \alpha_p, \alpha_{p-1}, ...., \alpha_{p-n}... $ from the set $ \{0,1,...,q-1\} $ then 
\begin{equation}
	r_n = \alpha_p q^p + \cdot \cdot\cdot \alpha_{p-n} q^{p-n} \nonumber
\end{equation}
and such that
\begin{equation}
	r_n \leqslant x < r_n + \frac{1}{q^{n-p}} \nonumber
\end{equation}

The set $ \{0,1,...,q-1\} $ is all the digits under base $ q $ while the power of $ q $ is the order of this corresponding digit.

\section{Basic Lemmas Connected with the Completeness of the Real Numbers}
\subsection{The Nested Interval Lemma(Cauchy-Cantor Principle)}

\begin{Definition}
	A function $ f : \Natural \to X$ of a natural-number argument is called a \textit{sequence} or a \textit{sequence of elements} of $ X $. $ f(n) $ corresponding to the number $ n \in \Natural $ is often denoted $ x_n $ and called the $ n $th term of the sequence.
\end{Definition}
\begin{Definition}
	Let $ X_1,X_2,...,X_n... $ be a sequence of sets. If $ X_{n+1}\subset X_n $ for all $ n \in \Natural $, we say the sequence is \textit{nested}.
\end{Definition}

\begin{Lemma}[Cauchy-Cantor]
	For any nested sequence $ I_1 \supset I_2 \supset \cdots \supset I_n \supset \cdots $ of closed intervals, there exists a point $ c\in \Real $ belonging to all of these intervals. \newpara
	If in addition it is known that for any $ \varepsilon >0 $ there is an interval $ I_k $ such that $ |I_k|<\varepsilon $, then $ c $ is the unique point common to all the intervals.
\end{Lemma}
\begin{proof}
	Lets find two sets in this sequence, denoted as $ M=[a_m,b_m] $ and $ N = [a_n,b_n] $ where $ m,n \in \Natural $. Suppose $ n <m $, and we obtain $ a_m \leqslant b_n $. Thus the two numerical sets $ A=\{a_m\} $ and $ B = \{b_n\} $ satisfy the axiom of completeness. So $ \exists c \in \Real (a_m\leqslant c \leqslant b_n) $. This also means that $ \forall n \in \Natural (a_n\leqslant c \leqslant b_n) $. Therefore this point $ c $ belongs to all the intervals. \newpara
	Now let $ c_1 $ and $ c_2 $ be two points having this property. Without loss of generality, let $ c_1 < c_2 $, then $ \forall n \in \Natural(a_n \leqslant c_1 < c_2 \leqslant b_n) \Rightarrow(0< c_2 - c_1 < b_n - a_n)$, and the length of any interval in this sequence cannot be less than $ c_2 - c_1 $. Hence if there are intervals of arbitrarily small length in the sequene, their common point is unique.
\end{proof}

\subsection{The Finite Covering Lemma(Borel-Lebesgue Principle, or Heine-Borel Theorem)}
\begin{Definition}
	A system $ S = \{X\} $ of sets $ X $ is said to \textit{cover} a set $ Y $ if $ Y \subset \bigcup_{X \in S}X $.
\end{Definition}
A subset of $ S $ that is also a system of sets will be called a \textit{subsystem} of $ S $.

\begin{Lemma}[Borel-Lebesgue]
	Every system of open intervals covering a closed interval contains a finite subsystem that covers the closed interval.
\end{Lemma}
\begin{proof}
	Let $ S = \{U\} $ be a system of open intervals $ U $ that cover the closed interval $ [a,b] = I_1 $. If the interval $ I_1 $ could not be covered by a finite set of intervals of the system $ S $, then, dividing $ I_1 $ into two halves, we could find that at least one of the two halves, which we denoted by $ I_2 $, does not admit a finite covering. We now repeat this procedure with the interval $ I_2 $, and so on. \newpara
	In this way a nested sequence $ I_1 \supset I_2 \supset \cdots \supset I_n \supset \cdots $ of closed intervals arises,none of which admit a covering by a finite subsystem of $ S $. Since the length of the interval $ |I_n|=|I_1|\cdot 2^{-n} $, the sequence $ \{I_n\} $ contains intervals of arbitrarily small length(The second part of \textbf{Cauchy-Contor Principle}), and thus there exists a unique point $ c $ belonging to all the intervals $ I_n,n\in \Natural $. Since $ c \in I_1 = [a,b] $, there exists an open interval $ (\alpha, \beta) = U \in S $ containing $ c $. Let $ \varepsilon = min \{c-\alpha, \beta - c\} $. In the sequence just constructed, we find an interval $ I_n $ such that $ |I_n|<\varepsilon $. Since $ c \in I_n $ and $ |I_n|<\varepsilon $, we conclude that $ I_n \subset (\alpha, \beta) $. But this contradicts the fact that the interval $ I_n $ cannot be covered by a finite set of intervals from the system.
\end{proof}


\subsection{The Limit Point Lemma(Bolzano-Weierstrass Principle)}

\begin{Definition}
	A point $ p \in \Real $ is a \textit{limit point} of the set $ X \subset \Real $ if every neighborhood of the point contains an infinite subset of $ X $.
\end{Definition}
Examples:\\
If $ X = \{\frac{1}{n}\in \Real | n\in \Natural\} $, the only limit point of $ X $ is the point $ 0 \in \Real $. \\
For an open interval $ (a,b) $ every point of the closed interval $ [a,b] $ is a limit point, and there are no others. \\
For the set $ \Quoziente $ every point of $ \Real $ is a limit point; for, as we know, every open interval of the real numbers contains rational numbers.
\begin{Lemma}[Bolzano-Weierstrass]
	Every bounded infinite set of real numbers has at least one limit point.
\end{Lemma}

\begin{proof}
	Let $ X $ be the given subset of $ \Real $. It follows from the definition of boundedness that $ X $ is contained in some closed interval $ I \subset \Real $. The next step is to show that at least one point of $ I $ is a limit point of $ X $. \newpara
	If, suppose the contrary, each point $ x \in I $ would have a neighborhood $ U(x) $ containing either no points of $ X $ or at most a finite number. The totality of such neighborhoods $ \{U(x)\} $ constructed for the points $ x \in I $ forms a covering of $ I $ by open intervals $ U(x) $. By the finite covering Lemma(Borel-Lebesgue) we can extract a system $ U(x_1), ..., U(x_n) $ of open intervals that cover $ I $. But, since $ X \subset I $, this same system also covers $ X $. However, there are only finitely many points of $ X $ in $ U(x_i) $(the definition of $ U(x) $), and hence only finitely many in their union. That is, $ X $ is a finite set. This contradiction completes the proof.
\end{proof}



\section{Countable and Uncountable Sets}

\begin{Definition}
	A set $ X $ is \textit{countable} if it is equipollent with the set $ \Natural $ of natural numbers, that is, $ card X = card \Natural $.
\end{Definition}

\begin{Proposition}
 An infinite subset of a countable set is countable.
\end{Proposition}
\begin{proof}
	Let's consider a countable set $ E $. There is a minimal element of $ E_1\coloneqq E $, which we assign to $ 1 \in \Natural $ and denote $ e_1 \in E $. $ E $ is infinite, so $ E_2 \coloneqq E \setminus e_1 $ is not empty. Following the principle of induction, we can construct a injective mapping from $ \{1,2...\} $ to $ \{e_1,e_2,...\} $. \newpara
	Now we have to prove that this mapping is also surjective. Suppose the contrary, that an element $ e \in E$ does not have a natural number assigned to it. The set $ K=\{n \in E | n \leqslant e\} $ is finite, since it's a subset of $ \Natural $ bounded both from below and above. According to our previous construction, we assign $ 1 $ to $ minK $, denoted as $ e_1 $, and we can acquire a sequence $ e_1, e_2,...e_{k=cardK} $. But $ e_{k=cardK} $ is $ maxK $, and because $ e\in K \land (\forall n\in K (n\leqslant e))$, $ e = maxK$. Therefore $ e = e_k $, or otherwise it will contradict the uniqueness of maximal element.
\end{proof}

\begin{Proposition}
	The Union of the sets of a finite or countable system of countable sets is also a countable set.
\end{Proposition}
\begin{proof}
	Let $ X_1,X_2...,X_n,... $ is a countable system of sets and each set $ X_m = \{x^1_m,...,x^n_m,...\} $ is itself countable. Since $ \forall m\in \Natural (card (X=\bigcup_{n \in \Natural}X_n) \geqslant X_m)$, $ X $ is an infinite set. The ordered pair $ (m,n) $ identifies the element $ x^n_m \in X_m $. We can construct a mapping, like $ f : \Natural \times \Natural \to \Natural \coloneqq (m,n) \to \frac{(m+n-2)(m+n-1)}{2}+m $, such that it is bijective. Thus $ X $ is countable. Then because $ card X \leqslant card \Natural $ and the fact that $ X $ is infinite, we conclude that $ card X = card \Natural $.
\end{proof}

If it is known that a set is either finite or countable, we say it is \textit{at most countable}($ card X \leqslant \Natural $).

\begin{Corollary}
	$ card \Zahlen = card \Natural $
\end{Corollary}

\begin{Corollary}
	$ card \Natural^2 = card \Natural $(The direct product of countable sets is countable).
\end{Corollary}

\begin{Corollary}
	$ card \Quoziente = card \Natural $, that is, the set of rational numbers is countable.
\end{Corollary}
\begin{proof}
	Let $ (m,n) $ denote a rational number $ \frac{m}{n} $. It is known that the pair $ (m,n) $ and $ (m^\prime, n^\prime) $ define the same number iff they are proportional. Thus $ \Quoziente $ is equipollent to some infinite subset of the set $ \Zahlen \times \Zahlen $. Since $ card \Zahlen^2 = card \Natural $, we can conclude that $ card \Quoziente = card \Natural $.
\end{proof}

\begin{Corollary}
	The set of algebraic numbers is countable.
\end{Corollary}
\begin{proof}
	It can be observed that $ card \Quoziente \times \Quoziente = card \Natural  $. By the principle of induction, $ \forall k \in \Natural (card \Quoziente^k = card \Natural) $. Let $ r \in \Quoziente^k $ be an ordered set $ (r_1,r_2,...,r_k) $ consists of $ k $ rational numbers. \newpara
	An algebraic equation of degree $ k $ with rational coefficient can be writtne in the reduced form $ x^k + r_1x^{k-1}+ \cdots + r_k = 0 $. Thus there are as many different algebraic equations of degree $ k $ as there are different ordered sets $ (r_1,...,r_k) $ of rational numbers, that is, a countable set. \newpara
	The algebraic equation with rational coefficients (of arbitrary degree) is the union of sets consisting of algebraic equation (of a fixed degree) which is countable, and this union is countable. Each such equation has only a finite number of roots. Hence the set of algebraic numbers is at most countable. But it is infinite, and therefore countable.
\end{proof}

\subsection{The Cardinality of the Continuum}
\begin{Definition}
	The set $ \Real $ of real numbers is also called the \textit{number continuum}(from Latin \textit{continuum}, meaning continuous, or solid), and its cardinality the \textit{cardinality of the continuum}.
\end{Definition}

\begin{Theorem}[Cantor]
	$ card \Natural < card \Real $
\end{Theorem}
\begin{proof}[Proof by Nested Interval Lemma]
	It is sufficient to show that even $ [0,1] $ in an uncountable set. \newpara
	Assume it is countable, that is, can be written as a sequence $ x_1,x_2,...,x_n,.... $. Take $ x_1 $ on $ I_0 = [0,1] $, and find $ I_1 $ such that $ x_1 \notin I_1 $. Then construct the nested interval $ I_n $ such that $ x_{n+1} \notin I_{n+1} $ and $ |I_n| > 0 $. It follows the nested interval lemma that there exist a point $ c \in [0,1]$ belonging to all $ I_n $. But by our construction, $ c \in \Real $ and $ c $ cannot be any point of the sequence $ x_1,x_2,...,x_n,.... $.
\end{proof}

\begin{proof}[Proof by Cantor's Diagonal Argument]
	Let's first consider an the set $ L $ and write out the infinite sequence of distinct binary numbers in it which has the form: 
	\begin{align}
		&s1 =	(0,	0,	0,	0,	0,	0,	0,	...) \\
		&s2 =	(1,	1,	1,	1,	1,	1,	1,	...) \\
		&s3 =	(0,	1,	0,	1,	0,	1,	0,	...)\\
		&s4 =	(1,	0,	1,	0,	1,	0,	1,	...)\\
		&s5 =	(1,	1,	0,	1,	0,	1,	1,	...)\\
		&s6 =	(0,	0,	1,	1,	0,	1,	1,	...)\\
		&s7 =	(1,	0,	0,	0,	1,	0,	0,	...)\\
		&... \\		
	\end{align}
	We then constrcut a number $ s $ such that its first digit is the complementary (swapping 0s for 1s and vice versa) of the first digit of $ s_1 $ and etc.
	\begin{align}
		&s1 =	(\mathbf{0},	0,	0,	0,	0,	0,	0,	...) \\
		&s2 =	(1,	\mathbf{1},	1,	1,	1,	1,	1,	...) \\
		&s3 =	(0,	1,	\textbf{0},	1,	0,	1,	0,	...)\\
		&s4 =	(1,	0,	1,	\textbf{0},	1,	0,	1,	...)\\
		&s5 =	(1,	1,	0,	1,	\textbf{0},	1,	1,	...)\\
		&s6 =	(0,	0,	1,	1,	0,	\textbf{1},	1,	...)\\
		&s7 =	(1,	0,	0,	0,	1,	0,	\textbf{0},	...)\\
		&... \\		
		&s = (\textbf{1},\textbf{0},\textbf{1},\textbf{1},\textbf{1},\textbf{0},\textbf{1},..)
	\end{align}
	By construction $ s $ differs from $ s_n $ at the $ n $th digit, so $ s $ is not in this sequence, and thus $ L $ is uncountable. \newpara
	We can now define a mapping $ f : L \to \Real $.$ f(s_n) = r_n\in \Real $ means that $ s_n $ and $ r_n $ have the same digit while $ r_n $ is under base 10 and $ s_n $ is under base 2. For $ s_n \neq s_m \Rightarrow (r_n=f(s_n)) \neq (r_m=f(s_m)) $, $ f $ is injective, and with the fact that all $ s_n $ corresponds to a $ r_n $ together give us $ cardf(L) = card L $. Since $ f(L) $ is a subset of $ \Real $, we can see that $ \Real $ is also uncountable.
\end{proof}
The cardinality of $ \Real $ is often denotes as $ \mathfrak{c} $.
\begin{Corollary}
	$ \Quoziente \neq \Real $, and so irrational numbers exist.
\end{Corollary}

\begin{Corollary}
	There exist transcendental numbers, since the set of algebraic numbers is countable.
\end{Corollary}

\subsection{Miscellaneous}
\begin{Statement}
	The cardinality of $ P(X) $, which is the power set of $ X $, satisfy that if $ card X = n $, $ card P(X) = 2^{n} $.
\end{Statement}
\begin{proof}
	We can use the principle of induction to complete the proof. If $ n= 1 $, $ X = \{x\} $, then $ P(X) =\{\varnothing, X\} $, then $ card P(X) = 2^{1} $. \newpara
	Now if $ n \in \Natural \Rightarrow card P(X) = 2^{n}  $, let $ X $ be a set that has $ x $ as one of its elements and has the cardinality of $ n+1 $. Therefore $ Y = X \setminus \{x\} $ has $ n $ elements. We can divide $ P(X) $ into two parts: the ones containing $ x $ and the ones don't. If $ x\in A \subset P(X) $, then $ A \setminus \{x\} \subset P(Y) $ and vice versa. Thus we can set up a bijection between $ P(Y) $ and the elements in $ P(X) $ that contains $ x $. Similarly, we can clearly see that a bijection between the subsets of $ P(X) $ that does not contains $ x $ and $ P(Y) $. Thus $ card P(X) = 2^n + 2^n = 2^{n+1} $, and we complete the proof.
\end{proof}







\chapter{Limits}
\section{The Limit of a Sequence}
\subsection{Definitions and Examples}
\begin{Definition}
	A number $ A \in \Real$  is called the \textit{limit of the numerical sequence} $ \{x_n\} $ if for every neighborhood $ V(A) $ of $ A $ there exists an index $ N $ (depending on $ V(A) $) such that all terms of the sequence having index larger than $ N $ belong to $ V(A) $. $ ((\lim\limits_{n \to \infty} x_n = A) \coloneqq \forall V(A) \exists N \in \Natural \forall n > N(x_n \in V(A)) ) $
	\newpara
	An equivalent way (or more common) way to say this is that a number $ A \in \Real $ is called the \textit{limit of the sequence} $ \{x_n\} $ if $ \forall \varepsilon >0$ there exists an index $ N $such that $ \forall n>N(|x_n-A|<\varepsilon )$. $ ((\lim\limits_{n \to \infty} x_n = A) \coloneqq \forall \varepsilon > 0 \exists N \in \Natural \forall n > N (|x_n-A|<\varepsilon) ) $
\end{Definition}

\begin{Definition}
	If $ \lim\limits_{n \to \infty} x_n = A $, we say that the sequence $ \{x_n\} $ \textit{converges} to $ A $ or \textit{tends} to $ A $ and write $ x_n \to A $ as $ n \to \infty $. Otherwise, it's called \textit{divergent}.
\end{Definition}

Examples: \\
$ \lim\limits_{n \to \infty} \frac{1}{n} = 0 $, since $ |\frac{1}{n}-0| = \frac{1}{n} < \varepsilon $ when $ n > N = [\frac{1}{\varepsilon}] $ (the integer part of $ \frac{1}{\varepsilon} $ ). \\
$ \lim\limits_{n \to \infty} \frac{n+1}{n} = 1 $, since $ |\frac{n+1}{n}-1| = \frac{1}{n} < \varepsilon $ if $ n > [\frac{1}{\varepsilon}] $. \\
$ \lim\limits_{n \to \infty} \frac{\sin n}{n} = 0$, since $|\frac{\sin n}{n}-0| \leqslant \frac{1}{n} < \varepsilon $ if $ n > [\frac{1}{\varepsilon}] $. \\
$ \lim\limits_{n \to \infty} \frac{1}{q^n} = 0$ if $ |q|>1 $.
\begin{proof}
	As shown in the proof in 2.2.4(Miscellaneous), for every $ \varepsilon >0 $ there exists $ N \in \Natural $ such that $ \frac{1}{|q|^N}<\varepsilon $. Since $ |q|>1 $, we have $ |\frac{1}{q^n}-0| \leqslant \frac{1}{|q|^n} <\frac{1}{|q|^N}<\varepsilon$ for $ n>N $. 
\end{proof}
The sequence $ 1,2,\frac{1}{3},4,\frac{1}{5},6,\frac{1}{7}... $ whose $ n $th term is $ x_n = n^{(-1)^n} $, $ n \in \Natural $, is divergent.
\begin{proof}
	If $ A $ is the limit of this sequence, then any neighborhood of $ A $ would contain all but a finite number of terms of the sequence. \\
	A number $ A \neq 0 $ cannot be the limit, since when $ \varepsilon = \frac{|A|}{2} > 0 $, any point of the form $ \frac{1}{2k+1} $ for which $ \frac{1}{2k+1} < \frac{|A|}{2}$ lie outside the $ \varepsilon $-neighborhood of $ A $. At the same time $ A=0 $ cannot be the limit of this sequence because there are infinitely many terms lie outside of even $ 1 $-neighborhood of $ 0 $.
\end{proof}







\subsection{Properties of the Limit of a Sequence}
\begin{Definition}
	If there exists a number $ A $ and an index $ N $ such that $ x_n = A $ for all $ n>N $, the sequence $ \{x_n\} $ will be called \textit{ultimately constant}.
\end{Definition}
\begin{Definition}
	A sequence $ \{x_n\} $ is \textit{bounded} if there exists $ M $ such that $ |x_n|<M $ for all $ n \in \Natural $.
\end{Definition}

\begin{Theorem}
	An ultimately constant sequence converges.
\end{Theorem}
\begin{proof}
	if $ x_N = A $, then $ \forall n > N (x_n \in V(A)) $.
\end{proof}

\begin{Theorem}
	Any neighborhood of the limit of a sequence contains all but a finite number of terms of the sequence.
\end{Theorem}

\begin{Theorem}
	A convergent sequence cannot have two different limits.
\end{Theorem}
\begin{proof}
	Suppose the contrary, that is, $ A_1 $ and $ A_2 $ are both the limit of the sequence $ {x_n} $. Then we find two nonintersecting neighborhoods $ V(A_1) $ and $ V(A_2) $ of $ A_1 $ and $ A_2 $. By the definition of limits we find two indices $ N_1 $ and $ N_2 $ such that $ \forall n > N_1 (x_n \in V(A_1)) $ and $ \forall n > N_2 (x_n \in V(A_2)) $. But then for $ N = max\{N_1,N_2\} $, we'll have $ \forall n > N (x_n \in V(A_1) \cap V(A_2)) $, and this is impossible since $ V(A_1) \cap V(A_2) = \varnothing $.
\end{proof}

\begin{Theorem}
	A convergent sequence is bounded.
\end{Theorem}
\begin{proof}
	Let $ \lim\limits_{n \to \infty}x_n = A $.Set $ \varepsilon = 1 $ in the common definition of limit, we find $ N $ such that $ |x_n-A| <1 $ for all $ n > N $. Then by the triangle inequality we have $ |x_n| < |A| +1 $. Considering  $ n<N $, we take $ M > max \{|x_1|,|x_2|,...,|x_n|,|A|+1\} $, and for all $ n \in \Natural $ we have $ M > |x_n| $.
\end{proof}

\subsubsection{Passage to the Limit and the Arithmetic Operations}
\begin{Definition}
	If $ \{x_n\} $ and $ \{y_n\} $ are two numerical sequences, their \textit{sum, product,} and \textit{quotient} are the sequences
	\begin{equation}
		\{(x_n+y_n)\},\quad \{(x_n \cdot y_n)\},   \quad  \{(\frac{x_n}{y_n})\}  \nonumber
	\end{equation}
	while the quotient is defined when $ \forall n \in \Natural (y_n \neq 0) $.
\end{Definition}

	Let $ \{x_n\} $ and $ \{y_n\} $ be two numerical sequences, if $ \lim\limits_{n \to \infty}x_n = A $ and $ \lim\limits_{n \to \infty}y_n = B $, then
\begin{Theorem}
	$ \lim\limits_{n \to \infty}\{x_n +y_n\}= A + B $. 
\end{Theorem}
\begin{proof}
	Set $ |A - x_n|  = \Delta(x_n)$, $ |B-y_n|=\Delta(y_n) $. Now we have
	\begin{equation}
		|(A+B)-(x_n+y_n)| \leqslant \Delta(x_n) + \Delta(y_n) \nonumber
	\end{equation}
	Suppose $ \varepsilon > 0 $ is given. Since $ \lim\limits_{n \to \infty}x_n = A $ , there exists $ N^\prime $ such that $ \Delta(x_n)<\varepsilon/2 $ for all $ n > N^\prime $. Similarly, since $ \lim\limits_{n \to \infty}y_n = B $, there exists $ N^{\prime\prime} $ such that $ \Delta(y_n)<\varepsilon/2 $ for all $ n > N^{\prime}\prime $. Then for $ n>max\{N^\prime, N^{\prime\prime} $ we have
	\begin{equation}
			|(A+B)-(x_n+y_n)| \leqslant \varepsilon \nonumber
	\end{equation}
	and our proof completes.
\end{proof}

\begin{Theorem}
	$ \lim\limits_{n \to \infty}\{x_n \cdot y_n\}= A \cdot B $.
\end{Theorem}
\begin{proof}
	Similar to our first proof, we show that $ |x_n|\Delta(y_n) $, $ |y_n|\Delta(x_n) $ and $ \Delta(x_n)\cdot\Delta(y_n)$ are less than $ \frac{\varepsilon}{3} $, since their sum is greater or equal to $ |(A\cdot B)-(x_n\cdot y_n)| $.
\end{proof}

\begin{Theorem}
	$ \lim\limits_{n \to \infty}\{\frac{x_n }{y_n}\}= \frac{A}{B} $ when $ \forall n \in \Natural (y_n \neq 0 )\land (B \neq 0) $.
\end{Theorem}
\begin{proof}
	If we prove that $ |x_n|\cdot \frac{1}{y_n^2}\Delta(y_n) = \frac{\varepsilon}{4} $, $ |\frac{1}{y_n}| \Delta(x_n) = \frac{\varepsilon}{4} $, and $ 0<\frac{1}{1-\delta(y_n)}<2 $, we'll have $ |\frac{A}{B}-\frac{x_n}{y_n}|<\varepsilon $
\end{proof}

\subsubsection{Passage to the Limit and Inequalities}
\begin{Theorem}
	Let $ \{x_n\} $ and $ \{y_n\} $ be two convergent sequences with $ \lim\limits_{n \to \infty}x_n =A$ and $ \lim\limits_{n \to \infty}y_n=B $. If $ A<B $, then there exists an index $ N \in \Natural $ such that $ x_n<y_n $ for all $ n>N $.
\end{Theorem}
\begin{proof}
	Choose a number $ C $ such that $ A<C<B$. By definition of limit, we can find numbers $ N^\prime $  and $ N^{\prime\prime} $ such that $ |x_n-A|<C-A $ for all $ n > N^\prime $ and $ |y_n-B| < B-C $ for all $ n > N^{\prime\prime} $. Then for $ n > N = max\{N^\prime,N^{\prime\prime}\} $ we shall have $ x_n < (A+C-A )= C = (B-(B-C))<y_n $.
\end{proof}

\begin{Theorem}
	Suppose the sequences $ \{x_n\} $, $ \{y_n\} $, and $ \{z_n\} $ are such that $ x_n \leqslant y_n \leqslant z_n $ for all $ n >N\in \Natural $. If the sequences $ \{x_n\} $ and $ \{z_n\} $ both converge to the same limit, then the sequence $ \{y_n\} $ also converges to that limit.
\end{Theorem}
\begin{proof}
Suppose $ \lim\limits_{n \to \infty}x_n = \lim\limits_{n \to \infty}z_n=A $. Given $ \varepsilon>0 $ choose $ N^\prime $ and $ N^{\prime\prime} $ such that $ \forall n>N^\prime (A-\varepsilon<x_n) $ and $ \forall n> N^{\prime\prime}(z_n < A+ \varepsilon) $. Then for $ n > N = max\{N^\prime,N^{\prime\prime}\} $ we shall have $ A-\varepsilon < x_n \leqslant y_n \leqslant z_n < A+ \varepsilon $, which says $ |y_n-A| < \varepsilon$, that is $ A = \lim\limits_{n \to \infty}y_n $.
\end{proof}

\begin{Corollary}
	Suppose $ \lim\limits_{n \to \infty}x_n = A $ and $ \lim\limits_{n \to \infty}y_n=B $. If there exists $ N $ such that for all $ n > N $ we have \\
	a) $ x_n>y_n $, then $ A \geqslant B $; \\
	b) $ x_n\geqslant y_n $, then $ A \geqslant B $; \\
	c) $ x_n>B $, then $ A \geqslant B $; \\
	d) $ x_n\geqslant B $, then $ A \geqslant B $. 
\end{Corollary}
\begin{proof}
	The first two statement can be proved by contradiction using the theorem we mentioned above, and the last two statement are the special cases when $ y_n \equiv B $.
\end{proof}





\subsection{Questions Involving the Existence of the Limit of a Sequence}
\subsubsection{The Cauchy Criterion}
\begin{Definition}
	A sequence $ \{x_n\} $ is called a \textit{fundamental} or \textit{Cauchy} sequence if for any $ \varepsilon>0 $ there exists an index $ N\in \Natural $ such that $ |x_m-x_n| <\varepsilon$ whenever $ n > N $ and $ m > N $.
\end{Definition}
\begin{Theorem}[Cauchy's convergence criterion]
	A numerical sequence converges if and only if it is a Cauchy sequence.
\end{Theorem}
\begin{proof}
	First, we prove that a convergent sequence is a Cauchy sequence. If $ \lim\limits_{n \to \infty}x_n = A $, we set $ |x_n-A| < \frac{\varepsilon}{2} $, and we obtain that $ |x_m-x_n| \leqslant |x_m -A|+ |x_n - A| < \frac{\varepsilon}{2}+\frac{\varepsilon}{2}=\varepsilon $ for $ m,n>N $ (imagine two points on a line, their distance to a fixed point is greater or equal to the distance between these two points ). \newpara
	Next, let $ \{x_k\} $ be a Cauchy sequence. We find $ |x_m-x_k|<\frac{\varepsilon}{3} $ for $ m,k\geqslant N $. Fix $ m = N $, we can see that $ x_N - \frac{\varepsilon}{3}<x_k<x_N+\frac{\varepsilon}{3} $, and thus this sequence is bounded. Then we set $ a_n\coloneqq inf_{k\geqslant n}x_k $ and $ b_n\coloneqq sup_{k\geqslant n}x_k $. Apply the \textbf{Nested Interval Lemma} to the close intervals $ [a_n,b_n] $ and denote that point $ A $. The inequality can be derived from the following inequalities:
	\begin{align}
		a_n\leqslant x_k &\leqslant b_k \nonumber \\
		|A-x_k|&\leqslant b_n - a_n \nonumber \\
		x_N - \frac{\varepsilon}{3} \leqslant a_n &\leqslant b_n \leqslant x_N+\frac{\varepsilon}{3} \nonumber
	\end{align}
\end{proof}

\subsubsection{A Criterion for the Existence of the Limit of a Monotonic Sequence}
\begin{Definition}
	A sequence $ \{x_n\} $ is \textit{increasing} if $ \forall n \in \Natural (x_n < x_{n+1}) $, \textit{nondecreasing} if $ \forall n \in \Natural (x_n \leqslant x_{n+1}) $, \textit{nonincreasing} if $ \forall n \in \Natural (x_n \geqslant x_{n+1}) $, and \textit{decreasing} if $ \forall n \in \Natural (x_n > x_{n+1}) $. Sequence of these four types are called \textit{monotonic} sequences.
\end{Definition}

\begin{Definition}
	A sequence $ \{x_n\} $ is \textit{bounded above} if there exists a number $ M $ such that $ \forall n \in \Natural (x_n <M)  $ .
\end{Definition}

\begin{Theorem}[Weierstrass]
	In order for a nondecreasing sequence to have a limit, it is necessary and sufficient that it be bounded above.
\end{Theorem}
\begin{proof}
	We already know that a convergent sequence is bounded, hence we just need to prove its sufficiency. \newpara
	Let $ s $ be the least upper bound of this sequence. Since it's a nondecreasing sequence, we have $ s-\varepsilon < x_N \leqslant x_n \leqslant s $ for all $ n>N $, and thus $ s-x_n<\varepsilon $.
\end{proof}
Analogouly one can prove that in order for a nonincreasing sequence to have a limit, it has to be bounded below.


\begin{Corollary}
	$ \lim\limits_{n \to \infty} \frac{n}{q^n} =0$  if $ q>1 $.
\end{Corollary}
\begin{proof}
	We prove that this sequence is bounded below(all terms are positive) . Let $ x = \lim\limits_{n \to \infty}x_n $ for $ n > N $, where $ N $ is the index that satisfies $ x_{n+1}<x_n $(from $ N $ the sequence is monotonically decreasing). From the relation $ x_{n+1} = \frac{n+1}{nq}x_n $, one can have:
	\begin{equation}
		x = \frac{1}{q}x \nonumber
	\end{equation}
	and hence $ (1-\frac{1}{q})x = 0 \Rightarrow (x=0)$.
\end{proof}

\begin{Corollary}
	$ \lim\limits_{n \to \infty} \sqrt[n]{n} = 1 $.
\end{Corollary}
\begin{proof}
	There exists $ N \in \Natural $ such that $ 1 \leqslant n < (1+\varepsilon)^n $ for all $ n > N $. Thus $ 1 \leqslant \sqrt[n]{n} < 1+ \varepsilon $, which implies $ \lim\limits_{n \to \infty}\sqrt[n]{n} = 1 $.
\end{proof}

\begin{Corollary}
	$ \lim\limits_{n \to \infty} \sqrt[n]{a} =1$ for any $ a >0 $.
\end{Corollary}
\begin{proof}
	Assume $ a \geqslant 1 $ first, use the same technique from above, and then prove $ \lim\limits_{n \to \infty} \sqrt[n]{a} =1$ for $ 0<a<1 $.
\end{proof}



\begin{Statement}
	$ \forall q \in \Real \forall n\in  \Natural (\lim\limits_{n \to \infty} \frac{q^n}{n!} = 0) $.
\end{Statement}

\begin{proof}
	If $ q = 0 $, the assertion is obvious. Since $ |\frac{q^n}{n!}| =\frac{|q|^n}{n!}  $, let's assume $ q >0 $. Use the same technique as we used in proving $  \lim\limits_{n \to \infty} \frac{n}{q^n} =0 $. First prove that $ 0<\frac{1}{n+1} <1 $ and $ x_{n+1}<x_n $ for all $ n >N \in \Natural $, thus this sequence is monotonically decreasing from $ N $. Let the limit be $ x $, one will have:
	\begin{equation}
		x=\lim\limits_{n \to \infty}\frac{q}{n+1} \cdot \lim\limits_{n \to \infty}x_n = 0\cdot x = 0 \nonumber
	\end{equation}
\end{proof}

\subsubsection{The Number $e $}
\begin{Theorem}[Jacob Bernoulli's inequality]
	\begin{equation}
		(1+\alpha)^{n} \geqslant 1+ n\alpha
	\end{equation}
	holds for $ n \in \Natural $ and $ \alpha > -1 $.
\end{Theorem}

\begin{proof}
	Prove by the principle of induction.
\end{proof}
Incidentaly, strict inequality holds if $ \alpha \neq 0 $ and $ n >1 $.





\begin{Statement}
	The limit $ \lim\limits_{n \to \infty}(1+\frac{1}{n})^n $ exists.
\end{Statement}
\begin{proof}
	Let $ y_n = (1+\frac{1}{n})^{n+1} $ and $ n \geqslant 2 $. Use Bernoulli's inequality, we find that $ \frac{y_{n-1}}{y_n}>1 $, and since all terms are positive, this sequence is bounded, monotonically decreasing and hence, has a limit. Then 
	\begin{equation}
		\lim\limits_{n \to \infty}(1+\frac{1}{n})^n = \lim\limits_{n \to \infty}(1+\frac{1}{n})^{n+1} \cdot \lim\limits_{n \to \infty}(1+\frac{1}{n})^{-1} =  \lim\limits_{n \to \infty}(1+\frac{1}{n})^{n+1} \nonumber
	\end{equation}
\end{proof}

\begin{Definition}
	$ e \coloneqq \lim\limits_{n \to \infty} (1+\frac{1}{n})^n $.
\end{Definition}

\subsubsection{Subsequences and Partial Limits of a Sequence}
\begin{Definition}
	If $ x_1,x_2,...,x_n,... $ is a sequence and $ n_1 < n_2<n_3<...<n_k<... $ an increasing sequence of natural numbers, then the sequence $ x_{n_1},x_{n_2},...,x_{n_k},... $ is called a \textit{subsequence} of the sequence $ \{x_n\} $.
\end{Definition}

\begin{Lemma}[Bolzano-Weierstrass]
	Every bounded sequence of real numbers contains a convergent subsequence.
\end{Lemma}
\begin{proof}
	Let $ E $ be the set of values of the bounded sequence $ \{x_n\} $. If $ E $ is finite, there exists a point $ x \in E $ and a sequence $ n_1<n_2<... $ of indices such that $ x_{n_1} = x_{n_2}=...=x $. The subsequence is constant and hence converges. \newpara
	If $ E $ is infinite, then by the \textbf{Bolzano-Weierstrass principle} from section 2.3.3 it has a limit point $ x $. Use the property of limit points, one can see that $ |x_{n_k}-x| < \frac{1}{k} $ and $ |x_{n_{k+1}}-x|<\frac{1}{k+1} $. Because $ \lim\limits_{k \to \infty} \frac{1}{k} = 0$, the sequence $ x_{n_1},x_{n_2},...,x_{n_k},... $ converges to $ x $.
\end{proof}

\begin{Definition}
	We shall write $ x_n \to+\infty $ and say that the sequence $ \{x_n\} $ \textit{tends to positive infinity} if for each number $ c $ there exists $ N \in \Natural $ such that $ \forall n > N (x_n >c )$. We can gneralize this two both positive and negative infinity:
	\begin{equation}
		(x_n\to \infty) \coloneqq \forall c \in \Real \exists N \in \Natural \forall n> N (c<|x_n|). \nonumber
	\end{equation}
\end{Definition}

\begin{Lemma}
	From each sequence of real numbers one can extract either a convergent subsequence of a subsequence that tends to infinity.
\end{Lemma}
\begin{proof}
	When the sequence $ \{x_n\} $ is not bounded, the new case occurs. Then for each $ k \in \Natural $ we can choose $ n_k \in \Natural $ such that $ |x_{n_k}>k| $ and $ n_k < n_{k+1} $. This sequence is monotonically increasing and not bounded above, hence it tends to infinity.
\end{proof}

Let $ \{x_k \} $ be an arbitrary sequence of real numbers that is bounded below. We can consider the sequence $ i_n = inf_{k \geqslant n} x_k $. The sequence $ \{i_n \} $ has a finite limit $ \lim\limits_{n \to \infty} i_n = l $, or $ i_n \to +\infty $, since $ \forall n \in \Natural (i_n \leqslant i_{n+1} )$.

\begin{Definition}
	The number $ l = \lim\limits_{n \to \infty}inf_{k \geqslant n}x_k $ is called the \textit{inferior} limit of the sequence $ \{x_k\} $ and denoted $  \varliminf_{k \to \infty}x_k $. If $ i_n\to + \infty $, it is said that the inferior limit of the sequence equals positive infinity, and we write $ \varliminf_{k \to \infty}x_k = + \infty $. If the original sequence $ \{x_k\} $ is not bounded below, then we shall have $ i_n = inf_{k \geqslant n}x_k = - \infty $ and write $ \varliminf_{k \to \infty} x_k = -\infty $.
	\begin{equation}
		\varliminf_{k \to \infty}x_k \coloneqq \lim\limits_{n \to \infty} inf_{k \geqslant n}x_k \nonumber
	\end{equation}
	Similarly, the superior limit of the sequence $ \{x_k\} $ can be defined as 
	\begin{equation}
			\varlimsup_{k \to \infty}x_k \coloneqq \lim\limits_{n \to \infty} sup_{k \geqslant n}x_k \nonumber
	\end{equation}
\end{Definition}
\includegraphics[width=\textwidth]{Lim_sup_example_5}
\begin{Definition}
	A number (or the symbol $ - \infty $ or $ +\infty $) is called a \textit{partial limit} of a sequence, if the sequence contains a subsequence converging to that number.
\end{Definition}

\begin{Proposition}
	The inferior and superior limits of any sequence are respectively the smallest and largest partial limits of the sequence.
\end{Proposition}
\begin{proof}
	Let's assume that this sequence is bounded. First consider the inferior limit $ i = \varliminf_{k \to \infty}x_k $. The sequence $ i_n = inf_{k\geqslant n}x_k $ is nondecreasing. Using the definition of the greatest lower bound, we choose by induction numbers $ k_n \in \Natural $ such that $ k_n < k_{n+1} $ and $ i_{k_n} \leqslant x_{k_n} < i_{k_n}+\frac{1}{n} $ (Taking $ i_1 $ we find $ k_1 $; taking $ i_{k_1+1} $ we find $ k_2 $, etc.). Since $ \lim\limits_{n \to \infty}i_n = \lim\limits_{n \to \infty}(i_n+\frac{1}{n}) = i $, we have $ \lim\limits_{n \to \infty}x_{k_n}=i $. It is the smallest partial limit since for every $ \varepsilon>0 $ there exists $ n \in \Natural $ such that $ i-\varepsilon<i_n $, that is $ i-\varepsilon <i_n = inf_{k\geqslant n}x_k \leqslant x_k $ for any $ k \geqslant n $. Now we have $ i-\varepsilon < x_k $ for $ k >n $ means that no partia limit of the sequence can be less than $ i - \varepsilon $. But $ \varepsilon >0 $ is arbitrary, and hence no partial limit can be less than $ i $. The proof for the superior limit is of course analogous. \newpara
	Now if the sequence is not bounded below(resp. above), one can select a subsequence of it tending to $ -\infty $ (resp. $ +\infty $). But then we also have $ \varliminf_{k \to \infty}x_k = -\infty $ (resp. $ \varlimsup_{k \to \infty}x_k = +\infty $). Finally, if $ \varlimsup_{k \to \infty}x_k = -\infty $ (resp. $ \varliminf_{k \to \infty}x_k = +\infty $), the sequence itself tends to $ -\infty $ (resp. $ +\infty $).
\end{proof}

\begin{Corollary}
	A sequence has a limit or tends to $ \pm \infty $ iff its inferior and superior limits are the same.
\end{Corollary}
\begin{proof}
	The cases when $ \varliminf_{k \to \infty}x_k = \varlimsup_{k \to \infty}x_k = \pm \infty $ have benn investigated above, and so we may assume that $ \varliminf_{k \to \infty}x_k = \varlimsup_{k \to \infty}x_k = A \in \Real $. Since $ (i_n = inf_{k \geqslant n}x_k) \leqslant x_k \leqslant (sup_{k \geqslant n}x_k = s_n) $, we have $ \lim\limits_{n \to \infty}x_n = A $.
\end{proof}

\begin{Corollary}
	A sequence converges iff every subsequence of it converges.
\end{Corollary}
\begin{proof}
	The inferior and superior limits of a subsequence lie between those of the sequence itself. If the sequence converges, then its subsequences must converge, and their limits are the same. The converse assertion is obvious, since the subsequence can be chosen as the sequence itself.
\end{proof}

\begin{Corollary}
	The Bolzano-Weierstrass Lemma in its restricted and wider formulations(corresponding to Lemmas at page 32 and page 33) follows from the Proposition we just proved.
\end{Corollary}
\begin{proof}
	If the sequence $ \{x_k\} $ is bounded, then $ i = \varliminf_{k \to \infty}x_k $ and $ s = \varlimsup_{k \to \infty}x_k $ are finite and partial limits of the sequence. When $ i=s $ some subsequences have a unique limit, and at least two when $ i<s $. If the sequence is unbounded on one side or the other, there exists a subsequence tending to the corresponding infinity.
\end{proof}



\subsection{Elementary Facts about Series}

\subsubsection{The Sum of a Series and the Cauchy Criterion for Convergence of Series}

\begin{Definition}
	The expression $ a_1 + a_2 + \cdots +a_n + \cdots$ is denoted by the symbol $ \sum_{n=1}^{\infty}a_n $ and usually called a \textit{series} or an \textit{infinite series}.
\end{Definition}

\begin{Definition}
	The elements of the sequence $ \{a_n\} $, when regarded as elements of the series, are called the \textit{terms} of the series. The element $ a_n $ is called the \textit{$ n $th term}.
\end{Definition}

\begin{Definition}
	The sum $ s_n = \sum_{k=1}^{n}a_k $ is called the \textit{partial sum of the series} or the \textit{$ n $th partial sum of the series}.
\end{Definition}

\begin{Definition}
	If the sequence $ \{s_n\} $ of partial sums of a series converges(resp. divergence), we say the series is \textit{convergent} (resp. \textit{divergent}).
\end{Definition}

\begin{Definition}
	The limit $ \lim\limits_{n \to \infty}s_n = s $ of the sequence of partial sums of the series, if it exists, is called the \textit{sum of the series}.
\end{Definition}
One can see that $ \sum_{n=1}^{\infty}a_n=s $.

\begin{Theorem}[The Cauchy convergence criterion for a series]
	The series $ a_1 + \cdots + a_n + \cdots $ converges iff for every $ \varepsilon >0 $ there exists $ N \in \Natural $ such that the inequalities $ m \geqslant n > N $ imply $ |a_n + \cdots +a_m|<\varepsilon $.
\end{Theorem}

\begin{Corollary}
	If only a finite number of terms of a series are changed, the resulting new series will converge if the original series did and diverge if it diverged.
\end{Corollary}

\begin{Corollary}
	A necessary condition for convergence of the series $ a_1 + \cdots + a_n + \cdots $ is that the terms tend to zero as $ n \to \infty $, that is, it is necessary that $ \lim\limits_{n \to \infty}a_n=0 $.
\end{Corollary}
\begin{proof}
	Set $ m=n $ in the Cauchy convergence criterion and use the definition of the limit of a sequence. \newpara
	Alternatively, $ a_n = s_n - s_{n-1} $, and, given that $ \lim\limits_{n \to \infty}s_n=s $, we have $ \lim\limits_{n \to \infty} a_n =  \lim\limits_{n \to \infty} (s_n - s_{n-1}) = s -s = 0$ .
\end{proof}

\begin{Statement}
	The series $ 1+q+q^2+ \cdots + q^n + \cdots  $ is often called the \textit{geometric series}. It converges iff $ |q|<1 $.
\end{Statement}
\begin{proof}
	Suppose $ |q|\geqslant 1 $, then we have $ |q^n| \geqslant 1 $, and in this case this series does not converges.\newpara
	Now let $ |q|<1 $, and we'll have $ s_n = 1+ q + \cdots + q^{n-1} = \frac{1-q^n}{1-q} $ and $ \lim\limits_{n \to \infty}s_n = \frac{1}{1-q} $, since $ \lim\limits_{n \to \infty}q^n = 0 $ if $ |q|<1 $.
\end{proof}

\begin{Statement}
	The series $ 1+\frac{1}{2}+ \cdots +\frac{1}{n}+ \cdots$ is called the \textit{harmonic series} diverges because its partial sums $ s_n = 1+ \frac{1}{2}+\cdots \frac{1}{n} $ diverges.
\end{Statement}
\begin{proof}
	It's sufficient to prove that its partial sum $ s_n = 1+ \frac{1}{2}+\cdots \frac{1}{n} $ diverges. For all $ n \in \Natural $ we have 
	\begin{equation}
		|x_{2n}-x_n| = \frac{1}{n+1}+ \cdots + \frac{1}{n+n} > n \cdot \frac{1}{2n} = \frac{1}{2}\nonumber
	\end{equation}
	and our proof completes.
\end{proof}

\textbf{\textit{Remark}} Usual laws for dealing with finite sums does not apply to series in general (e.g. insert parentheses to a divergent series). 

\subsubsection{Absolute Convergence. The Comparison Theorem and Its Consequences}

\begin{Definition}
	The series $ \sum_{n=1}^{\infty} $ is \textit{absolutely} convergent if the series $ \sum_{n=1}^{\infty}|a_n| $ converges.
\end{Definition}

Since $ |a_n + \cdots + a_m| \leqslant |a_n|+\cdots |a_m|$, the Cauchy convergence criterion implies that an absolutely convergent series converges, but the converse is generally not ture.

\begin{Statement}
	The series $ 1-1+\frac{1}{2}-\frac{1}{2}+ \frac{1}{3}-\frac{1}{3}+ \cdots$, whose partial sums are either $ \frac{1}{n} $ or $ 0 $, converges to $ 0 $. However, this sequence does not absolutely converges, and its proof is similar to the proof for the divergence of harmonic series.
\end{Statement}

\begin{Theorem}[Criterion for convergence of series of nonnegative terms]
	A series whose terms are nonnegative converges iff the sequence of partial sums is bounded above.
\end{Theorem}

\begin{Theorem}[Comparison Theorem]
	Let $ \sum_{n=1}^{\infty}a_n $ and $ \sum_{n=1}^{\infty}b_n $ be two series with non-negative terms. If there exists an index $ N\in \Natural $ such that $ a_n \leqslant b_n $ for all $ n >N $, then the convergence of the series $  \sum_{n=1}^{\infty}b_n $ implies the convergence of $ \sum_{n=1}^{\infty}a_n $, and the divergence of $ \sum_{n=1}^{\infty}a_n $ implies the divergence of $\sum_{n=1}^{\infty}b_n $.
\end{Theorem}
\begin{proof}
	content...
\end{proof}

\begin{Corollary}[The Weierstrass M-test for absolute convergence]
		Let $ \sum_{n=1}^{\infty}a_n $ and $ \sum_{n=1}^{\infty}b_n $ be series. Suppose there exists an index $ X $ such that $ |a_n| \leqslant b_n $ for all $ n > N $. Then a sufficient condition for absolute convergence of the series $ \sum_{n=1}^{\infty}a_n $ is that the series $ \sum_{n=1}^{\infty}b_n $ converge. \newpara
		It's often summarized as following: I\textbf{f the terms of a series are majorized (in absolute value) by the terms of a convergent numerical series, then the original series converges absolutely}.
\end{Corollary}
\begin{proof}
	By the comparison theorem the series $ \sum_{n=1}^{\infty}|a_n| $ will then converge, and that is what is meant by the absolute convergence of $ \sum_{n=1}^{\infty}a_n $.
\end{proof}

\begin{Corollary}[Cauchy's Test]
	Let $ \sum_{n=1}^{\infty}a_n $ be a given series and $ \alpha = \varlimsup_{n \to \infty} \sqrt[n]{|a_n|}$. Then the following are true:\\
	a) if $ \alpha <1 $, the series converges absolutely;\\
	b) if $ \alpha >1 $, the series diverges; \\
	c) there exist both absolutely convergent and divergent series for which $ \alpha = 1 $.
\end{Corollary}
\begin{proof}
	content...
\end{proof}

\begin{Corollary}[d'Alembert's Test]
	Suppose the limit $ \lim\limits_{n \to \infty} |\frac{a_{n+1}}{a_n}|= \alpha $ exists for the series $ \sum_{n=1}^{\infty}a_n $. Then, \\
	a) if $ \alpha <1 $, the series converges absolutely;\\
	b) if $ \alpha >1 $, the series diverges; \\
	c) there exist both absolutely convergent and divergent series for which $ \alpha = 1 $.
\end{Corollary}

\begin{proof}
	content...
\end{proof}

\begin{Proposition}[Cauchy]
	If $ a_1 \geqslant a_2 \geqslant \cdots \geqslant 0 $, the series $ \sum_{n=1}^{\infty}a_n $ converges iff the series $ \sum_{k=0}^{\infty}2^ka_{2^k} = a_1 + 2a_2 + 4a_4 + 8a_8 \cdots$ converges.
\end{Proposition}
\begin{proof}
	content...
\end{proof}

\begin{Corollary}
	The series $ \sum_{n=1}^{\infty}\frac{1}{n^p} $ converges for $ p>1 $ and diverges for $ p \leqslant1 $.
\end{Corollary}
\begin{proof}
	content...
\end{proof}


\subsubsection{The Number $ e $ as the Sum of a Series}
We know that $ e=\lim\limits_{n \to \infty}(1+\frac{1}{n})^n $. By Newton's binomial formula:
\begin{align}
	(1+\frac{1}{n})^n &= 1 + \frac{n}{1!}\frac{1}{n} + \frac{n(n-1)}{2!}\frac{1}{n^2} + \cdots + \frac{(n(n-1)\cdots(n-k+1))}{k!}\frac{1}{n^k}+ \cdots +\frac{1}{n^n} \\
	&= 1+ 1+ \frac{1}{2!}(1-\frac{1}{n}) + \cdots + \frac{1}{k!}(1-\frac{1}{n})(1-\frac{2}{n}) \times \cdots \\
	&\times (1-\frac{k-1}{n}) + \cdots + \frac{1}{n!}(1-\frac{1}{n})\cdots(1-\frac{n-1}{n})
\end{align}
Setting $ (1+\frac{1}{n})=e_n $ and $ 1+1+\cdots \frac{1}{2!} + \cdots + \frac{1}{n!}=s_n$, we thus have $\forall n \in \Natural(e_n < s_n)  $. \newpara
On the other hand, for any fixed $ k $ and $ n \geqslant k $, as can be seen from the same expansion, we have
\begin{equation}
	1+1+1+ 1+ \frac{1}{2!}(1-\frac{1}{n}) + \cdots + \frac{1}{k!}(1-\frac{1}{n})(1-\frac{2}{n})\cdots (1-\frac{k-1}{n})<e_n \nonumber
\end{equation}
As $ n \to \infty $ the left-hand side of the inequality tends to $ s_k $ and the right-hand side to $ e $. We can now conclude that $ s_k \leqslant e $ for all $ k \in \Natural $. Then from the relation $ e_n < s_n \leqslant e $ we find that $ \lim\limits_{n \to \infty}s_n=e $.
\begin{Definition}
	$ e=1+\frac{1}{1!}+ \frac{1}{2!}+\cdots+\frac{1}{n!}+\cdots $
\end{Definition}
The difference between $ e $ and its estimation $ s_n $ can be expressed as following
\begin{align}
	0<e-s_n &= \frac{1}{(n+1)!}+ \frac{1}{(n+2)!}+ \cdots =\\
	&= \frac{1}{(n+1)!}[1+\frac{1}{n+2}+\frac{1}{(n+2)(n+3)}+\cdots] < \\
	&< \frac{1}{(n+1)!}[1+\frac{1}{n+2}+\frac{1}{(n+2)^2}+\cdots] \\
	&=\frac{1}{(n+1)!}\frac{1}{1-\frac{1}{n+2}} = \frac{n+2}{n!(n+1)^2} < \frac{1}{n!n}
\end{align}
This estimate of the difference $ e-s_n $ can be written as the equality 
\begin{equation}
	e=s_n+\frac{\theta_n}{n!n} \nonumber
\end{equation}

where $ 0<\theta_n<1 $. \\

Hence $ e $ is irrational.
\begin{proof}[$ e $'s irrationality]
	Suppose the contrary, that $ e = \frac{p}{q} $, where $ p,q \in \Natural $. Then the number $ q!e $ must be an integer, while
	\begin{equation}
		q!e = q!(s_q+\frac{\theta_n}{q!q}) = q!+\frac{q!}{1!}+ \frac{q!}{2!}+\cdots+\frac{q!}{n!}+\frac{\theta_q}{q} \nonumber
	\end{equation}
	and therefore $ \frac{\theta_q}{q} $ would have to be an integer, which is impossible.
\end{proof}

Moreover, $ e $ is transcendental.





\section{The Limit of a Function}

\subsection{Definitions and Examples}
\subsection{Properties of the Limit of a Function}
\subsection{The General Definition of the Limit of a Function(Limit over a Base)}
\subsection{Existence of the Limit of a Function}




























\chapter{Continuous Functions}
\section{Basic Definitions and Examples}
\section{Properties of Continuous Functions}

\chapter{Differential Calculus}
\section{Differentiable Functions}
\section{The Basic Rules of Differentiation}
\section{The Basic Theorems of Differential Calculus}
\section{Differential Calculus Used to Study Functions}
\section{Complex Numbers and Elementary Functions}
\section{Examples of Differential Calculus in Natural Science}
\section{Primitives}


\chapter{Integration}
\section{Definition of the Integral and Description of the Set of Integrable Functions}
\section{Linearity, Additivity and Monotonicity on the Space $ \mathcal{R} $[a,b]}
\section{The Integral and the Derivative}
\section{Some Applications of Integration}
\section{Improper Integrals}


\chapter{Functions of Several Variables: Their Limits and Continuity}
\section{The Space $ \Real^m $ and the Most Important Classes of Its Subsets}
\section{Limits and Continuity of Functions of Several Variables}


\chapter{The Differential Calculus of Functions of Several Variables}
\section{The Linear Structure on $ \Real^m $}
\section{The Differential of a Function of Several Variables}
\section{The Basic Laws of Differentiation}
\section{The Basic Facts of Differential Calculus of Real-Valued Functions of Several Variables}
\section{The Implicit Function Theorem}
\section{Some Corollaries of the Implicit Function Theorem}
\section{Surfaces in $ \Real^n $ and the Theory of Extrema with Constraint}





\end{document}