%%%%%%%%%%%%%%%%%%%%%%%%%%%%%%%%%%%%%%%%%%%%%%%%%%%
%% LaTeX book template                           %%
%% Author:  Amber Jain (http://amberj.devio.us/) %%
%% License: ISC license                          %%
%%%%%%%%%%%%%%%%%%%%%%%%%%%%%%%%%%%%%%%%%%%%%%%%%%%

\documentclass[a4paper,11pt]{book}
\usepackage[T1]{fontenc}
\usepackage[utf8]{inputenc}
\usepackage{lmodern}
%%%%%%%%%%%%%%%%%%%%%%%%%%%%%%%%%%%%%%%%%%%%%%%%%%%%%%%%%
% Source: http://en.wikibooks.org/wiki/LaTeX/Hyperlinks %
%%%%%%%%%%%%%%%%%%%%%%%%%%%%%%%%%%%%%%%%%%%%%%%%%%%%%%%%%
\usepackage{/Users/HechenHu/Development/NoteTaking/Mathematics-Notes/Customized}
%%%%%%%%%%%%%%%%%%%%%%%%%%%%%%%%%%%%%%%%%%%%%%%%
% Chapter quote at the start of chapter        %
% Source: http://tex.stackexchange.com/a/53380 %
%%%%%%%%%%%%%%%%%%%%%%%%%%%%%%%%%%%%%%%%%%%%%%%%

\makeatletter

\renewcommand{\@chapapp}{}% Not necessary...

\newenvironment{chapquote}[2][2em]
{\setlength{\@tempdima}{#1}%
	\def\chapquote@author{#2}%
	\parshape 1 \@tempdima \dimexpr\textwidth-2\@tempdima\relax%
	\itshape}
{\par\normalfont\hfill--\ \chapquote@author\hspace*{\@tempdima}\par\bigskip}
\makeatother

%%%%%%%%%%%%%%%%%%%%%%%%%%%%%%%%%%%%%%%%%%%%%%%%%%%
% First page of book which contains 'stuff' like: %
%  - Book title, subtitle                         %
%  - Book author name                             %
%%%%%%%%%%%%%%%%%%%%%%%%%%%%%%%%%%%%%%%%%%%%%%%%%%%

% Book's title and subtitle
\title{\Huge \textbf{Abstract Algebra}}
% Author
\author{\textsc{Hechen Hu}}

\begin{document}
	\frontmatter
	\maketitle
	%%%%%%%%%%%%%%%%%%%%%%%%%%%%%%%%%%%%%%%%%%%%%%%%%%%%%%%%%%%%%%%%%%%%%%%%
	% Auto-generated table of contents, list of figures and list of tables %
	%%%%%%%%%%%%%%%%%%%%%%%%%%%%%%%%%%%%%%%%%%%%%%%%%%%%%%%%%%%%%%%%%%%%%%%%
	
	\tableofcontents
	
	\mainmatter
	
	
	
	
	
\chapter{Groups}
\section{Semigroups, Monoids and Groups}
\begin{Definition}
	A \textit{semigroup} is a nonempty set $ G $ together with a binary operation on $ G $ which is associative.
\end{Definition}
\begin{Definition}
	A \textit{monoid} is a semigroup $ G $ which contains a (two-sided) identity element $ e \in G $ such that $ ae=ea=a $ for all $ a \in G $.
\end{Definition}
\begin{Definition}
	A \textit{group} is a monoid $ G $ such that there exists a (two-sided) inverse element and the operation between the inverse element and the original element yields the identity element regardless of order of operation.
\end{Definition}
\begin{Definition}
	A semigroup $ G $ is said to be \textit{abelian} or \textit{commutative} if its binary operation is commutative.
\end{Definition}
\begin{Definition}
	The \textit{order} of a group $ G $ is the cardinal number $ |G| $. $ G $ is said to be finite(resp. infinite) if $ |G| $ is finite(resp. infinite).
\end{Definition}
\begin{Theorem}
	If $ G $ is a monoid, then the identity element $ e $ is unique. If $ G $ is a group, then
	\begin{itemize}
		\item $ c \in G $ and $ (cc=c)\Rightarrow (c=e) $;
		\item for all $ a,b,c\in G $ we have $ (ab=ac)\Rightarrow(b=c) $ and $ (ba=ca)\Rightarrow(b=c) $ (left and right cancellation);
		\item for each element in $ G $ its inverse element is unique;
		\item for each element in $ G $ the inverse of its inverse is itself;
		\item for $ a,b\in G $ we have $ (ab)^{-1}=b^{-1}a^{-1} $;
		\item for $ a,b\in G $ the equation $ ax=b $ and $ ya=b $ have unique solutions in $ G:x=a^{-1}b $ and $ y=b a^{-1} $.
	\end{itemize}
\end{Theorem}
\begin{Proposition}
	Let $ G $ be a semigroup. $ G $ is a group iff the following conditions hold:
	\begin{itemize}
		\item there exists an element $ e \in G $ such that $ ea=a $ for all $ a \in G $ (left identity element);
		\item for each $ a \in G $, there exists an element $ a^{-1} \in G$ such that $ a^{-1}a=e $ (left inverse).
	\end{itemize}
and an analogous result holds for "right inverses" and a "right identity".
\end{Proposition}

\begin{Proposition}
	Let $ G $ be a semigroup. $ G $ is a group iff for all $ a,b\in G $ the equations $ ax=b $ and $ ya=b $ have solutions in $ G $.
\end{Proposition}
\begin{proof}
	\exercise
\end{proof}

\begin{Example}
	Let $ S $ be a nonempty set and $ A(S) $ the set of all bijections $ S \to S $. Under the operation of composition of functions, $ \circ $, $ A(S) $ is a group. The elements of $ A(S) $ are called \textit{permutations} and $ A(S) $ is called the group of permutations on the set $ S $. If $ S=\{1,2,3,\cdots,n\} $, then $ A(S) $ is called the symmetric group on $ n $ letters and denoted $ S_n $. $ |S_n|=n! $.
\end{Example}
\begin{Definition}
	The \textit{direct product} of two groups $ G $ and $ H $ with identities $ e_G $ and $ e_H $ is the group whose underlying set is $ G \times H $ and whose binary operation is given by:
	\begin{equation}
		(a,b)(a^\prime,b^\prime)=(a a^\prime,b b^\prime),\quad \text{ where }a,a^\prime \in G;b, b^\prime \in H\nonumber
	\end{equation}
	$ G \times H $ is abelian if both $ G $ and $ H $ are; $ (e_G,e_H) $ is the identity and $ (a^{-1},b^{-1}) $ is the inverse of $ (a,b) $. Clearly $ |G \times H|=|G||H| $.
\end{Definition}

\begin{Theorem}
	Let $ R(\sim) $ be an equivalence relation on a monoid $ G $ such that $ a_1~a_2 $ and $ b_1~b_2 $ imply $ a_1 b_1 ~ a_2 b_2 $ for all $ a_i, b_i \in G $. Then the set $ G/R $ of all equivalence classes of $ G $ under $ R $ is a monoid under the binary operation defined by $ (\bar{a})(\bar{b})=\bar{ab} $, where $ \bar{x} $ denoted the equivalence class of $ x \in G $. If $ G $ is an [abelian] group, then so is $ G/R $.\newpara
	An equivalence relation on a monoid $ G $ that satisfies these hypothesis is called a \textbf{congruence relation} on $ G $.
	
\end{Theorem}
\begin{Example}
	The following relation on the additive froup $ \Quoziente $ is a congruence relation:
	\begin{equation}
		a \sim b \Leftrightarrow a-b \in \Zahlen \nonumber
	\end{equation}
	The set of equivalence classes (denoted $ \Quoziente / \Zahlen $) is an infinite abelian group, with addition given by $ \bar{a}+\bar{b}=\bar{a+b} $, and called the \textit{group of rationals modulo one}.
\end{Example}
\begin{Definition}
	The \textit{meaningful product} on any sequence of elements of a semigroup $ G $, $ \{a_1,a_2,\cdots\} $, $ a_1,\cdots,a_n $(in this order), is defined inductively as below: If $ n=1 $, the only meaningful product is $ a_1 $. If $ n>1 $, then a meaningful product is defined to be any product of the form $ (a_1\cdots a_m)(a_{m+1}\cdots a_n) $ where $ m<n $ and $ (a_1\cdots a_m) $ and $ (a_{m+1}\cdots a_n) $ are meaningful products of $ m $ and $ n-m $ elements respectively.
\end{Definition}
\begin{Definition}
	The \textit{standard $ n $ product} $ \prod_{i=1}^{n}a_i $ is defined as follows:
	\begin{equation}
		\prod_{i=1}^{1}a_i=a_i;\quad\text{ for }n>1,\prod_{i=1}^{n}a_i = (\prod_{i=1}^{n-1}a_i)a_n \nonumber
	\end{equation}
\end{Definition}

\begin{Theorem}[Generalized Associative Law]
	If $ G $ is a semigroup and $ a_1,\cdots,a_n \in G $, then any two meaningful products of $ a_1,\cdots,a_n  $ in this order are equal.
\end{Theorem}
\begin{Theorem}[Generalized Commutative Law]
	If $ G $ is a commutative semigroup and $ a_1,\cdots,a_n \in G $, then for any permutation $ i_1,\cdots,i_n $ of $ 1,2,\cdots,n $, $ a_1 a_2 \cdots a_n = a_{i_1}a_{i_2}\cdots a_{i_n} $.
\end{Theorem}
\begin{Definition}
	Let $ G $ be a semigroup, $ a \in G$ and $ n \in \Natural $. The element $ a^n \in G $ is defined to be the standard $ n $ product $ \prod_{i=1}^{n}a_i $ with $ a_i=a  $ for $ 1 \leqslant i \leqslant n $. If $ G $ is a monoid, $ a^0 $ is defined to be the identity element $ e $. If $ G $ is a group, then for each $ n \in \Natural $, $ a^{-n} $ is defined to be $ (a^{-1})^n \in G $.
\end{Definition}
\begin{Theorem}
	If $ G $ is a group(resp. semigroup, monoid) and $ a \in G $, then for all $ m,n \in \Zahlen $ (resp. $ \Natural $ and $ \Natural \cup \{0\} $) :
	\begin{itemize}
		\item $ a^m a^n = a^{m+n} $ 
		\item $ (a^m)^n=a^{mn} $
	\end{itemize}
\end{Theorem}





\section{Homomorphisms and Subgroups}
\begin{Definition}
	Let $ G $ and $ H $ be semigroups. A function $ f:G \to H $ is a \textit{homomorphism} provided
	\begin{equation}
		f(ab)=f(a)f(b) \text{ for all }a,b \in G\nonumber
	\end{equation}
	If $ f $ is injective as a map of sets, $ f $ is said to be a \textit{monomorphism}. If $ f $ is surjective, $ f $ is called an \textit{epimorphism}. If $ f $ is bijective, $ f $ is called an \textit{isomorphism}. In this case $ G $ and $ H $ are said to be \textit{isomorphic} (written $ G \cong H $). A homomorphism $ f:G \to G $ is called an \textit{endomorphism} of $ G $ and an isomorphism $ f:G \to G $ is called an \textit{automorphism} of $ G $.
\end{Definition}
\begin{Definition}
	Let $ f:G \to H $ be a homomorphism of groups. The \textit{kernel} of $ f $(denoted $ \Ker{f} $) is $ \{a\in G | f(a) = e \in H  \} $. If $ A $ is a subset of $ G $, then $ f(A)=\{b \in H | b= f(a) \text{ for some }a \in A \} $  is the \textit{image of $ A $}. $ f(G) $ is called the \textit{image of $ f $} and denoted $ \Im{f} $. If $ B $ is a subset of $ H $, $ f^{-1}(B)=\{a \in G |f(a) \in B  \} $ is the \textit{inverse image} of $ B $.
\end{Definition}
\begin{Theorem}
	Let $ f:G \to H $ be a homomorphism of groups. Then
	\begin{itemize}
		\item $ f $ is a monomorphism iff $ \Ker{f}=\{e\} $.
		\item $ f $ is an isomorphism iff there is a homomorphism $ f^{-1}:H \to G $ such that $ f f^{-1}=1_H $ and $ f^{-1}f=1_G $.
	\end{itemize}
\end{Theorem}
\begin{Definition}
	Let $ G $ be a semigroup and $ H $ a nonempty subset of it. If for every $ a,b \in H $ we have $ ab \in H $, we say that $ H $ is \textit{closed} under the product in $ G $. This is the same as saying that the binary operation on $ G $, when restricted to $ H $, is a binary operation on $ H $.
\end{Definition}
\begin{Definition}
	Let $ G $ be a group and $ H $ a nonempty subset that is closed under the product in $ G $. If $ H $ is itself a group under the product in $ G $, then $ H $ is said to be a \textit{subgroup} of $ G $, denoted $ H<G $.
\end{Definition}
\begin{Definition}
	If a subgroup $ H $ is not $ G $ itself or the \textit{trivial subgroup}, which consists only of the identity element, is called a \textit{proper subgroup}.
\end{Definition}
\begin{Theorem}
	Let $ H $ be a nonempty subset of a group $ G $. Then $ H $ is a subgroup of $ G $ iff $ a b^{-1}\in H $ for all $ a,b \in H $.
\end{Theorem}
\begin{Corollary}
	If $ G $ is a group and $ \{H_i|i \in I \} $ is a nonempty family of subgroups, then $ \underset{i \in I}{\bigcap}H_i $ is a subgroup of $ G $.
\end{Corollary}


\begin{Definition}
	Let $ G $ be a group and $ X $ a subset of $ G $. Let $ \{H_i|i \in I \} $ be the family of all subgroups of $ G $ which contain $ X $. Then $ \underset{i \in I}{\bigcap}H_i $ is called the \textit{subgroup of $ G $ generated by the set $ X $} and denoted $ \angbracket{X}$. The elements of $ X $ are the \textit{generators} of $ \angbracket{X} $. If $ G=\angbracket{a_1,\cdots,a_n} ,(a_i \in G)$, $ G $ is said to be finitely generated. If $ a \in G $, the subgroup $ \angbracket{a} $ is called the \textit{cyclic (sub)group} generated by $ a $.
\end{Definition}

\begin{Theorem}
	If $ G $ is a group and $ X $ a nonempty subset of $ G $, then the subgroup $ \angbracket{X} $ generated by $ X $ consists of all finite products $ a_1^{n_1} a_2^{n_2}\cdots a_t^{n_t}(a_i \in X;n_i \in \Zahlen)$. In particular for every $ a \in G $, $ \angbracket{a}=\{a^n|n\in \Zahlen \} $.
\end{Theorem}

\begin{Definition}
	The subgroup $ \angbracket{\underset{i \in I}{\bigcap}H_i } $ generated by the set $ \underset{i \in I}{\bigcap}H_i  $ is called the \textit{subgroup generated by the groups $ \{H_i | i \in I \} $}. If $ H $ and $ K $ are subgroups, the subgroup $ \angbracket{H \cup K} $ generated by $ H $ and $ K $ is called the \textit{join} of $ H $ and $ K $ and is denoted $ H \lor K $.
\end{Definition}



\section{Cyclic Groups}
\begin{Definition}
	A \textit{cyclic group} or \textit{monogenous group} is a group that is generated by a single element. That is, it consists of a set of elements with a single invertible associative operation, and it contains an element such that every other element of the group may be obtained by repeatedly applying the group operation or its inverse to it.
\end{Definition}
\begin{Theorem}
	Every subgroup $ H $ of the additive group $ \Zahlen $ is cyclic. Either $ H = \angbracket{0} $ or $ H = \angbracket{m} $, where $ m $ is the least positive interger in $ H $. If $ H \neq \angbracket{0} $, then $ H $ is infinite.
\end{Theorem}
\begin{Theorem}
	Every infinite cyclic group is isomorphic to the additive group $ \Zahlen $ and every finite group of order $ m $ is isomorphic to the additive group $ \Zahlen_m $.
\end{Theorem}

\begin{Definition}
	Let $ G $ be a group and $ a \in G $. The \textit{order} of $ a $ is the order of the cyclic subgroup $ \angbracket{a} $ and is denoted $ |a| $.
\end{Definition}
\begin{Theorem}
	Let $ G $ be a group and $ a \in G $. If $ a $ has infinite order, then
	\begin{itemize}
		\item $ a^k=e $ iff $ k=0 $;
		\item the elements $ a^k (k \in \Zahlen) $ are all distinct.
	\end{itemize}
If $ a $ has inite order $ m>0 $, then
\begin{itemize}
	\item $ m $ is the least positive integer such that $ a^m=e $;
	\item $ a^k=e $ iff $ m|k $;
	\item $ a^r = a^s $ iff $ r \equiv s(\text{mod} m) $;
	\item $ \angbracket{a} $ consists of the distinct elements $ a,a^2,\cdots,a^{m-1},a^m=e $;
	\item for each $ k $ such that $ k|m $, $ |a^k| = m/k $.
\end{itemize}
\end{Theorem}
\begin{Theorem}
	Every homomorphic image and every subgroup of a cyclic group $ G $ is cyclic. In particular, if $ H $ is a nontrivial subgroup of $ G = \angbracket{a} $ and $ m $ is the least positive integer such that $ a^m \in H $, then $ H = \angbracket{a^m} $.
\end{Theorem}
\begin{Theorem}
	Let $ G = \angbracket{a} $ be a cyclic group. If $ G $ is infinite, then $ a $ and $ a^{-1} $ are the only generators of $ G $. If $ G $ is finite of order $ m $, then $ a^k $ is a generator of $ G $ iff $ (k,m)=1 $.
\end{Theorem}

\section{Cosets and Counting}
\begin{Definition}
	Let $ H $ be a subgroup of a group $ G $ and $ a,b \in G $. $ a $ is \textit{right congruent to $ b $ modulo $ H $}, denoted $ a \equiv_r b (\text{mod }H) $ if $ a b^{-1}\in H $. $ a $ is \textit{left congruent to $ b $ modulo $ H $ }, denoted $ a \equiv_l b (\text{mod }H) $ if $  a^{-1}b\in H $.
\end{Definition}
\begin{Theorem}
	Let $ H $ be a subgroup of a group $ G $.
	\begin{itemize}
		\item Right(resp. left) congruence modulo $ H $ is an equivalence relation on $ G $.
		\item The equivalence class of $ a \in G $ under right(resp. left) congruence modolo $ H $ is the set $ Ha = \{ha |h\in H \} $(resp. $ aH = \{ah |h\in H \} $).
		\item $ |Ha| = |H| = |aH| $ for all $ a \in G $.
	\end{itemize}
\end{Theorem}
\begin{Definition}
	The set $ Ha $ above is called a \textit{right coset} of $ H $ in $ G $ and $ aH $ is called an \textit{left coset} of $ H $ in $ G $.
\end{Definition}
\begin{Corollary}
	Let $ H $ be a subgroup of a group $ G $.
	\begin{itemize}
		\item  $ G $ is the union of the right(resp. left) cosets of $ H $ in $ G $.
		\item Two right(resp. left) cosets of $ H $ in $ G $ are either disjoint or equal.
		\item For all $ a,b\in G $, $ (Ha = Hb) \Leftrightarrow (a b^{-1}\in H)$ and $  (aH = bH) \Leftrightarrow ( a^{-1}b\in H) $.
		\item If $ \mathcal{R} $ is the set of distinct right cosets of $ H $ in $ G $ and $ \mathcal{L} $ is the set of distinct left cosets of $ H $ in $ G $, then $ |\mathcal{R}|=|\mathcal{L}| $.
	\end{itemize}
\end{Corollary}
\begin{proof}
	The first three statements are consequences of properties of equivalence classes. For $ (iv) $ it's easy to see that the map $ \mathcal{R}\to\mathcal{L} $ given by $ Ha \to a^{-1}H $ is a bijection.
\end{proof}
\begin{Definition}
	Let $ H $ be a subgroup of a group $ G $. The \textit{index of $ H $ in $ G $}, denoted $ [G:H] $, is the cardinal number of the set of distince right(resp. left) cosets of $ H $ in $ G $.
\end{Definition}
\begin{Definition}
	A \textit{complete set of right coset representatives} of a subgroup $ H $ in a group $ G $ is a set $ \{a_i\} $ consisting of precisely one element from each right coset of $ H $ in $ G $ and having cardinality $ [G:H] $.
\end{Definition}
\begin{Theorem}
	If $ K,H,G $ are groups with $ K<H<G $, then $ [G:K]=[G:H][H:K] $. If any two of these indices are finite, then so is the third.
\end{Theorem}
\begin{proof}
	By previous Corollary $ G = \bigcup_{i \in I}H a_i $ with $ a_i \in G $, $ |I| = [G:H] $ and the cosets $ Ha_i $ are mutually disjoint. Similarly $ H = \bigcup_{j \in J}Kb_j $ with $ b_j \in H $, $ |J| = [H:K] $ and the cosets $ Kb_j $ are mutually disjoint. Therefore $ G = \bigcup_{i \in I}Ha_i = \bigcup_{i \in I}(\bigcup_{j \in J}Kb_j)a_i =\bigcup_{(i,j)\in I \times J}Kb_j a_i$. It suffices to show that the cosets $ Kb_ja_i $ are mutually disjoint, for then we must have $ [G:K]=|I\times J| = |I||J| = [G:H][H:K] $. If $ Kb_ja_i=Kb_ra_t $, then $ b_j a_i = kb_r a_t(k \in K) $ (because $ Kk = Ke=K $). Since $ b_j,b_r,k \in H $ we have $ H a_i = H b_ja_i = H k b_r a_t = H a_t $, hence $ i=t $ and $ b_j = kb_r $. Thus $ K b_j = Kkb_r=Kb_r $ and $ j = r $. Therefore the cosets $ Kb_ja_i $ are mutually disjoint. The last statement of the theorem is obvious.
\end{proof}
\begin{Corollary}[Lagrange]
	If $ H $ is a subgroup of a group $ G $, then $ |G| = [G:H]|H| $. In particular if $ G $ is finite, the order $ |a| $ of $ a \in G $ devides $ |G| $.
\end{Corollary}
\begin{proof}
	Apply the theorem with $ K = \langle e\rangle $ for the first statement. The second is a special case of the first with $ H = \langle a\rangle $.
\end{proof}
\begin{Theorem}
	If the set $ \{ab|a\in H,b \in K \} $ is denoted $ HK $, then for two finite subgroups $ H $ and $ K $ of a group $ G $ $ |HK|=|H||K|/|H \cap K| $.
\end{Theorem}
\begin{proof}
	$ C = H \cap K $ is a subgroup of $ K $ of index $ n = |K|/|H \cap K|$ (apply the Lagrange Corollary) and $ K $ is the disjoint union of right cosets $ Ck_1 \cup C k_2 \cup \cdots \cup Ck_n $ for some $ k_i \in K $ (because $ C $ is a subgroup of $ K $). Since $ HC = H $, this implies that $ HK=HCk_1 \cup HCk_2 \cup \cdots \cup HCk_n =  Hk_1 \cup Hk_2 \cup \cdots \cup Hk_n$, which are disjoint. Therefore, $ |HK| = |H|\cdot n = |H||K|/|H \cap K| $.
\end{proof}
\begin{Proposition}
	If $ H $ and $ K $ are subgroups of a group $ G $, then $ [H:H\cap K]\leqslant [G:K] $. If $ [G:K] $ is finite, then $ [H:H\cap K]= [G:K]  $ iff $ G = KH $.
\end{Proposition}
\begin{proof}
	Let $ A $ be the set of all right cosets of $ H \cap K $ in $ H $ and $ B $ the set of all right cosets of $ K $ in $ G $. The map $ \varphi:A \to B $ given by $ (H \cap K)h \mapsto Kh(h \in H) $ is well defined since $ (H \cap K)h^\prime =  (H \cap K)h$ implies $ h^\prime h^{-1}\in H\cap K \subset K $ and hence $ Kh^\prime = Kh $. To show that $ \varphi $ is injective, noted that for $ Kh^\prime = Kh (h \in H) $ we have $ Hh^\prime = Hh $, thus $ Kh^\prime \cap Hh^\prime = Kh \cap Hh \Leftrightarrow (H \cap K)h^\prime = (H \cap K)h $. Then $ [H:H \cap K]=|A| \leqslant |B| = [G:K] $. If $ [G:K] $ is finite, clearly $ [H:H\cap K]=[G:K] $ iff $ \varphi $ is surjective. Suppose that $ \varphi $ is surjective but $ G \neq KH $. Then there exist an element $ g \in G $ such that $ g \neq kh $ for all $ k \in K, h \in H $. Then $ Kg \neq K(kh)\Leftrightarrow Kg \neq Kh $. Since $ h $ is arbitrary, the non-existence of $ \varphi^{-1}(Kg) $ and that $ Kg \in B $ contradicts with the fact that $ \varphi $ is a bijection. If $ G = KH $, we have that for any $ Kg \in B(g \in G) $ the mapping $ \varphi^{-1} $ is defined, thus it must be surjective.
\end{proof}
\begin{Proposition}
	Let $ H $ and $ K $ be subgroups of finite index of a group $ G $. Then $ [G:H\cap K] $ is finite and $ [G:H\cap K] \leqslant [G:H][G:K] $. Furthermore, $ [G:H\cap K] =[G:H][G:K]$ iff $ G=HK $.
\end{Proposition}
\begin{proof}
	This proposition is an easy consequence of previous proposition and theorem.
\end{proof}



\section{Normality, Quotient Groups, and Homomorphisms}
\begin{Theorem}
	If $ N $ is a subgroup of a group $ G $, then the following conditions are equivalent.
	\begin{enumerate}
		\item Left and right congruence modulo $ N $ coincide (that is, define the same equivalence relation on $ G $);
		\item every left coset of $ N $ in $ G $ is also a right coset of $ N $ in $ G $;
		\item $ aN = Na $ for all $ a\in G $;
		\item for all $ a\in G $, $ aNa^{-1}\subset N $;
		\item for all $ a \in G $, $ aN^{-1}=N $
	\end{enumerate}
\end{Theorem}
\begin{Definition}
	A subgroup $ N $ of a group $ G $ which satisfies the equivalent conditions of the previous theorem is said to be \textit{normal} in $ G $ (or a \textit{normal subgroup} in $ G $); we write $ N\vartriangleleft G$ if $ N $ is normal in $ G $.
\end{Definition}
\begin{Theorem}
	Let $ K $ and $ N $ be subgroups of a group $ G $ with $ N $ normal in $ G $. Then
	\begin{enumerate}
		\item $ N \cap K $ is a normal subgroup of $ K $;
		\item $ N $ is a normal subgroup of $ N \vee K $;
		\item $ NK = N \vee K = KN $;
		\item if $ K $ is normal in $ G $ and $ K \cap N = \langle e\rangle $, then $ nk=kn $ for all $ k\in K $ and $ n \in N $.
	\end{enumerate}
\end{Theorem}
\begin{proof}
	\begin{enumerate}
		\item If $ n \in N\cap K $, then for an element $ k \in K <G $ $ knk^{-1}\in N $ and $ knk^{-1}\in K $. Thus $ k (N \cap K)k^{-1}\subset N \cap K $ and $ N \cap K \vartriangleleft K $.
		\item It is trivial.
		\item All elements of $ N \vee K $ are of the form $ n_1 k_1 n_2 k_2 \cdots n_r k_r $, where $ n_i \in N $ and $ k_i \in K $. Since $ N \vartriangleleft G $, $ n_i k_j = k_j n_i^\prime $ (because $ k_j n_i^\prime k_j^{-1}\in N $) and therefore these elements can be written in the form $ n(k_1\cdots k_r) ,n \in N$. Thus $ N \vee K \subset NK $. Since $ NK \subset N \vee K $, we have $ NK =  N \vee K$. Similarly $ KN = N \vee K $.
		\item Let $ k \in K $ and $ n \in N $. Then $ nkn^{-1}\in K $ and $ k n^{-1}k^{-1}\in N $. Hence $ (nkn^{-1}) =n(kn^{-1}k^{-1})\in N \cap K = \langle e \rangle $, which implies $ nk = kn $.
	\end{enumerate}
\end{proof}
\begin{Theorem}
	If $ N \vartriangleleft G $ and $ G/N $ is the set of all (left) cosets of $ N $ in $ G $, then $ G / N $ is a group of order $ [G:N] $ under the binary operation given by $ (aN)(bN)=abN $.
\end{Theorem}
\begin{proof}
	Since the coset $ aN $ is simply the equivalence class of $ a \in G $ under the equivalence relation of congruence modulo $ N $, it suffices to show that congruence modulo $ N $ is a congruence relation, that is, that $ a_1 \equiv a \pmod{N} $ and $ b_1 \equiv b \pmod{N} $ imply $ a_1 b_1 \equiv ab \pmod{N} $. By assumption $ a_1 a^{-1}=n_1 \in N $ and $ b_1 b^{-1}=n_2 \in N $. Hence $ (a_1 b_1)(ab)^{-1} =a_1 b_1 b^{-1}a^{-1}=(a_1 n_2)a^{-1}$. But since $ N $ is normal, $ a_1N = Na_1 $ which implies that $ a_1 n_2 = n_3 a_1$ for some $ n_3 \in N $. Consequently $ (a_1 b_1)(ab)^{-1} = (a_1 n_2)a^{-1}=n_3 a_1 a^{-1} = n_3 n_1 \in N $, whence $ a_1 b_1 \equiv ab \pmod{N} $.
\end{proof}
\begin{Definition}
	If $ N $ is a normal subgroup of a group $ G $, then the group $ G/N $ as defined before is called the \textit{quotient group} of \textit{factor group} of $ G $ by $ N $. If $ G $ is written additively, then the group operation in $ G /N $ is given by $ (a+N)+(b+N)=(a+b)+N $.
\end{Definition}
\begin{Example}
	$ \Zahlen_m = \Zahlen / \langle m \rangle $.
\end{Example}
\begin{Theorem}
	If $ f:G \to H $ is a homomorphism of groups, then the kernel of $ f $ is a normal subgroup of $ G $. Conversely, if $ N $ is normal in $ G $, then the map $ \pi:G \to G/N $ given by $ \pi(a)=aN $ is an epimorphism with kernel $ N $.
\end{Theorem}
\begin{proof}
	If $ x \in \Ker f $ and $ a \in G $, then clearly $ f(axa^{-1})=e $ and therefore $ \ker f \vartriangleleft G $. The map $ \pi $ is clearly surjective and since $ \pi(ab)=ab N = aNbN = \pi(a)\pi(b) $, it is an epimorphism. $ \Ker \pi = \{a \in G |\pi(a)=eN=N \}=\{a\in G |aN = N \}=\{a \in G| a \in N \}=N $.
\end{proof}
\begin{Definition}
	The map $ \pi:G \to G/N $ is called the \textit{canonical epimorphism} or \textit{projection}. Unless otherwise stated $ G \to G/N(N \vartriangleleft G) $ always denotes the canonical epimorphism.
\end{Definition}
\begin{Theorem}
	If $ f:G \to H $ is a homomorphism of groups and $ N $ is a normal subgroup of $ G $ contained in the kernel of $ f $, then there is a unique homomorphism $ \bar{f}:G/N \to H $ such that $ \bar{f}(aN)=f(a) $ for all $ a \in G $. $ \Im f = \Im \bar{f} $ and $ \Ker \bar{f}=(\Ker f)/N $. $ \bar{f} $ is an isomorphism iff $ f$ is an epimorphism and $ N = \Ker f$. 
\end{Theorem}
This essential part of the conclusion may be rephrased: there exists a unique homomorphism $ \bar{f}:G/N \to H $ such that the diagram
\begin{equation}
	\begin{tikzcd}[row sep=3.5em,column sep=3.5em]
	G \arrow[r,"f"] \arrow[d] & H \\
	G/H \arrow[ru,"\bar{f}"']
	\end{tikzcd} \nonumber
\end{equation}
is commutative.
\begin{proof}
	If $ b \in aN $, then $ b=an $, $ n \in N $, and $ f(b)=f(an)=f(a)f(n)=f(a) $ since $ N < \Ker f $. Therefore, $ f $ has the same effect on every element of the coset $ aN $ and the map $ \bar{f}:G/N \to H $ given by $ \bar{f}(aN)=f(a) $ is a well-defined function. Since $ \bar{f}(aNbN)=\bar{f}(abN)=f(ab)=f(a)f(b)=\bar{f}(aN)\bar{f}(bN) $, $ \bar{f} $ is a homomorphism. Clearly $ \Im f = \Im \bar{f} $ and
	\begin{equation}
		aN \in \Ker \bar{f} \Leftrightarrow f(a)=e \Leftrightarrow a \in \Ker f \nonumber
	\end{equation}
	whence $ \Ker \bar{f}=\{aN|a \in \Ker f \} = (\Ker f)/N $. $ \bar{f} $ is unique since it is completely determined by $ f $. It is clear that $ \bar{f} $ is an epimorphism iff $ f$ is. $ \bar{f}$ is a monomorphism if its kernel is the trivial subgroup of $ G/N $, which occurs iff $ \Ker f = N $.
\end{proof}
\begin{Definition}
	A \textit{commutative diagram} is a diagram consists of arrows and objects such that all directed paths that has the same start and endpoints yield the same result.
\end{Definition}
\begin{Corollary}(First Isomorphism Theorem)
	If $ f:G \to H $ is a homomorphism of groups, then $ f $ induces an isomorphism $ G/(\Ker f)\cong \Im f $.
\end{Corollary}
\begin{proof}
	Clearly $ f:G \to \Im f $ is an epimorphism. Apply the previous theorem with $ N=\Ker f $ yield the desired result.
\end{proof}
\begin{Corollary}
	If $ f:G \to H $ is a homomorphism of groups, $ N \vartriangleleft G $, $ M \vartriangleleft H $, and $ f(N)<M $, then $ f $ induces a homomorphism $ \bar{f}:G/N \to H/M $, given by $ aN \mapsto f(a)M $.\\
	$ \bar{f} $ is an isomorphism iff $ \Im f \vee M = H$ and $ f^{-1}(M)\subset N $. In particular if $ f $ is an epimorphism such that $ f(N)=M $ and $ \Ker f \subset N $, then $ \bar{f} $ is an isomorphism.
\end{Corollary}
\begin{proof}
	Consider the composition $ G \overset{f}{\to} H \overset{\pi}{\to}H/M$. $ N \subset f^{-1}(M) $ because $ f(N)<M $. Apparently $ \pi f(a) = f(a)M $, then the kernel of $ \pi f $ consists of those elements whose image is in $ M $, which is equivalent to $ \Ker \pi f = f^{-1}(M) $. Apply the previous theorem to $ \pi f $ the map $ G/N \to H/M $ given by $ aN \mapsto \pi f(a)=f(a)M $ is a homomorphism that is an isomorphism iff $ \pi f $ is an epimorphism and $ N = \Ker \pi f $. But the latter conditions hold iff $ \Im f \vee M = H $ and $ f^{-1}(M)\subset N $: the second part is trivial; for the first one, $ \pi f $ is an epimorphism implies that there exists some $ g \in G $ such that $ \pi f(g)=hM $ for all distinct cosets in $ H/M $, then $ H = \Im f M = \Im f \vee M $; conversely, if $ \Im f \vee M = H $, we have $ \Im f M =H $, which says that there exist some elements $ f(a_1),f(a_2),\cdots $ in $ \Im f $ that are the distinct cosets of $ M $ in $ H $, which says that there must exists some $ a_i $ for any $ hM \in H/M $. If $ f $ is an epimorphism, then $ H = \Im f = \Im f \vee M $. If $ f(N )=M $ and $ \Ker f \subset N $, then $ f^{-1}(M)\subset N $, whence $ \bar{f} $ is an isomorphism.
\end{proof}
\begin{Corollary}(Second Isomorphism Theorem)
	If $ K $ and $ N $ are subgroups of a group $ G $ with $ N \vartriangleleft G $, then $ K/(N \cap K)\cong NK/N $.
\end{Corollary}
\begin{proof}
	$ N \vartriangleleft NK = N \vee K $. The composition $ K \overset{h}{\to}NK \overset{\pi}{\to} NK/N $ is a homomorphism $ f $ with kernel $ K \cap N $, whence $ \bar{f}:K/K\cap N \cong \Im f $. Every element in $ NK/N $ is of the form $ nkN $. The normality of $ N $ implies that $ nk=kn_1 $, whence $ nkN = kn_1N=kN=f(k) $. Therefore $ f $ is an epimorphism and hence $ \Im f = NK/N $.
\end{proof}
\begin{Corollary}(Third Isomorphism Theorem)
	If $ H $ and $ K $ are normal subgroups of a group $ G $ such that $ K<H $, then $ H /K $ is a normal subgroup of $ G/K $ and $ (G/K)/(H/K)\cong G/H $.
\end{Corollary}
\begin{proof}
	The identity map $ 1_G:G \to G $ has $ 1_G(K)<H $ and therefore induces an epimorphism $ I:G/K \to G/H $, with $ I(aK)=aH $. Since $ H=I(aK) $ iff $ a \in H $, $ \Ker I = \{aK|a \in H \}=H/K $. Hence $ H/K \vartriangleleft G/K $ and $ G/H=\Im I \cong (G/K)/\Ker I = (G/K)/(H/K) $.
\end{proof}
\begin{Theorem}
	If $ f:G \to H $ is an epimorphism of groups, then the assignment $ K \mapsto f(K) $ defines a bijection between the set $ S_f(G) $ of all subgroups $ K $ of $ G $ which contain $ \Ker f $ and the set $ S(H) $ of all subgroups of $ H $. Under the bijection normal subgroups correspond to normal subgroups.
\end{Theorem}
\begin{proof}
	The assignment $ K \mapsto f(K)$ defines a function $ \varphi: S_f(G)\to S(H) $ and $ f^{-1}(J) $ is a subgroup of $ G $ for every subgroup $ J $ of $ H $. Since $ J<H $ implies $ \Ker f <f^{-1}(J) $ and $ f(f^{-1}(J))=J $, $ \varphi $ is surjective (since for any subgroup $ J<H $ we have another subgroup $ J<H $ such that it is the image of a subgroup $ f^{-1}(J) $ in $ G $). $ f^{-1}(f(K))=K \Ker f $ since $ f(K \Ker f )=f(K) f(\Ker f) $, $ f^{-1}(f(K))=K $ iff $ \Ker f <K $. It follows that $ \varphi $ is injective. If $ K \vartriangleleft G $, then $ f(K) = f(gKg^{-1})=f(g)f(K)f(g)^{-1}= f(K) $. The argument for $ J \vartriangleleft H$ and for $ f^{-1}(J) $ is similar.
\end{proof}

\begin{Corollary}
	If $ N $ is a normal subgroup of a group $ G $, then every subgroup of $ G/N $ is of the form $ K/N $, where $ K $ is a subgroup of $ G $ that contains $ N $. Furthermore, $ K/N $ is normal in $ G/N $ iff $ K $ is normal in $ G $.
\end{Corollary}
\begin{proof}
	Apply the theorem above to the canonical epimorphism $ \pi :G\to G/N $. If $ N<K<G $, then $ \pi(K)=K/N $.
\end{proof}










\section{Symmetric, Alternating, and Dihedral Groups}
\begin{Definition}
	Let $ i_1,i_2,\cdots, i_r $, $ (r<n) $ be distinct elements of $ I_n=\{1,2,\cdots,n \} $. Then $ (i_1 i_2 i_3 \cdots i_r) $ denotes the permutation that moves each element to the element on its right ($ i_1 $ to $ i_2 $, $ i_r $ to $ i_1 $, etc.) and fix elements in $ I_n $ that are not in $ i_1 ,\cdots,i_r$. $ (i_1i_2 \cdots i_r) $ is called a \textit{cycle} of length $ r $ or an \textit{$ r $-cycle}; a $ 2 $-cycle is called a \textit{transposition}.
\end{Definition}
\begin{Definition}
	The permutations $ \sigma_1,\sigma_2,\cdots,\sigma_r $ of $ S_n $ are said to be \textit{disjoint} provided that for each $ 1\leqslant i \leqslant r $, and every $ k\in I_n $, $ \sigma_i(k)\neq k $ implies $ \sigma_j(k)=k $ for all $ j \neq i $. In other words, an element of $ I_n $ will only be moved once if we apply all of $ \sigma_i$ to it. In this case the composition of disjoint permutations commutes.
\end{Definition}
\begin{Theorem}
	Every nonidentity permutation in $ S_n $ is uniquely a product of disjoint cycles, each of which has length at least $ 2 $.
\end{Theorem}
\begin{Definition}
	Define an equivalence relation on $ I_n $ with a given permutation $ \sigma $ as follows: $ x \sim y $ iff $ y=\sigma^m(x) $ for some $ m \in \Zahlen $. The equivalence classes are called the \textit{orbit} for $ \sigma $ and form a partition of $ I_n $. Note that if $ x \in B_i $, then $ B_i $ consists of all elements $ \{x, \sigma(x),\sigma^2(x),\cdots,\sigma^d(x) \} $ for some $ d \in \Zahlen $.
\end{Definition}
\begin{Corollary}
	The order of a permutation $ \sigma \in S_n $ is the least common multiplier of the order of its disjoint cycles.
\end{Corollary}
\begin{Corollary}
	Every permutation in $ S_n $ can be written as a product of (not necessarily disjoint) transpositions.
\end{Corollary}
\begin{Definition}
	A permutation $ \tau \in S_n $ is said to be \textit{even} (resp. \textit{odd}) if $ \tau $ can be written as a product of an even (resp. odd) number of transpositions.
\end{Definition}
The \textit{sign of a permutation $ \tau $}, denoted $ \sgn \tau $, is $ 1 $ or $ -1 $ according as $ \tau $ is even or odd.
\begin{Theorem}
	A permutation $ \tau \in S_n (n \geqslant 2)$ cannot be both even and odd.
\end{Theorem}
\begin{Theorem}
	For each $ n \geqslant 2 $, let $ A_n $ be the set of all even permutations of $ S_n $. Then $ A_n $ is a normal subgroup of $ S_n $ of index $ 2 $ and order $ |S_n|/2=n!/2 $. Furthermore $ A_n $ is the only subgroup of $ S_n $ of index $ 2 $.
\end{Theorem}
The proof proceeds from the following two lemmas.
\begin{Lemma}
	If $ H < G$  and $ [G:H]=2 $, then $ H $ contains all squares of elements in $ G $.
\end{Lemma}
\begin{proof}
	Let $ g \in G$ be arbitrary. if $ gH=H $, the lemma is trivial. If $ gH=aH $ for some $ a \notin H $, then we have $ g^2 H = a^2 H $; if $ aH \neq a^2 H=H $, the lemma obviously holds. If $ aH = a^2 H $, the lemma also holds.
\end{proof}
\begin{Lemma}
	If $ H $ is a subgroup of index $ 2 $ in $ G $, then $ H $ contains all elements of odd order in $ G $.
\end{Lemma}
\begin{proof}
	Suppose that an element $ g \in G $ has order $ 2k+1 (k \in \Natural)$. Then $ H = g^{2k+1}H=(g^{k})^2 H \cdot gH = gH $.
\end{proof}
Then it follows that since any permutation of order $ 3 $(like all $ 3 $-cycles) is contained in any subgroup of index $ 2 $ (which must be normal) of $ S_n $, it must be $ A_n $.
\begin{Definition}
	A group is said to be \textit{simple} if it has no proper normal subgroups.
\end{Definition}
\begin{Theorem}
	The alternating group $ A_n $ is simple iff $ n \neq 4 $.
\end{Theorem}
The proof of the theorem proceeds from two lemmas below.
\begin{Lemma}
	Let $ r,s $ be distinct elements of $ \{1,2,\cdots,n\} $. Then $ A_n (n\geqslant 3)$ is generated bu the $ 3 $-cycles $ \{(rsk)|1\leqslant k \leqslant n,k \neq r,s\} $.
\end{Lemma}
\begin{Lemma}
	If $ N \triangleleft A_n(n\geqslant 3)$  and $ N $ contains a $ 3 $-cycle, then $ N=A_n $.
\end{Lemma}
\begin{Definition}
	The subgroup $ D_n $ of $ S_n (n \geqslant 3)$ generated by $ a=(123\cdots n) $ and
	\begin{align}
		b&= \begin{pmatrix}
		1 & 2 & 3 & \cdots & i &\cdots & n-1 & n \\
		1 & n & n-1 & \cdots & n+2-i & \cdots &  3  & 2
		\end{pmatrix} \nonumber\\
	&= \prod_{2\leqslant i < n+2-i} (i \quad n+2-i) \nonumber
	\end{align}
is called the \textit{dihedral group of degree $ n $}. The group $ D_n $ is isomorphic to and usually identified with the group of all symmetries of a regular polygon with $ n $ sides.
\end{Definition}
\begin{Theorem}
	For each $ n \geqslant 3 $ the dihedral group $ D_n $ is a group of order $ 2n $ whose generators $ a $ and $ b $ satisfy:
	\begin{enumerate}
		\item $ a^n=(1) $; $ b^2=(1)$; $ a^k\neq (1) $ if $ 0<k<n $($ (1) $ is the identity permutation);
		\item $ ba = a^{-1}b $
	\end{enumerate}
Any group $ G $ which is generated by elements $ a,b \in G $ satisfying both conditions for some $ n \geqslant 3 $ is isomorphic to $ D_n $.
\end{Theorem}


\section{Categories: Products, Coproducts, and Free Objects}
\begin{Definition}
	A \textit{category} is a class $ \mathscr{C} $ of objects (denoted $ \Cat{A},\Cat{B}, \Cat{C}, ... $) together with
	\begin{enumerate}
		\item a class of disjoint sets, denoted $ \hom(\Cat{A},\Cat{B}) $, one(set) for each pair of objects in $ \mathscr{C} $; an element of $ \hom (\Cat{A},\Cat{B}) $ is called a \textit{morphism} from $ \Cat{A} $ to $ \Cat{B} $ and denoted $ f:\Cat{A}\to \Cat{B} $. 
		\item for each triple $ (\Cat{A},\Cat{B},\Cat{C}) $ of objects of $ \mathscr{C} $ a function
		\begin{equation}
			\hom(\Cat{B},\Cat{C}) \times \hom(\Cat{A},\Cat{B})  \to \hom(\Cat{A},\Cat{C}) \nonumber
		\end{equation}
		which is the \textit{composite} of morphisms. The composition must be associative. An identity morphism $ 1_{\Cat{B}}:\Cat{B} \to \Cat{B} $ also exists and for any $ f:\Cat{A}\to \Cat{B} $ and $ g:\Cat{B}\to \Cat{C} $
		\begin{equation}
			1_{\Cat{B}} \circ f = f\qquad \text{and}\qquad g \circ 1_{\Cat{B}} = g \nonumber
		\end{equation}
	\end{enumerate}
\end{Definition}
\begin{Definition}
	In a category $ \mathscr{C} $ a morphism $ f:\Cat{A} \to \Cat{B} $ is called an \textit{equivalence} if there is in $ \mathscr{C} $ a morphism $ g:\Cat{B} \to \Cat{A}  $ such that $ g \circ f=1_{\Cat{A}} $ and $ f \circ g=1_{\Cat{B}} $. The composite of two equivalences, when defined, is an equivalence. If $ f:\Cat{A} \to \Cat{B}$ is an equivalence, $ \Cat{A} $ and $ \Cat{B} $ are said to be \textit{equivalent}.
\end{Definition}
\begin{Example}
	For $ \mathscr{S} $, the class of all sets, equipped with functions between these sets as morphisms, a morphism is an equivalence iff it is bijective. 
\end{Example}
\begin{Example}
	For $ \mathscr{G} $, the category of all groups, equipped with homomorphisms between these groups as morphisms, a morphism is an equivalence iff it is an isomorphism.
\end{Example}
\begin{Example}
	A (multiplicative) group $ G $ can be considered as a category with one object $ G $. Let $ \hom(G,G) $ be the set of elements of $ G $(then each morphism is an element of $ G $); composite of morphisms is simply the composition given by the binary operation in $ G $. Every morphism is an equivalence and $ 1_G $ is the identity element $ e $ of $ G $. \textbf{This example shows that morphisms need not to be functions}. In this case it is said that the category is not concrete.
\end{Example}

\begin{Definition}
	Let $ \mathscr{C} $ be a category and $ \{\Cat{A}_i|i\in I \} $ a family of objects of $ \mathscr{C} $. A \textit{product} for the family $ \{\Cat{A}_i|i\in I \} $ is an object $ \Cat{P} $ of $ \mathscr{C} $ together with a family of morphisms $ \{\pi_i: \Cat{P} \to \Cat{A_i}|i \in I \} $ such that for any object $ \Cat{B} $ and family of morphisms $ \{\varphi_i: \Cat{B} \to \Cat{A_i}|i \in I \} $, there is a unique morphism $ \varphi :\Cat{B} \to \Cat{P} $ such that $ \pi_i \circ \varphi = \varphi_i $ for all $ i \in I $. The product of $ \{\Cat{A}_i|i\in I \} $ is usually denoted $ \prod_{i\in I}\Cat{A_i} $.
\end{Definition}
When $ I=\{1,2 \} $, the product $ \Cat{P}$ expressed using a commutative diagram:
\begin{tikzpicture}
\begin{tikzcd}[row sep=3.5em,column sep=3.5em]
& \Cat{B} \arrow[ld,"\varphi_1"] \arrow[d,"\varphi"] \arrow[rd,"\varphi_2"] \\
\Cat{A_1} & \Cat{P} \arrow[l,"\pi_1"] \arrow[r,"\pi_2"] & \Cat{A_2} 
\end{tikzcd} \nonumber
\end{tikzpicture}
\newpara
In the category of sets the Cartesian product $ \prod_{i\in I} A_i $ is a product of the family of sets $ \{A_i \} $. The map $ \pi_i $ would be the canonical projections onto the $ i $th components. The map $ \varphi $ would be $ (\varphi_1,\varphi_2,\cdots,\varphi_i) $ that takes an element of $ B $ and maps it to an element of $ \prod_{i\in I} A_i $.
\begin{Theorem}
	If $ (\Cat{P},\{\pi_i \}) $ and $ (\Cat{Q},\{\psi_i\}) $ are both products of the family $ \{\Cat{A_i}|i\in I  \} $ of objects of a category $ \mathscr{C} $, then $ \Cat{P} $ and $ \Cat{Q} $ are equivalent.
\end{Theorem}
\begin{Definition}
	A \textit{coproduct} (or \textit{sum}) for the family $ \{\Cat{A_i}|i \in I \} $ of objects in a category $ \mathscr{C} $ is an object $ \Cat{S} $ of $ \mathscr{C} $, together with a family of morphisms $ \{\iota_i:\Cat{A_i}\to \Cat{S}|i \in I \} $ such that for any object $ \Cat{B} $ and family of morphisms $ \{\psi_i:\Cat{A_i}\to \Cat{B} |i \in I \} $, there is a unique morphism $ \psi:\Cat{S} \to \Cat{B} $ such that $ \psi \circ \iota_i = \psi_i $ for all $ i \in I $. Although no universal notation exists for coproducts, it is usually denoted $ \coprod_{i \in I }\Cat{A_i} $.
\end{Definition}
It's easy to see that by reversing the arrows in the commutative diagram above for product we obtain the diagram for coproduct. 
\begin{Theorem}
		If $ (\Cat{S},\{\iota_i \}) $ and $ (\Cat{S^\prime},\{\lambda_i\}) $ are both products of the family $ \{\Cat{A_i}|i\in I  \} $ of objects of a category $ \mathscr{C} $, then $ \Cat{S} $ and $ \Cat{S^{\prime}} $ are equivalent. 
\end{Theorem}
\begin{Definition}
	A \textit{concrete category} is a category $ \mathscr{C} $ together with a function $ \sigma $ that assigns to each object $ \Cat{A} $ of $ \mathscr{C} $ a set $ \sigma(\Cat{A}) $(called the underlying set of $ \Cat{A} $) in such a way that:
	\begin{enumerate}
		\item every morphism $ \Cat{A} \to \Cat{B} $ of $ \mathscr{C} $ is a function on the underlying sets $ \sigma(\Cat{A})\to \sigma(\Cat{B}) $;
		\item the identity morphism of each object $ \Cat{A} $ of $ \mathscr{C} $ is the identity function on the underlying set $ \sigma(\Cat{A}) $;
		\item composition of morphisms in $ \mathscr{C} $ agrees with composition of functions on their underlying sets.
	\end{enumerate}
\end{Definition}
\textbf{It is worth noticing that in a concrete category morphisms are also functions on their corresponding underlying sets, but maps, functions on these underlying sets, might not be morphisms.}
\begin{Definition}
	Let $ \Cat{F} $ be an object in a concrete category $ \mathscr{C} $, $ X $ a nonempty set, and $ i:X \to \Cat{F} $ a map (of sets). $ \Cat{F} $ is \textit{free on the set $ X $} provided that for any object $ \Cat{A} $ of $ \mathscr{C} $ and maps (of sets) $ f:X \to \Cat{A} $, there exists a unique morphism of $ \mathscr{C} $, $ \bar{f}:\Cat{F}\to \Cat{A} $, such that $ \bar{f}i=f $ (as a map of sets $ X \to \Cat{A} $).
\end{Definition}
\begin{tikzpicture}
\begin{tikzcd}[row sep=3.5em,column sep=3.5em]
 \Cat{F} \arrow[rd,"\bar{f}"] \\
X \arrow[u,"i"] \arrow[r,"f"]& \Cat{A}
\end{tikzcd} \nonumber
\end{tikzpicture}
\newpara
The essential fact about a free object $ \Cat{F} $ is that in order to define a morphism with domain $ \Cat{F} $, it suffices to specify the image of the subset $ i(X) $.
\begin{Example}
	Let $ G $ be any group and $ g \in G $. Then the map $ \bar{f}:\Zahlen \to G $ defined by $ \bar{f}(n)=g^n $ is the unique homomorphism $ \Zahlen \to G $ such that $ 1 \mapsto g $. Consequently, if $ X = \{1\} $ and $ i:X \to \Zahlen $ is the inclusion map, then $ \Zahlen $ is free on $ X $ in the category of groups. In other words, to determine a unique homomorphism from $ \Zahlen $ to $ G $ we need only specify the image of $ 1 \in \Zahlen $ (that is, the image of $ i(X) $).
\end{Example}
\begin{Theorem}
	If $ \mathscr{C} $ is a concrete category, $ \Cat{F} $ and $ \Cat{F^\prime} $ are objects of $ \mathscr{C} $ such that $ \Cat{F} $ is free on the set $ X $ and $ \Cat{F^\prime} $ is free on the set $ X^\prime $ and $ |X|=|X^\prime| $, then $ \Cat{F} $ is equivalent to $ \Cat{F^\prime} $.
\end{Theorem}
We have seen that two products(resp. coproducts) for a given family of objects are equivalent. Likewise two free objects on the same set are equivalent. This characteristic is captured via the following definition.
\begin{Definition}
	An object $ \Cat{I} $ in a category $ \mathscr{C} $ is said to be \textit{universal} (or \textit{initial}) if for each object $ \Cat{C} $ of $ \mathscr{C} $ there exists one and only one morphism $ \Cat{I} \to \Cat{C} $. An object $ \Cat{T} $ of $ \mathscr{C} $ is said to be \textit{couniversal} (or \textit{terminal}) if for each object $ \Cat{C} $ of $ \mathscr{C} $ there exists one and only one morphism $ \Cat{C} \to \Cat{T} $. 
\end{Definition}
\begin{Theorem}
	Any two universal (resp. couniversal) objects in a category $ \mathscr{C} $ are equivalent.
\end{Theorem}
\begin{Example}
	The trivial group is both universal and couniversal in the category of groups.
\end{Example}




\section{Direct Products and Direct Sums}
\section{Free Groups, Free Products, Generators and Relations}


\chapter{The Structure of Groups}
\section{Free Abelian Groups}
\section{Finitely Generated Abelian Groups}
\section{The Krull-Schmidt Theorem}
\section{The Action of a Group on a Set}
\section{The Sylow Theorem}
\section{Classification of Finite Groups}
\section{Nilpotent and Solvable Groups}
\section{Normal and Subnormal Series}


\chapter{Rings}
\section{Rings and Homomorphisms}
\section{Ideals}
\section{Factorization in Commutative Rings}
\section{Rings of Quotients and Localization}
\section{Rings of Polynomials and Formal Power Series}
\section{Factorization in Polynomial Rings}

\chapter{Modules}
\section{Modules, Homomorphisms and Exact Sequences}
\section{Free Modules and Vector Spaces}
\section{Projective and Injective Modules}
\section{Hom and Duality}
\section{Tensor Products}
\section{Modules over a Principal Ideal Domain}
\section{Algebras}


\chapter{Fields and Galois Theory}
\section{Field Extensions}
\section{The Fundamental Theorem}
\section{Splitting Fields, Algebraic Closure and Normality}
\section{The Galois Group of a Polynomial}
\section{Finite Fields}
\section{Separability}
\section{Cyclic Extensions}
\section{Cyclotomic Extensions}
\section{Radical Extensions}


\chapter{The Structure of Fields}
\section{Transcendence Bases}
\section{Linear Disjointness and Separability}

\chapter{Commutative Rings and Modules}
\section{Chain Conditions}
\section{Prime and Primary Ideals}
\section{Primary Decomposition}
\section{Noetherian Rings and Modules}
\section{Ring Extensions}
\section{Dedekind Domains}
\section{The Hilbert Nullstellensatz}


\chapter{The Structure of Rings}
\section{Simple and Primitive Rings}
\section{The Jacobson Radical}
\section{Semisimple Rings}
\section{The Prime Radical; Prime and Semiprime Rings}
\section{Algebras}
\section{Division Algebras}


\chapter{Categories}
\section{Functors and Natural Transformations}
\section{Adjoint Functors}
\section{Morphisms}
	
	
	
	
\end{document}