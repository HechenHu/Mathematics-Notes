%%%%%%%%%%%%%%%%%%%%%%%%%%%%%%%%%%%%%%%%%%%%%%%%%%%
%% LaTeX book template                           %%
%% Author:  Amber Jain (http://amberj.devio.us/) %%
%% License: ISC license                          %%
%%%%%%%%%%%%%%%%%%%%%%%%%%%%%%%%%%%%%%%%%%%%%%%%%%%

\documentclass[a4paper,11pt]{book}
\usepackage[T1]{fontenc}
\usepackage[utf8]{inputenc}
\usepackage{lmodern}
%%%%%%%%%%%%%%%%%%%%%%%%%%%%%%%%%%%%%%%%%%%%%%%%%%%%%%%%%
% Source: http://en.wikibooks.org/wiki/LaTeX/Hyperlinks %
%%%%%%%%%%%%%%%%%%%%%%%%%%%%%%%%%%%%%%%%%%%%%%%%%%%%%%%%%
\usepackage{/Users/HechenHu/Development/NoteTaking/Mathematics-Notes/Customized}
\graphicspath{ {/Users/HechenHu/Development/NoteTaking/Mathematics-Notes/Point-Set\_Topology} }
%%%%%%%%%%%%%%%%%%%%%%%%%%%%%%%%%%%%%%%%%%%%%%%%
% Chapter quote at the start of chapter        %
% Source: http://tex.stackexchange.com/a/53380 %
%%%%%%%%%%%%%%%%%%%%%%%%%%%%%%%%%%%%%%%%%%%%%%%%

\makeatletter

\renewcommand{\@chapapp}{}% Not necessary...

\newenvironment{chapquote}[2][2em]
{\setlength{\@tempdima}{#1}%
	\def\chapquote@author{#2}%
	\parshape 1 \@tempdima \dimexpr\textwidth-2\@tempdima\relax%
	\itshape}
{\par\normalfont\hfill--\ \chapquote@author\hspace*{\@tempdima}\par\bigskip}
\makeatother

%%%%%%%%%%%%%%%%%%%%%%%%%%%%%%%%%%%%%%%%%%%%%%%%%%%
% First page of book which contains 'stuff' like: %
%  - Book title, subtitle                         %
%  - Book author name                             %
%%%%%%%%%%%%%%%%%%%%%%%%%%%%%%%%%%%%%%%%%%%%%%%%%%%

% Book's title and subtitle
\title{\Huge \textbf{Point-Set Topology}}
% Author
\author{\textsc{Hechen Hu}}

\begin{document}
	\frontmatter
	\maketitle
	%%%%%%%%%%%%%%%%%%%%%%%%%%%%%%%%%%%%%%%%%%%%%%%%%%%%%%%%%%%%%%%%%%%%%%%%
	% Auto-generated table of contents, list of figures and list of tables %
	%%%%%%%%%%%%%%%%%%%%%%%%%%%%%%%%%%%%%%%%%%%%%%%%%%%%%%%%%%%%%%%%%%%%%%%%
	
	\tableofcontents
	\mainmatter
	
	\chapter{Basic Set Theory}
	\section{Classification of Relations}
	\begin{Definition}
		An \textit{equivalence relation} is a relation that satisfy the following properties: \\
		\newline
		$ aRa $ (Reflexivity);\\
		$ aRb \Rightarrow bRa $ (Symmetry); \\
		$ (aRb)\land (bRc) \Rightarrow aRc $ (Transitivity).\\
		\newline
		An equivalence relation is denoted by the special symbol $ \sim $. $ a\sim b $ means $ a $ is \textit{equivalent} to $ b $.	
	\end{Definition}
	\begin{Definition}
		Let $ R(\sim) $ be an equivalence relation on $ A $. If $ a \in A $, the \textit{equivalence class} of $ a $ (denoted $ \bar{a} $) is the class of all those elements of $ A $ that are equivalent to $ a $. The class of all equivalence classes in $ A $ is denoted $ A/R $ and called the \textit{quotient class} of $ A $ by $ R $.
	\end{Definition}
	\begin{Theorem}
		Two equivalence classes are either disjoint or equal.
	\end{Theorem}
	
	\begin{Definition}
		A \textit{partial ordering} on a set $ X^2 $ is a relation $ R $ that have the following properties: \\
		\newline
		$ aRa $ (Reflexivity);\\
		$ (aRb)\land (bRc) \Rightarrow aRc $ (Transitivity).\\
		$ (aRb) \land (bRa )\Rightarrow (a=b) $ (Anti-symmetry);	
	\end{Definition}
	
	We often write $ a\preceq b $ and say that $ b $ \textit{follows} $ a $.
	If the condition
	\begin{equation}
	\forall a \forall b((aRb)\lor(bRa) \nonumber
	\end{equation}
	holds in addition to transitivity and anti-symmetry defining a partial ordering relation(this means any two elements of $ X $ is comparable), the relation $ R $ is called an \textit{ordering}, and the set $ X $ is said to be \textit{linearly ordered}.
	\begin{Definition}
		A relation $ \prec $ is called a \textit{strict partial order} if it's nonreflexive and transitive.
	\end{Definition}
	\begin{Theorem}[The Maximum Principle]
		Let $ A $ be a set and $ \prec $ be a strict partial order on $ A $. Then there exists a maximal simply ordered(linearly ordered) subset of $ A $.
	\end{Theorem}
	\begin{Lemma}[Zorn's Lemma]
		Let $ A $ be a set that is strictly partially ordered. If every simply ordered subset of $ A $ has an upper bound in $ A $, then $ A $ has a maximal element.
	\end{Lemma}
	\begin{Definition}
		If $ X $ is a set and $ < $ is an order relation on $ X $, and if $ a<b $, the notation $ (a,b) $ to denote the set
		\begin{equation}
		\{x|a<x<b \} \nonumber
		\end{equation}
		it is called an \textit{open interval} in $ X $. If this set is empty, $ a $ is called the \textit{immediate predecessor} of $ b $, and $ b $ is called the \textit{immediate successor} of $ a $.
	\end{Definition}
	\begin{Definition}
		Suppose that $ A $ and $ B $ are two sets with order relations $ <_A $ and $ <_B $ respectively. We say that $ A $ and $ B $ have the same \textit{order type} if there is a bijective correspondence between them that preserves order, that is, if $ f:A \to B $ is a bijection
		\begin{equation}
		a_1 <_A a_2 \Rightarrow f(a_1)<_B f(a_2) \nonumber
		\end{equation}
	\end{Definition}
	\begin{Example}
		The interval $ (1,1) $ of real numbers has the same order type as $ \Real $, for the function $ f:(-1,1)\to \Real $ given by
		\begin{equation}
		f(x)=\frac{x}{1-x^2} \nonumber
		\end{equation}
		is an order-preserving bijection.
	\end{Example}
	\begin{Definition}
		Suppose $ A $ and $ B $ are two sets with order relations $ <_A $ and $ <_B $ respectively. Define an order relation $ < $ on $ A \times B $ by defining
		\begin{equation}
		a_1 \times b_1 < a_2 \times b_2
		\end{equation}
		if $ a_1 <_A a_2 $, or if $ a_1 = a_2 $ and $ b_1 <_B b_2 $. It is called the \textit{dictionary order relation} on $ A \times B $.
	\end{Definition}
	
	\subsubsection{Functions and Their Graphs}
	\begin{Definition}
		A relation $ R $ is said to be functional if
		\begin{equation}
		(xRy_1)\land(xRy_2)\Rightarrow (y_1=y_2) \nonumber
		\end{equation}
		and it is called a \textit{function}.\\
		$ R\subset X \times Y $ is a \textit{mapping from $ X $ into $ Y $}, or a \textit{function from $ X $ into $ Y $}.\\
	\end{Definition}
	\section{Sets and Operations on Them}
	
	\subsection{Naive Set Theory}
	1$^0$.\quad A set may consist of any distinguishable objects($x\in A\Rightarrow\exists!x\in A$) \\
	2$^0$.\quad A set is unambiguously determined by the collection of objects that comprise it. \\
	3$^0$.\quad Any property defines the set of objects having that property($A = \{x|P(x)\}\Rightarrow P(A)$). \\ \\
	However, this will lead to Russell's Paradox: \\
	Let's have$P(M) \coloneqq M\notin M$ \\
	Consider the class $K = \{M|P(M)\}$. If so $K$ is not a set, since whether $P(K)$ is true or false, contradiction arises.\\
	
	\subsection{ZFC: Zermelo-Fraenkel Axioms and Axiom of Choice}
	1$^0$.\quad \textbf{(Axiom of Extensionality)} Sets $ A $ and $B$ are equal iff they have the same elements. $ (A=B)\Leftrightarrow(\forall x((x\in A)\Leftrightarrow(x \in B))) $\\
	2$^0$. \quad \textbf{(Axiom of Seperation)} To any set $A$ and any property P there corresponds a set $B$ whose elements are those elements of $A$, and only those, having property P(if $A$ is a set, then $B=\{x\in A|P(x)\}$ is also a set). \\
	3$^0$. \quad \textbf{(Union Axiom)} For any set $ \mathscr{M} $ whose elements are sets there exists a set $\bigcup M$, called the union of $ M $ and consisting of those elements and only those that belong to some element of $ \mathscr{M} $ $ (x\in \bigcup \mathscr{M} \Leftrightarrow \exists X((X\in \mathscr{M})\land (x\in X))) $ \\
	Similarly, the intersection of the set $ \mathscr{M} $ is defined as:
	\begin{equation}
	\bigcap \mathscr{M}  \coloneqq \{x\in \bigcup \mathscr{M} | \forall X((X\in \mathscr{M})\Rightarrow (x\in X))\} \nonumber
	\end{equation}
	4$^0$ \quad \textbf{(Pairing Axiom)} For any sets $ X $ and $ Y $ there exists a set $ Z $ such that $ X $ and $ Y $ are its only elements. \\
	5$^0$ \quad \textbf{(Power Set Axiom)} For any set $ X $ there exists a set $ P(X)$ having each subset of $ X $ as an element, and having no other elements. \\
	\newline
	\begin{Definition}
		The \textit{successor} $X^+$ of the set $ X $ is $X^+ = X\cup \{X\}$.\\
	\end{Definition} 
	\begin{Definition}
		An \textit{inductive} set is a set that $\varnothing$ is one of its elements and the successor of each of its elements aso belongs to it.
	\end{Definition}
	6$^0$ \quad \textbf{(Axiom of Infinity)} There exist inductive sets (Example: $\mathbb{N}_0$).\\ 
	7$^0$ \quad \textbf{(Axiom of Replacement)} Let $ F(x,y) $ be a statement(a formula) such that for every $ x_0 \in X$ there exists a unique object $ y_0 $ such that $ F(x_0,y_0) $ is true. Then the objects $ y $ for which there exists an element $ x\in X $ such that $ F(x,y) $ is true form a set.\\
	And finally, an axiom that is independent of ZF. \\
	\newline
	8$^0$ \quad \textbf{(Axiom of Choice/Zermelo's Axiom)}  Given a collection of disjoint nonempty sets, there exists another set consisting of exactly one element from each element of the original set.
	\begin{Definition}
		A choice function is a function $ f $, defined on a collection $ X $ of nonempty sets, such that for every set $ A $ in $ X $, $ f(A) $ is an element of $ A $.
	\end{Definition}
\begin{Corollary}
	There exists a choice function for any collection of nonempty sets.
\end{Corollary}

	
	
	\subsection{The \textit{Cardinality} of a Set(\textit{Cardinal Numbers})}
	\begin{Definition}
		The set $ X $ is said to be \textit{equipollent} to the set $ Y $ if there exists a bijective mapping of $ X $ onto $ Y $(then $ X\sim Y $).
	\end{Definition}
	\begin{Definition}
		\textit{Cardinality} is a measure of the number of elements of the set. If $ X \sim Y $, we write $ \card X = \card Y $.
	\end{Definition}
	If $ X $ is equipollent to some subset of $ Y $, we say $ \card X\leqslant \card Y $, thus
	\begin{equation}
	(\card X \leqslant \card Y)\coloneqq \exists Z\subset Y (\card X = \card Z) \nonumber
	\end{equation}
	A set is called \textit{finite} if it is not equipollent to any proper subset of itself; otherwise it is called \textit{infinite}. \\
	It has the  properties below:\\
	\newline
	1$^0  \quad (cardX\leqslant \card Y)\land(\card Y \leqslant \card Z)\Rightarrow(\card X \leqslant \card Z)$.\\
	2$^0 \quad (\card X \leqslant \card Y)\land(\card Y \leqslant \card X)\Rightarrow(\card X = \card Y)$(The Schröder–Bernstein theorem).\\
	3$^0 \quad \forall X\forall Y(\card X \leqslant \card Y)\lor(\card Y \leqslant \card X)$(Cantor's theorem).\\
	\newline
	We say $ \card X < \card Y $ if $ (\card X \leqslant \card Y) \land (\card X \neq \card Y)$.\\
	let $ \varnothing $ be the empty set and $ P(X) $ the set of all subsets(thus, the power set) of the set $ X $. Then:
	\begin{Theorem}
		$ \card X < \card P(X) $
	\end{Theorem}
	\begin{proof}
		The assertion is obvious for the empty set, and we shall assume that $ X\neq \varnothing $.\\
		Since $ P(X) $ contains all the one-element subsets of $ X $, $ \card X \leqslant \card P(X)$.\\
		Suppose, contrary to the assertion, that there exists a bijective mapping $ f : X \to P(X) $. Let set $ A = \{x\in X:x\notin f(x)\}$ consisting of the elements $ x \in X $ that do not belong to the set $ f(x)\in P(X) $ assigned to them by the bijection. Because $ A \in P(X) $, there exists $ a\in X $ such that $ f(a) = A $. For the element $ a $ the relation $ a \in A $ or $ a \notin A $ is impossible by the definition of $ A $(Similar to Russell's Paradox).
	\end{proof}
	
	
	
	\subsection{Operations on Sets}
	\begin{tabular}{| l | l | l |}
		\hline
		Notation &Meaning &Definition \\ \hline
		$A\subset B$ &$A$ is a subset of $B$ &$\forall x ((x\in A) \Rightarrow(x\in B))$ \\ \hline
		$A\subsetneq B$ &$A$ is a proper subset of $B$ &$A \neq B \land A \subset B$ \\ \hline
		$A = B$ &$A$ equals to $B$ &$(A \subset B)\land(B \subset A)$ \\ \hline
		$\varnothing$ &Empty Set & $\{x|x\neq x\}$ \\ \hline
		$A \cup B$ &The union of $ A $ and $ B $ &$ \{x|x\in A \lor x\in B\} $ \\
		$ A \cap B $ &The intersection of $ A $ and $ B $ &$ \{x|x\in A \land x\in B\} $ \\ \hline
		$ A\setminus B $ &The difference between $ A $ and $ B $ &$ \{x|x\in A \land x\notin B\} $ \\ \hline
		$ C_M A $ &The complement of $A$ in M &$ \{x|x\in M \land x\notin A\} where A\subset M $ \\ \hline
		$ A \times B $ &The Cartesian Product of $ A $ and $ B $ &$ \{(x,y)|x\in A \land y\in B\} $ \\ \hline
		$ A^2 $ &$ A \times A $ & \\
		\hline
	\end{tabular}
	\newline \\
	In the ordered pair$ z = (x_1,x_2) $ where$ Z = X_1 \times X_2 ,z\in Z,x_1 \in X_1, x_2 \in X_2$, $ x_1 $ is called the \textit{first projection} of the pair $ z $ and denoted  $\proj_1 z $ while $ x_2 $ is called the \textit{second projection} of the pair $ z $ and denoted  $\proj_2 z $.
	
	
	
	
	
	
	
	\section{Countable and Uncountable Sets}
	
	\begin{Definition}
		A set $ X $ is \textit{countable} if it is equipollent with the set $ \Natural $ of natural numbers, that is, $ \card X = \card \Natural $.
	\end{Definition}
	
	\begin{Proposition}
		An infinite subset of a countable set is countable.
	\end{Proposition}
	\begin{proof}
		Let's consider a countable set $ E $. There is a minimal element of $ E_1\coloneqq E $, which we assign to $ 1 \in \Natural $ and denote $ e_1 \in E $. $ E $ is infinite, so $ E_2 \coloneqq E \setminus e_1 $ is not empty. Following the principle of induction, we can construct a injective mapping from $ \{1,2...\} $ to $ \{e_1,e_2,...\} $. \newpara
		Now we have to prove that this mapping is also surjective. Suppose the contrary, that an element $ e \in E$ does not have a natural number assigned to it. The set $ K=\{n \in E | n \leqslant e\} $ is finite, since it's a subset of $ \Natural $ bounded both from below and above. According to our previous construction, we assign $ 1 $ to $ \min K $, denoted as $ e_1 $, and we can acquire a sequence $ e_1, e_2,...e_{k=\card K} $. But $ e_{k=\card K} $ is $ \max K $, and because $ e\in K \land (\forall n\in K (n\leqslant e))$, $ e = \max K$. Therefore $ e = e_k $, or otherwise it will contradict the uniqueness of maximal element.
	\end{proof}
	
	\begin{Proposition}
		The Union of the sets of a finite or countable system of countable sets is also a countable set.
	\end{Proposition}
	\begin{proof}
		Let $ X_1,X_2...,X_n,... $ is a countable system of sets and each set $ X_m = \{x^1_m,...,x^n_m,...\} $ is itself countable. Since $ \forall m\in \Natural (\card (X=\bigcup_{n \in \Natural}X_n) \geqslant X_m)$, $ X $ is an infinite set. The ordered pair $ (m,n) $ identifies the element $ x^n_m \in X_m $. We can construct a mapping, like $ f : \Natural \times \Natural \to \Natural \coloneqq (m,n) \to \frac{(m+n-2)(m+n-1)}{2}+m $, such that it is bijective. Thus $ X $ is countable. Then because $ \card X \leqslant \card \Natural $ and the fact that $ X $ is infinite, we conclude that $ \card X = \card \Natural $.
	\end{proof}
	
	If it is known that a set is either finite or countable, we say it is \textit{at most countable}($ \card X \leqslant \Natural $).
	
	\begin{Corollary}
		$ \card \Zahlen = \card \Natural $
	\end{Corollary}
	
	\begin{Corollary}
		$ \card \Natural^2 = \card \Natural $(The direct product of countable sets is countable).
	\end{Corollary}
	
	\begin{Corollary}
		$ \card \Quoziente = \card \Natural $, that is, the set of rational numbers is countable.
	\end{Corollary}
	\begin{proof}
		Let $ (m,n) $ denote a rational number $ \frac{m}{n} $. It is known that the pair $ (m,n) $ and $ (m^\prime, n^\prime) $ define the same number iff they are proportional. Thus $ \Quoziente $ is equipollent to some infinite subset of the set $ \Zahlen \times \Zahlen $. Since $ \card \Zahlen^2 = \card \Natural $, we can conclude that $ \card \Quoziente = \card \Natural $.
	\end{proof}
	
	\begin{Corollary}
		The set of algebraic numbers is countable.
	\end{Corollary}
	\begin{proof}
		It can be observed that $ \card \Quoziente \times \Quoziente = \card \Natural  $. By the principle of induction, $ \forall k \in \Natural (\card \Quoziente^k = \card \Natural) $. Let $ r \in \Quoziente^k $ be an ordered set $ (r_1,r_2,...,r_k) $ consists of $ k $ rational numbers. \newpara
		An algebraic equation of degree $ k $ with rational coefficient can be writtne in the reduced form $ x^k + r_1x^{k-1}+ \cdots + r_k = 0 $. Thus there are as many different algebraic equations of degree $ k $ as there are different ordered sets $ (r_1,...,r_k) $ of rational numbers, that is, a countable set. \newpara
		The algebraic equation with rational coefficients (of arbitrary degree) is the union of sets consisting of algebraic equation (of a fixed degree) which is countable, and this union is countable. Each such equation has only a finite number of roots. Hence the set of algebraic numbers is at most countable. But it is infinite, and therefore countable.
	\end{proof}
	
	\subsection{The Cardinality of the Continuum}
	\begin{Definition}
		The set $ \Real $ of real numbers is also called the \textit{number continuum}(from Latin \textit{continuum}, meaning continuous, or solid), and its cardinality the \textit{cardinality of the continuum}.
	\end{Definition}
	
	\begin{Theorem}[Cantor]
		$ \card \Natural < \card \Real $
	\end{Theorem}
	\begin{proof}[Proof by Nested Interval Lemma]
		It is sufficient to show that even $ [0,1] $ in an uncountable set. \newpara
		Assume it is countable, that is, can be written as a sequence $ x_1,x_2,...,x_n,.... $. Take $ x_1 $ on $ I_0 = [0,1] $, and find $ I_1 $ such that $ x_1 \notin I_1 $. Then construct the nested interval $ I_n $ such that $ x_{n+1} \notin I_{n+1} $ and $ |I_n| > 0 $. It follows the nested interval lemma that there exist a point $ c \in [0,1]$ belonging to all $ I_n $. But by our construction, $ c \in \Real $ and $ c $ cannot be any point of the sequence $ x_1,x_2,...,x_n,.... $.
	\end{proof}
	
	\begin{proof}[Proof by Cantor's Diagonal Argument]
		Let's first consider an the set $ L $ and write out the infinite sequence of distinct binary numbers in it which has the form: 
		\begin{align}
		&s1 =	(0,	0,	0,	0,	0,	0,	0,	...) \\
		&s2 =	(1,	1,	1,	1,	1,	1,	1,	...) \\
		&s3 =	(0,	1,	0,	1,	0,	1,	0,	...)\\
		&s4 =	(1,	0,	1,	0,	1,	0,	1,	...)\\
		&s5 =	(1,	1,	0,	1,	0,	1,	1,	...)\\
		&s6 =	(0,	0,	1,	1,	0,	1,	1,	...)\\
		&s7 =	(1,	0,	0,	0,	1,	0,	0,	...)\\
		&... \\		
		\end{align}
		We then constrcut a number $ s $ such that its first digit is the complementary (swapping 0s for 1s and vice versa) of the first digit of $ s_1 $ and etc.
		\begin{align}
		&s1 =	(\mathbf{0},	0,	0,	0,	0,	0,	0,	...) \\
		&s2 =	(1,	\mathbf{1},	1,	1,	1,	1,	1,	...) \\
		&s3 =	(0,	1,	\textbf{0},	1,	0,	1,	0,	...)\\
		&s4 =	(1,	0,	1,	\textbf{0},	1,	0,	1,	...)\\
		&s5 =	(1,	1,	0,	1,	\textbf{0},	1,	1,	...)\\
		&s6 =	(0,	0,	1,	1,	0,	\textbf{1},	1,	...)\\
		&s7 =	(1,	0,	0,	0,	1,	0,	\textbf{0},	...)\\
		&... \\		
		&s = (\textbf{1},\textbf{0},\textbf{1},\textbf{1},\textbf{1},\textbf{0},\textbf{1},..)
		\end{align}
		By construction $ s $ differs from $ s_n $ at the $ n $th digit, so $ s $ is not in this sequence, and thus $ L $ is uncountable. \newpara
		We can now define a mapping $ f : L \to \Real $.$ f(s_n) = r_n\in \Real $ means that $ s_n $ and $ r_n $ have the same digit while $ r_n $ is under base 10 and $ s_n $ is under base 2. For $ s_n \neq s_m \Rightarrow (r_n=f(s_n)) \neq (r_m=f(s_m)) $, $ f $ is injective, and with the fact that all $ s_n $ corresponds to a $ r_n $ together give us $ \card f(L) = \card L $. Since $ f(L) $ is a subset of $ \Real $, we can see that $ \Real $ is also uncountable.
	\end{proof}
The proof above illustrates the theorem below.
\begin{Definition}
	Let $ X $ denote the two element set $ \{ 0,1\} $. Then $ X^\omega $ is uncountable.
\end{Definition}
	The cardinality of $ \Real $ is often denotes as $ \mathfrak{c} $.
	\begin{Corollary}
		$ \Quoziente \neq \Real $, and so irrational numbers exist.
	\end{Corollary}
	
	\begin{Corollary}
		There exist transcendental numbers, since the set of algebraic numbers is countable.
	\end{Corollary}
	
	\begin{Example}
		The cardinality of $ P(X) $, which is the power set of $ X $, satisfy that if $ \card X = n $, $ \card P(X) = 2^{n} $.
	\end{Example}
	\begin{proof}
		We can use the principle of induction to complete the proof. If $ n= 1 $, $ X = \{x\} $, then $ P(X) =\{\varnothing, X\} $, then $ \card P(X) = 2^{1} $. \newpara
		Now if $ n \in \Natural \Rightarrow \card P(X) = 2^{n}  $, let $ X $ be a set that has $ x $ as one of its elements and has the cardinality of $ n+1 $. Therefore $ Y = X \setminus \{x\} $ has $ n $ elements. We can divide $ P(X) $ into two parts: the ones containing $ x $ and the ones don't. If $ x\in A \subset P(X) $, then $ A \setminus \{x\} \subset P(Y) $ and vice versa. Thus we can set up a bijection between $ P(Y) $ and the elements in $ P(X) $ that contains $ x $. Similarly, we can clearly see that a bijection between the subsets of $ P(X) $ that does not contains $ x $ and $ P(Y) $. Thus $ \card P(X) = 2^n + 2^n = 2^{n+1} $, and we complete the proof.
	\end{proof}
	We'll use a script letter to denote the collection of sets, for example, $ \mathscr{A} $ for collection of sets and $ A $ for individual sets in it.
	\begin{Definition}
		A \textit{partition} of a set $ A $, besides the definition we have when studying Riemann Sum, can be defined as a collection of disjoint nonempty subsets of $ A $ whose union is all of $ A $.
	\end{Definition}
\begin{Theorem}
	Given any partition $ \mathscr{D} $ of $ A $, there is exactly one equivalence relation on $ A $ from which it is derived.
\end{Theorem}
\begin{Example}
	Defined two points in the plane to be equivalent if they lie at the same distance from the origin. The collection of equivalence classes consists of all circles centered at the origin, along with the set consisting of the origin alone.
\end{Example}
\begin{Definition}
	Any set together with a order relation $ < $ that satisfy both the following properties
	\begin{enumerate}
		\item $ < $ has the least upper bound property.
		\item if $ x<y $, then there exists an element $ z $ such that $ x<z<y $.
	\end{enumerate}
is called a \textit{linear continuum}.
\end{Definition}
\begin{Definition}
	Let $ \mathscr{A} $ be a nonempty collection of sets. An \textit{indexing function} for $ \mathscr{A} $ is a surjective function $ f $ from some set $ J $, called the \textit{index set}, to $ \mathscr{A} $. The collection $ \mathscr{A} $, together with the indexing function is called an \textit{indexed family of sets}. Given $ \alpha \in J $, the set $ f(\alpha) $ is denoted $ A_\alpha $. The indexed family itself is denoted by
	\begin{equation}
		\{A_\alpha \}_{\alpha \in J} \nonumber
	\end{equation}
\end{Definition}
\begin{Theorem}[Principle of Recursive Definition]
	Let $ A $ be a set. Given a formula that defined $ h(1) $ as a unique element of $ A $, and for $ i>1 $ defines $ h(i) $ uniquely as an element of $ A $ in terms of the values of $ h $ for positive integers less than $ i $, this formula determines a unique function $ h:\Natural \to A $.
\end{Theorem}
\begin{Definition}
	A set $ A $ with an order relation $ < $ is said to be \textit{well-ordered} if every nonempty subset of it has a smallest element.
\end{Definition}
\begin{Theorem}
	Any subset of a well-ordered set is well-ordered. The cartesian product of two well-ordered sets is well-ordered.
\end{Theorem}
\begin{Theorem}
	Every nonempty finite ordered set has the order type of a section of $ \Natural $, so it's well-ordered.
\end{Theorem}
\begin{Theorem}[Well-ordering theorem, proved by Zermelo]
	If $ A $ is a set, there exists an order relation on $ A $ that is a well-ordering.
\end{Theorem}
\begin{Corollary}
	There exists an uncountable well-ordered set.
\end{Corollary}
\begin{Definition}
	Let $ X $ be a well-ordered set. Given $ \alpha \in X $, let $ S_\alpha $ denote the set
	\begin{equation}
		S_\alpha = \{x|x \in X \land x< \alpha \} \nonumber
	\end{equation}
	It is called the \textit{section} of $ X $ by $ \alpha $.
\end{Definition}
\begin{Lemma}
	There exists a well-ordered set $ A $ having a largest element $ \Omega $, such that the section $ S_\Omega $ of $ A $ by $ \Omega $ is uncountable but every other section of $ A $ is countable.
\end{Lemma}
\begin{proof}
	We begin with an uncountable well-ordered set $ B $. Let $ C $ be the well-ordered set $ \{1,2 \}\times B $ in the dictionary order, then some section of $ C $ is uncountable. Let $ \Omega $ be the smallest element of $ C $ for which the section of $ C $ by $ \Omega $ is uncountable, then let $ A $ consist of this section along with $ \Omega $.
\end{proof}
The set $ S_\Omega $ is called a \textit{minimal uncountable well-ordered set}, and the well-ordered set $ A=S_\Omega \cup \{\Omega \} $ by $ \bar{S}_\Omega $.
\begin{Theorem}
	If $ A $ is a countable subset of $ S_\Omega $, then $ A $ has an upper bound in $ S_\Omega $.
\end{Theorem}	
\begin{proof}
	Let $ A $ be a countable subset of $ S_\Omega$. For each $ a \in A $, the section $ S_a $ is countable. Therefore, the union $ B=\bigcup_{\alpha \in A}S_a $ is also countable. Since $ S_\Omega \neq B $, let $ x $ be a point of $ S_\Omega $ that is not in $ B $, and then $ x $ is an upper bound for $ A $.
\end{proof}
	\chapter{Topological Spaces and Continuous Functions}
	\section{Definition for Topological Spaces}
	\begin{Definition}
		A \textit{topology} on a set $ X $ is a collection $ \mathscr{T} $ of subsets of $ X $ having the following properties:
		\begin{enumerate}
			\item $ \varnothing $ and $ X $ are in $ \mathscr{T} $.
			\item The union of the elements of any subcollection of $ \mathscr{T} $ is in $ \mathscr{T} $.
			\item The intersection of the elements of any finite subcollection of $ \mathscr{T} $ is in $ \mathscr{T} $.
		\end{enumerate}
	A set $ X $ for which a topology $ \mathscr{T} $ has been specified is called a \textit{topological space}. A subset $ U $ of $ X $ is an \textit{open set} of $ X $ if $ U $ belongs to the collection $ \mathscr{T} $. Then a topological space is a set $ X $ with a collection of subsets of $ X $, called open sets, such that $ X $ and $ \varnothing $ are both open and arbitrary unions and finite intersections of open sets are open.
	\end{Definition}
\begin{Definition}
	If $ X $ is any set, the collection of all subsets of $ X $ is a topology on $ X $ and called the \textit{discrete topology}. The collection consisting of $ X $ and $ \varnothing $ is also a topology on $ X $ and is called the \textit{indiscrete topology} or the \textit{trivial topology}.
\end{Definition}
\begin{Definition}
	Let $ X $ be a set; let $ \mathscr{T}_f $ be the collection of all subsets $ U $ of $ X $ such that $ X \setminus U $ either is finite or is all of $ X $. Then $ \mathscr{T}_f $ is a topology on $ X $ and called the \textit{finite complement topology}.
\end{Definition}
\begin{Definition}
	Suppose $ \mathscr{T} $ and $ \mathscr{T}^\prime $ are two topologies on a given set $ X $. If $ \mathscr{T}^\prime \supset \mathscr{T} $, we say that $ \mathscr{T}^\prime $ is \textit{finer} than $ \mathscr{T} $; If $ \mathscr{T}^\prime $ properly contains $ \mathscr{T} $, we say that $ \mathscr{T}^\prime $ is \textit{strictly finer} then $ \mathscr{T} $. We also say that $ \mathscr{T} $ is \textit{coarser} than $ \mathscr{T}^\prime $, or \textit{strictly coarser}, in these two respective situations. We say $ \mathscr{T} $ is \textit{comparable} with $ \mathscr{T}^\prime $ if either $ \mathscr{T}^\prime \supset \mathscr{T}  $ or $ \mathscr{T} \supset \mathscr{T}^\prime  $
\end{Definition}
\section{Basis for Topology}
	\begin{Definition}
		If $ X $ is a set, a \textit{basis} for a topology on $ X $ is a collection $ \mathscr{B} $ of subsets of $ X $(called \textit{basis elements}) such that
		\begin{enumerate}
			\item For each $ x \in X $, there is at least one basis element $ B $ containing $ x $.
			\item If $ x $ belongs to the intersection of two basis elements $ B_1 $ and $ B_2 $, then there is a basis element $ B_3 $ containing $ x $ such that $ B_3 \subset B_1 \cap B_2$.
		\end{enumerate}
	If $ \mathscr{B} $ satisfies these two conditions, then we define the \textit{topology $ \mathscr{T} $ generated by $ \mathscr{B} $} as follows: A subset $ U $ of $ X $ is said to be open in $ X $ if for each $ x\in U $, there is a basis element $ B \in \mathscr{B} $ and $ x \in B $ and $ B \subset U $.
	\end{Definition}
\begin{Example}
	If $ X $ is any set, the collection of all one-point subsets of $ X $ is a basis for the discrete topology on $ X $.
\end{Example}
\begin{Lemma}
	Let $ X $ be a set; Let $ \mathscr{B} $ be a basis for a topology $ \mathscr{T} $ on $ X $. Then $ \mathscr{T} $ equals toe collection of all unions of elements of $ \mathscr{B} $.
\end{Lemma}
	\begin{proof}
		Given a collection of elements of $ \mathscr{B} $, they are also elements of $ \mathscr{T} $. Because $ \mathscr{T} $ is a topology, their union is in $ \mathscr{T} $. Conversely, given $ U \in \mathscr{T} $, choose for each $ x\in U $ an element $ B_x $ of $ \mathscr{B} $ such that $ x \in B_x \subset U $. Then $ U = \bigcup_{x\in U}B_x $, so $ U $ equals a union of elements of $ \mathscr{B} $.
	\end{proof}
\begin{Lemma}
	Let $ X $ be a topological space. Suppose that $ \mathscr{C} $ is a collection of open sets of $ X $ such that for each open set $ U $ of $ X $ and each $ x $ in $ U $, there is an element $ C $ of $ \mathscr{C} $ such that $ x \in C \subset U $. Then $ \mathscr{C} $ is a basis for the topology of $ X $.
\end{Lemma}	
\begin{proof}
	First we show that $ \mathscr{C} $ is a basis. Given $ x \in X $, since $ X $ is open, there is by hypothesis an element $ C $ of $ \mathscr{C} $ such that $ x\in C \subset X $. Now let $ x \in C_1 \cap C_2 $, where $ C_1 $ and $ C_2 $ are elements of $ \mathscr{C} $. The intersection of them is open, and there exists by hypothesis an element $ C_3 $ in $ \mathscr{C} $ such that $ x \in C_3 \subset C_1 \cap C_2 $. \newpara
	Let $ \mathscr{T} $ be the collection of open sets of $ X $; we will show that the topology $ \mathscr{T}^\prime $ generated by $ \mathscr{C} $ equals the topology. First, note that if $ U $ belongs to $ \mathscr{T} $ and if $ x \in U $, then there is by hypothesis an element $ C $ of $ \mathscr{C} $ such that $ x \in C \subset U $. It follows that $ U $ belongs to the topology $ \mathscr{T}^\prime $ by definition. Conversely, if $ W $ belongs to the topology $ \mathscr{T} $, then $ W $ equals a union of elements of $ \mathscr{C} $ by the preceding lemma. Since each element of $ \mathscr{C} $ belongs to $ \mathscr{T} $ and $ \mathscr{T} $ is a topology, $ W $ also belongs to $ \mathscr{T} $.
\end{proof}
\begin{Lemma}
	Let $ \mathscr{B} $ and $ \mathscr{B}^\prime $ be bases for the topologies $ \mathscr{T} $ and $ \mathscr{T}^\prime $, respectively, on $ X $. Then the following are equivalent:
	\begin{enumerate}
		\item $ \mathscr{T}^\prime $ is finer than $ \mathscr{T} $.
		\item For each $ x \in X $ and each basis element $ B\in \mathscr{B} $ containing $ x $, there is a basis element $ B^\prime \in \mathscr{B}^\prime $ such that $ x \in B^\prime \subset B $.
	\end{enumerate}
\end{Lemma}
\begin{proof}
	First, we prove that the second condition implies the first one. Given an element $ U $ of $ \mathscr{T} $, we wish to show that $ U \in \mathscr{T}^\prime $. Let $ x \in U $. Since $ \mathscr{B} $ generates $ \mathscr{T} $, there is a element $ B \in \mathscr{B} $ such that $ x \in B \subset U $. Condition $ (2) $ tells us there exists an element $ B^\prime \in \mathscr{B}^\prime $ such that $ x \in B^\prime \subset B $. Then $ x \in B^\prime \subset U $, so $ U \in \mathscr{T}^\prime $ by definition.
	\newpara
	Then we prove that the first condition implies the second. We are given $ x \in X $ and $ B \in \mathscr{B} $. Now $ B $ belongs to $ \mathscr{T} $ by definition and $ \mathscr{T}\subset \mathscr{T}^\prime $ by condition $ (1) $; therefore, $ B \in \mathscr{T}^\prime $. Since $ \mathscr{T}^\prime $ is generated by $ \mathscr{B}^\prime $, there is an element $ B^\prime \in \mathscr{B}^\prime $ such that $ x \in B^\prime \subset B $.
\end{proof}
\begin{Definition}
	If $ \mathscr{B} $ is the collection of all intervals in the real line, the topology generated by $ \mathscr{B} $ is called the \textit{standard topology} on the real line. If $ \mathscr{B}^\prime $ is the collection of all half-open intervals $ [a,b) $ where $ a<b $, the topology generated by $ \mathscr{B}^\prime $ is called the \textit{lower limit topology} on $ \Real $. When $ \Real $ is given the lower limit topology, it's denoted $ \Real_{l} $. Let $ K $ denote the set of all numbers of the form $ 1/n $ for $ n\in \Natural $, and let $ \mathscr{B}^{\prime\prime} $ be the collection of all open intervals $ (a,b) $, along with all sets of the form $ (a,b)\setminus K $. The topology generated by $ \mathscr{B}^{\prime\prime} $ will be called the \textit{K-topology} on $ \Real $. When $ \Real $ is given this topology, it's denoted by $ \Real_{K} $.
\end{Definition}
\begin{Lemma}
	The topologies of $ \Real_l $ and $ \Real_K $ are strictly finer than the standard topology on $ \Real $, but are not comparable with one another.
\end{Lemma}
\begin{proof}
	Let $ \mathscr{T} $, $ \mathscr{T}^\prime $, and $ \mathscr{T}^{\prime\prime} $ be the topologies of $ \Real $, $ \Real_l $, and $ \Real_K $ respectively. Given a basis element $ (a,b) $ for $ \mathscr{T} $ and a point $ x $ of $ (a,b) $, the basis element $ [x,b) $ for $ \mathscr{T}^\prime $ contains $ x $ and lies in $ (a,b) $. On the other hand, given the basis element $ [x,d) $ for $ \mathscr{T}^\prime $, there is no open interval $ (a,b) $ that contains $ x $ and lies in $ [x,d) $, and thus $ \mathscr{T}^\prime $ is strictly finer than $ \mathscr{T} $. \newpara
	A similar argument applies to $ \Real_K $. Given a basis element $ (a,b) $ for $ \mathscr{T} $ and a point $ x\in(a,b) $, this same interval is a basis for $ \mathscr{T}^{\prime\prime} $ that contains $ x $. On the other hand, given the basis element $ B=(-1,1)\setminus K $ for $ \mathscr{T}^{\prime\prime} $ and the point $ 0 $ of $ B $, there is no open interval that contains $ 0 $ and lies in $ B $. \newpara
	Now we show that $ \Real_l $ and $ \Real_K $ are not comparable. For any basis element in $ \Real_l  $ that has $ 0 $ as its lower limit, it always contains number of the form $ 1/n $, thus not any subset of sets of the form $ (a,b)\setminus K $. The rest of this argument is trivial.
\end{proof}
\begin{Definition}
	A \textit{subbasis} $ \mathscr{S} $ for a topology on $ X $ is a collection of subsets of $ X $ whose union equals $ X $. The \textit{topology generated by the subbasis} $ \mathscr{S} $ is defined to be the collection $ \mathscr{T} $ of all unions of finite intersections of elements of $ \mathscr{S} $.
\end{Definition}	
For the purpose of checking whether $ \mathscr{T} $ is a topology, it's sufficient to show that the collection $ \mathscr{B} $ of all finite intersections of elements of $ \mathscr{S} $ is a basis, for then the collection $ \mathscr{T} $ of all unions of elements of $ \mathscr{B} $ is a topology. Given $ x \in X $, it belongs to an element of $ \mathscr{S} $ and hence to an element of $ \mathscr{B} $; to check the second condition, let
\begin{equation}
	B_1 = S_1 \cap \cdots \cap S_m\quad \text{and}\quad B_2=S^\prime_1 \cap \cdots \cap S^\prime_n \nonumber
\end{equation}
to be two elements of $ \mathscr{B} $. Their intersection
\begin{equation}
	B_1 \cap B_2=(S_1 \cap \cdots \cap S_m)\cap (S^\prime_1 \cap \cdots \cap S^\prime_n) \nonumber
\end{equation}
is also a finite intersection of elements of $ \mathscr{S} $, so it belongs to $ \mathscr{B} $.
	
\section{The Order Topology}
\begin{Definition}
	Let $ X $ be a set with a linear order relation; assume $ X $ has, more than one element. Let $ \mathscr{B} $ be the collection of all sets of the following types:
	\begin{enumerate}
		\item All open intervals $ (a,b) $ in $ X $.
		\item All interval of the form $ [a_0,b) $, where $ a_0 $ is the smallest element of $ X $.
		\item All intervals of the form $ (a,b_0] $, where $ b_0 $ is the largest element of $ X $.
	\end{enumerate}
The collection $ \mathscr{B} $ is a basis for a topology on $ X $, which is called the \textit{order topology}.
\end{Definition}	
\begin{Definition}
	If $ X $ is an ordered set, and $ a $ is an element of $ X $, there are four subsets of $ X $ that are called the rays determined by $ a $. The \textit{open rays} are $ (a,+\infty) $ and $ (-\infty,a) $, the \textit{closed rays} are $ [a,+\infty) $ and $ (-\infty,a] $.
\end{Definition}
\section{The Product Topology on $ X \times Y $}
\begin{Definition}
	Let $ X $ and $ Y $ be topological spaces. The \textit{product topology} on $ X \times Y $ is the topology having as basis the collection $ \mathscr{B} $ of all sets of the form $ U \times V $, where $ U $ and $ V $ are open sets of $ X $ and $ Y $ respectively.
\end{Definition}
\begin{Theorem}
	If $ \mathscr{B} $ is a basis for the topology of $ X $ and $ \mathscr{C} $ is a basis for the topology of $ Y $, then the collection
	\begin{equation}
		\mathscr{D}=\{B \times C|B\in\mathscr{B}\land C \in \mathscr{C}  \} \nonumber
	\end{equation}
	is a basis for the topology of $ X \times Y $.
\end{Theorem}
\begin{proof}
	Given an open set $ W $ of $ X \times Y $ and a point $ x \times y $ of $ W $, by definition of the product topology there is a basis element $ U \times V $ such that $ x \times y \in U \times V \subset W $. Then $ \mathscr{D} $ is a basis for $ X \times Y $.
\end{proof}
\begin{Theorem}
	The collection
	\begin{equation}
		\mathscr{S}=\{\proj_1^{-1}{U}|U \text{ open in }X \}\cup\{ \proj_2^{-1}{V}|V \text{ open in }Y\} \nonumber
	\end{equation}
	is a subbasis for the product topology on $ X \times Y $.
\end{Theorem}
\begin{proof}
	Let $ \mathscr{T} $ denote the product topology on $ X \times Y $; Let $ \mathscr{T}^\prime $ be the topology generated by $ \mathscr{S} $. Because every element of $ \mathscr{S} $ belongs to $ \mathscr{T} $, so do arbitrary unions of finite intersections of elements of $ \mathscr{S} $. Thus $ \mathscr{T}^\prime \subset \mathscr{T} $. On the other hand, every basis element $ U \times V $ for the topology $ \mathscr{T} $ is a finite intersection of elements in $ \mathscr{S} $, since
	\begin{equation}
		U \times V = \proj_1^{-1}(U)\cap\proj_2^{-1}(V) \nonumber
	\end{equation}
	Therefore, $ U \times V $ belongs to $ \mathscr{T}^\prime $, so that $ \mathscr{T}\subset \mathscr{T}^\prime $.
\end{proof}
\section{The Subspace Topology}
\begin{Definition}
	Let $ X $ be a topological space with topology $ \mathscr{T} $. If $ Y $ is a subset of $ X $, the collection
	\begin{equation}
		\mathscr{T}_Y = \{Y \cap U|U \in \mathscr{T} \} \nonumber
	\end{equation}
	is a topology on $ Y $, called the \textit{subspace topology}. With this topology, $ Y $ is called a \textit{subspace} of $ X $; its open sets consist of all intersections of open sets of $ X $ with $ Y $.
\end{Definition}
\begin{Lemma}
	If $ \mathscr{B} $ is a basis for the topology of $ X $ then the collection
	\begin{equation}
		\mathscr{B}_Y = \{B \cap Y | B \in \mathscr{B} \} \nonumber
	\end{equation}
	is a basis for the subspace topology on $ Y $.
\end{Lemma}	
\begin{proof}
Given $ U $ open in $ X $ and given $ y\in U \subset Y $, we can choose an element $ B $ of $ \mathscr{B} $ such that $ y \in B \subset U $. Then $ y \in B \cap Y \subset U \cap Y $. It follows that $ \mathscr{B} $ is a basis for the subspace topology on $ Y $.
\end{proof}
\begin{Definition}
	If $ Y $ is a subspace of $ X $, a set $ U $ is \textit{open in $ Y $}(or \textit{open relative to $ Y $}) if it belongs to the topology of $ Y $; this implies in particular that it is a subset of $ Y $. We say that $ U $ is \textit{open in $ X $} if it belongs to the topology of $ X $.
\end{Definition}
\begin{Lemma}
	Let $ Y $ be a subspace of $ X $. If $ U $ is open in $ Y $ and $ Y $ is open in $ X $, then $ U $ is open in $ X $.
\end{Lemma}	
\begin{proof}
	Since $ U $ is open in $ Y $, $ U=Y\cap V $ for some set $ V $ open in $ X $. Since $ Y $ and $ V $ are both open in $ X $, so is $ Y \cap V $.
\end{proof}
\begin{Theorem}
	If $ A $ is a subspace of $ X $ and $ B $ is a subspace of $ Y $, then the product topology on $ A \times B $ is the same as the topology $ A \times B $ inherits as a subspace of $ X \times Y $.
\end{Theorem}
\begin{proof}
	The set $ U \times V $ is the general basis element for $ X \times Y $, where $ U $ is open in $ X $ and $ V $ is open in $ Y $. Therefore, $ (U\times V)\cap (A \times B) $ is the general basis element for the subspace topology on $ A \times B $. Now
	\begin{equation}
		(U \times V)\cap (A \times B)=(U \cap A)\times (V \cap B) \nonumber
	\end{equation}
	The set $ (U \cap A)\times (V \cap B) $ is the general basis element for the product topology on $ A \times B $.
\end{proof}
If $ X $ is an ordered set and $ Y $ is a subset of it. The order relation on $ X $, when restricted to $ Y $, makes $ Y $ into an ordered set. However, \textbf{the resulting order topology on $ Y $ need not be the same as the topology inherits as a subspace of $ X $}. We'll give two examples below.
\begin{Example}
	Let $ Y $ be the subset $ [0,1) \cup \{2\} $ of $ \Real $. In the subspace topology on $ Y $ the one-point set $ \{2\} $ is open, because it is the intersection of the open set $ (\frac{3}{2},\frac{5}{2}) $ with $ Y $. But in the order topology on $ Y $, $ \{2\} $ is not open. Any basis element for the order topology on $ Y $ that contains $ 2 $ is of the form
	\begin{equation}
		\{x|x\in Y\land a<x \leqslant 2 \} \nonumber
	\end{equation}
	for some $ a \in Y $; such a set necessarily contains points of $ Y $ less than $ 2 $.
\end{Example}
\begin{Example}
	Let $ I=[0,1] $. The dictionary order on $ I \times I $ is just the restriction to $ I \times I $ of the dictionary order on the plane $ \Real \times \Real $. However, the dictionary order topology on $ I \times I $ is not the same as the subspace topology on $ I \times I $ obtained from the dictionary topology on $ \Real \times \Real $. For example, the set $ \{1/2 \}\times (1/2,1] $ is open in $ I \times I $ in the subspace topology, but not in the order topology.
\end{Example}
The set $ I \times I $ in the dictionary order topology will be  called the \textit{ordered square} and denoted by $ I_o^2 $.
\begin{Definition}
	Given an ordered set $ X $, a subset $ Y $ is \textit{convex} in $ X $ if for each pair of points $ a<b $ of $ Y $, the entire interval $ (a,b) $ of points of $ X $ lies in $ Y $. Intervals and rays in $ X $ are convex in $ X $.
\end{Definition}
\begin{figure}[h]
	\includegraphics[width=\textwidth]{Convex_polygon_illustration1.png}
	\caption{A convex set}
\end{figure}
\begin{figure}[h]
	\includegraphics[width=\textwidth]{Convex_polygon_illustration2.png}
	\caption{A non-convex set}	
\end{figure}
\newpage
\begin{Theorem}
	Let $ X $ be an ordered set in the order topology; Let $ Y $ be a subset of $ X $ that is convex in $ X $. Then the order topology on $ Y $ is the same as the topology $ Y $ inherits as a subspace of $ X $.
\end{Theorem}
\begin{proof}
	Consider the ray $ (a,+\infty) $ in $ X $. If $ a \in Y $, its intersection with $ Y $ is
	\begin{equation}
		(a,+\infty)\cap Y =\{x|x\in Y \land x>a \} \nonumber
	\end{equation}
	this is an open ray of the ordered set $ Y $. If $ a \notin Y $, then $ a $ is either a lower bound on $ Y $ or an upper bound on $ Y $ since $ Y $ is convex. In the former case, the set $ (a,+\infty)\cap Y $ equals all of $ Y $; in the latter case, it is empty. \newpara
	A similar argument holds for the intersection of $ (-\infty,a) $ and $ Y $. Since the sets $ (a,+\infty) \cap Y$ and $ (-\infty,a)\cap Y $ form a subbasis for the subspace topology on $ Y $, and since each is open in the order topology, the order topology contains(or is coarser than) the subspace topology. \newpara
	Note that any open ray of $ Y $ equals the intersection of an open ray of $ X $ with $ Y $, so it is open in the subspace topology on $ Y $. Since the open rays of $ Y $ are a subbasis for the order topology on $ Y $, this topology is contained in(or finer than) the subspace topology.
\end{proof}
To avoid ambiguity, if $ X $ is an ordered set in the order topology and $ Y $ is a subset of $ X $, we shall assume that $ Y $ is given the subspace topology unless specified.

\section{Closed Sets and Limit Points}	
\subsection{Closed Sets}
\begin{Theorem}
	Let $ X $ be a topological space. Then the following conditions hold:
	\begin{enumerate}
		\item $ \varnothing $ and $ X $ are closed.
		\item Arbitrary intersections of closed sets are closed.
		\item Finite unions of closed sets are closed.
	\end{enumerate}
\end{Theorem}	
\begin{proof}
	$ (1) $ is obvious. Given a collection of closed sets $ \{A_\alpha \}\alpha \in J $, we have $ X- \bigcap_{\alpha \in J} =\bigcup_{\alpha \in J}(X-A_\alpha)$. Since the set on the right hand side is open, $ \bigcap A_\alpha $ is closed. The third condition can be verified similarly.
\end{proof}
If $ Y $ is a subspace of $ X $, we say that a set $ A $ is \textit{closed in $ Y $} if $ A $ is a subset of $ Y $ and if $ A $ is closed in the subspace topology of $ Y $(that is, if $ Y \setminus A $ is open in $ Y $).
\begin{Theorem}
	Let $ Y $ be a subspace of $ X $. Then a set $ A $ is closed in $ Y $ iff it equals the intersection of a closed set of $ X $ with $ Y $.
\end{Theorem}	
\begin{proof}
	Assume that $ A=C \cap Y $, where $ C $ is closed in $ X $. Then $ X \setminus C $ is open in $ X $, so that $ (X\setminus C)\cap Y $ is open in $ Y $, but then $ (X\setminus C)\cap Y=Y \setminus A $, and hence $ A $ is closed in $ Y $. Conversely, assume that $ A $ is closed in $ Y $. Then $ Y \setminus A $ is open in $ Y $, and by definition it equals the intersection of an open set $ U $ of $ X $ with $ Y $. The set $ X \setminus U $ is closed in $ X $, so $ A=Y\cap (X \setminus U) $, as desired.
\end{proof}
\begin{Theorem}
	Let $ Y $ be a subspace of $ X $. If $ A $ is closed in $ Y $ and $ Y $ is closed in $ X $, then $ A $ is closed in $ X $.
\end{Theorem}	
\subsection{Closure and Interior of a Set}
\begin{Definition}
	Given a subset $ A $ of a topological space $ X $, the \textit{interior} of $ A $ is defined as the union of all open sets contained in $ A $, and the \textit{closure} of $ A $ is defined as the intersection of all closed sets containing $ A $. The interior is denoted by $ \Int A $ and the closure of $ A $ is denoted by $ \Cl A $ or $ \bar{A} $. Obviously $ \Int A $ is open and $ \bar{A} $ is closed. Furthermore
	\begin{equation}
		\Int A \subset A \subset \bar{A} \nonumber
	\end{equation}
	If $ A $ is open, then $ A=\Int A $; if $ A $ is closed, $ A = \bar{A} $.
\end{Definition}
\begin{Theorem}
	Let $ Y $ be a subspace of $ X $; let $ A $ be a subset of $ Y $; let $ \bar{A} $ denote the closure of $ A $ in $ X $. Then the closure of $ A $ in $ Y $ equals $ \bar{A} \cap Y $.
\end{Theorem}
\begin{proof}
	Let $ B $ denote the closure of $ A $ in $ Y $. The set $ \bar{A} $ is closed in $ X $, so $ \bar{A}\cap Y $ is closed. Since $ \bar{A} $ contains $ A $, we have that $ B \subset (\bar{A}\cap Y) $. \newpara
	On the other hand, $ B $ is closed in $ Y $. Hence $ B=C \cap Y $ for some $ C $ closed in $ X $. Then $ C $ is a closed set of $ X $ containing $ A $; because $ \bar{A} $ is the intersection of all such closed sets, we have $ \bar{A} \subset C$. Then $ (\bar{A}\cap Y)\subset (C \cap Y)=B $.
\end{proof}
\begin{Theorem}
	Let $ A $ be a subset of the topological space $ X $.
	\begin{enumerate}
		\item Then $ x \in \bar{A} $ iff every open set $ U $ containing $ x $ intersects $ A $.
		\item Supposing the topology of $ X $ is given by a basis, then $ x \in \bar{A} $ iff every basis element $ B $ containing $ x $ intersects $ A $.
	\end{enumerate}
\end{Theorem}
\begin{proof}
	If $ x \notin \bar{A} $, the set $ U=X\setminus \bar{A} $ is an open set containing $ x $ that does not intersect $ A $. Conversely, if there exists an open set $ U $ containing $ x $ which does not intersect $ A $, then $ X \setminus U $ is a closed set containing $ A $. By the definition of the closure, the set $ X \setminus U $ must contain $ \bar{A} $; therefore, $ x \notin \bar{A} $. The second statement follows from our proof, since any basis element is open.
\end{proof}
\subsection{Limit Points}
\begin{Definition}
	If $ A $ is a subset of the topological space $ X $ and if $ x $ is a point of $ X $, then $ x $ is a \textit{limit point}(or \textit{cluster point} or \textit{point of accumulation}) of $ A $ if every neighborhood of $ x $ intersects $ A $ in some point other than $ x $ itself. Equivalently, $ x $ is a limit point of $ A $ if it belongs to the closure of $ A\setminus{x} $.
\end{Definition}	
\begin{Theorem}
	Let $ A $ be a subset of the topological space $ X $; let $ A^\prime $ be the set of all limit points of $ A $. Then
	\begin{equation}
		\bar{A}=A \cup A^\prime \nonumber
	\end{equation}
\end{Theorem}
\begin{proof}
	If $ x $ is in $ A^\prime $, every neighborhood of $ x $ intersects $ A $. Therefore, $ x $ belongs to $ \bar{A} $. Hence $ A^\prime \subset \bar{A} $. By definition $ A \subset \bar{A} $, and it follows that $ A \cup A^\prime \subset \bar{A} $. \newpara
	Now let $ x $ be a point of $ \bar{A} $. If $ x $ lies in $ A $, the relation $ x \in A \cup A^\prime $ is trivial; suppose that $ x $ is not in $ A $. Since $ x \in \bar{A}$, we know that every neighborhood $ U $ of $ x $ intersects $ A $. Then $ x \in A^\prime $, so that $ x \in A \cup A^\prime $.
\end{proof}
\begin{Corollary}
	A subset of a topological space is closed iff it contains all its limit points.
\end{Corollary}
\section{Hausdorff Spaces}
\begin{Definition}
	A topological space is called a \textit{Hausdorff space} if each pair of points has disjoint neighborhoods.
\end{Definition}	
\begin{Theorem}
	Every finite point set in a Hausdorff space is closed.
\end{Theorem}
\begin{proof}
	It's sufficient to prove that the closure of any one-point set is itself, and this is true since for any point of the complement of this one-point set there exists a neighborhood that does not intersect with the original set.
\end{proof}
\begin{Definition}
	The condition that finite point sets be closed is called the \textit{$ T_1 $ axiom}.
\end{Definition}
\begin{Theorem}
	Let $ X $ be a space satisfying the $ T_1 $ axiom; let $ A $ be a subset of $ X $. Then the point $ x $ is a limit point of $ A $ iff every neighborhood of $ x $ contains infinitely many points of $ A $.
\end{Theorem}
\begin{proof}
	If every neighborhood of a limit point $ x $ only intersects $ A $ at finitely many number of points, then by the $ T_1 $ axiom, any neighborhood of $ x $ is closed, and the complement of any deleted neighborhood of $ x $ is open. Then the intersection of the original neighborhood of $ x $ with this complement is a neighborhood of $ x $, but it does not intersect with $ A $ at all, and this contradicts to the fact that $ x $ is a limit point.
\end{proof}
\begin{Theorem}
	If $ X $ is a Hausdorff space, then a sequence of points of $ X $ converges to at most one point of $ X $.
\end{Theorem}
\begin{proof}
	Suppose that $ x_n $ is a sequence of points of $ X $ that converges to $ x $. If $ y \neq x $, let $ U $ and $ V $ be disjoint neighborhoods of $ x $ and $ y $, respectively. Since $ U $ contains $ x_n $ for all but a finitely many values of $ n $, the set $ V $ cannot. Then $ x_n $ cannot converge to $ y $.
\end{proof}
\begin{Theorem}
	Every simply ordered set is a Hausdorff space in the order topology. The product of two Hausdorff spaces is a Hausdorff space. A subspace of a Hausdorff space is a Hausdorff space.
\end{Theorem}
\begin{proof}
	The first two statements can be verified easily. For any pair of points of this subspace, one can choose two disjoint neighborhoods, and thus their intersection with this subspace are disjoint neighborhoods of this two points respectively in the subspace, but since the subspace topology is just the intersection of open sets in the original topological space with the subspace, we had done selecting two disjoint neighborhoods of two points in the subspace topology.
\end{proof}
\section{Continuous Functions}
\subsection{Continuity of a Function}
\begin{Definition}
	Let $ X $ and $ Y $ be topological spaces. A function $ f:X \to Y $ is said to be \textit{continuous} if for each open subset $ V $ of $ Y $, the set $ f^{-1}(V) $ is an open subset of $ X $. If the topology of $ Y $ is given by a basis $ \mathscr{B} $, then to prove continuity of $ f $ it suffices to show that the inverse image of every basis element is open, since any arbitrary open set of $ Y $ can be written as a union of basis elements. If the topology $ Y $ is given by a subbasis $ \mathscr{S} $, to prove continuity of $ f $ it will suffice to show that the inverse image of each subbasis element is open, since the arbitrary basis element of $ Y $ can be written as a finite intersection of subbasis elements.
\end{Definition}
\begin{Theorem}
	Let $ X $ and $ Y $ be topological spaces; let $ f:X \to Y $. Then the following are equivalent:
	\begin{enumerate}
		\item $ f $ is continuous.
		\item For every subset $ A $ of $ X $, one has $ f(\bar{A})\subset \overline{f(A)}$.
		\item For every closed set $ B $ of $ Y $, the set $ f^{-1}(B) $ is closed in $ X $.
		\item For each $ x \in X $ and each neighborhood $ V $ of $ f(x) $, there is a neighborhood $ U $ of $ x $ such that $ f(U)\subset(V) $.
	\end{enumerate}
If condition $ (4) $ holds for the point $ x $ of $ X $, we say that $ f $ is \textit{continuous at the point $ x $}.
\end{Theorem}
\begin{proof}
	$ (1)\Rightarrow (2) $. Let $ A $ be a subset of $ X $ and $ V $ be a neighborhood of $ f(x) $. Assume that $ x \in \bar{A} $. Then $ f^{-1}(V) $ is an open set of $ X $ containing $ x $; it must intersect $ A  $ in some point $ y $. Then $ V $ intersects $ f(A) $ in the point $ f(y) $, so that $ f(x)\in \overline{f(A)} $, as desired. \newpara
	$ (2)\Rightarrow (3) $. Let $ B $ be closed in $ Y $ and let $ A = f^{-1}(B) $. We have $ f(A)= f(f^{-1}(B))\subset B $. Therefore, if $ x \in \bar{A} $,
	\begin{equation}
		f(x)\in f(\bar{A}) \subset \overline{f(A)}\subset \bar{B} = B \nonumber
	\end{equation}
	so that $ x \in f^{-1}(B)=A $. Thus $ \bar{A} \subset A $, and $ \bar{A} = A $. \newpara
	$ (3)\Rightarrow(1) $. Let $ V $ be an open set of $ Y $. Set $ B=Y\setminus V $. Then
	\begin{equation}
		f^{-1}(B)=f^{-1}(Y)\setminus f^{-1}(V)=X \setminus f^{-1}(V) \nonumber
	\end{equation}
	now $ f^{-1}(B) $ is closed by hypothesis, and $ f^{-1}(V) $ is open in $ X $, as desired.\newpara
	$ (1)\Leftrightarrow(4) $. Let $ x \in X $ and let $ V $ be a neighborhood of $ f(x) $. Then the set $ U=f^{-1}(V) $ is a neighborhood of $ x $ such that $ f(U)\subset V $. Conversely, let $ V $ be an open set of $ Y $; let $ x $ be a point of $ f^{-1}(V) $. Then $ f(x)\in V $, and by hypothesis there is a neighborhood $ U_x $ of $ x $ such that $ f(U_x)\subset V $. Then $ U_x \subset f^{-1}(V)$. It follows that $ f^{-1}(V) $ can be written as the union of the open sets $ U_x $, so that it is open.
\end{proof}
\subsection{Homeomorphisms}
\begin{Definition}
	If a bijection function between two topological spaces and its inverse function are continuous, the original function is called a \textit{homeomorphism}. Equivalently, it means that a homeomorphism is a bijective mapping such that the image of a set is open iff the original set is open. It is analogous with the concept of isomorphism for algebraic structures, which means a homeomorphism preserves the topological structure involved.
\end{Definition}	
\begin{Definition}
	Suppose $ f:X \to Y $ is an injective continuous mapping between two topological spaces. Let $ Z $ be the image set $ f(X) $, considered as a subspace of $ Y $; then the function $ f^\prime:X \to Z $ obtained by restricting the range of $ f $ is bijective. If $ f^\prime $ happens to be a homeomorphism of $ X $ with $ Z $, it is said that the mapping $ f:X \to Y $ is a \textit{topological imbedding}, or an \textit{imbedding}, of $ X $ in $ Y $.
\end{Definition}
\begin{Example}
	The bijective order-preserving correspondence $ F:(1,-1)\to \Real $ defined by
	\begin{equation}
		F(x)=\frac{x}{1-x^2} \nonumber
	\end{equation}
	is a homeomorphism between $ (-1,1) $ with $ \Real $. Its inverse $ G $ is
	\begin{equation}
		G(y)=\frac{2y}{1+(1+4y^2)^{1/2}} \nonumber
	\end{equation}
\end{Example}
\section{Constructing Continuous Functions}
\begin{Theorem}[Rules for Constructing Continuous Functions]
	Let $ X $, $ Y $, and $ Z $ be topological spaces.
	\begin{enumerate}
		\item (Constant function) If $ f:X \to Y $ maps all of $ X $ into the single point $ y_0 $ of $ Y $, then $ f $ is continuous.
		\item (Inclusion) If $ A $ is a subspace of $ X $, the inclusion function $ j:A \to X $ is continuous.
		\item (Composites) If $ f:X \to Y $ and $ g:Y \to Z $ are continuous, then the map $ g \circ f $ is continuous.
		\item (Restricting the domain) If $ f:X \to Y $ is continuous, and if $ A $ is a subspace of $ X $, then the restricted function $ f|_A:A \to Y $ is continuous.
		\item (Restricting or expanding the range) Let $ f:X \to Y $ be continuous. If $ Z $ is a subspace of $ Y $ containing the image set $ f(X) $, then the function $ g:X \to Z $ obtained by restricting the range of $ f $ is continuous. If $ Z $ is a space having $ Y $ as a subspace, then the function $ h:X \to Z $ obtained by expanding the range of $ f $ is continuous.
		\item (Local formulation of continuity) The map $ f:X \to Y $ is continuous if $ X $ can be written as the union of open sets $ U_\alpha $ such that $ f|_{U_\alpha} $ is continuous for each $ \alpha $.
	\end{enumerate}
\end{Theorem}
\begin{proof}
	\begin{enumerate}
		\item Let $ f(x)=y_0 $ for each $ x \in X $ and let $ V $ be open in $ Y $. The set $ f^{-1}(V) $ equals $ X $ or $ \varnothing $, depending on whether $ V $ contains $ y_0 $ or not. In either case, it is open.
		\item If $ U $ is open in $ X $, then $ j^{-1}(U)=U\cap A $, which is open in $ A $.
		\item If $ U $ is open in $ Z $, then $ g^{-1}(U) $ is open in $ Y $ and $ f^{-1}(g^{-1}(U)) $ is open in $ X $, but
		\begin{equation}
			f^{-1}(g^{-1}(U)) =(g \circ f)^{-1}(U) \nonumber
		\end{equation}
		\item The function $ f|_A$ equals the composite of the inclusion mapping $ j:A \to X $ and the mapping $ f:X \to Y $, both of which are continuous.
		\item Let $ f:X \to Y $ be continuous. If $ f(X)\subset Z \subset Y $, we show that the function $ g:X \to Z $ obtained from $ f $ is continuous. Let $ B $ be open in $ Z $. Then $ B=Z \cap U $ for some open set $ U $ of $ Y $. Because $ Z $ contains the entire image set $ f(X) $,
		\begin{equation}
			f^{-1}(U)=g^{-1}(B)\nonumber
		\end{equation}
		Since the left side is open by the definition of continuity, so is $ g^{-1}(B) $.\\
		To show $ h:X \to Z $ is continuous if $ Z $ has $ Y $ as a subspace, note that $ h $ is the composition of the mapping $ f:X \to Y $ and the inclusion map $ j:Y \to Z $.
		\item By hypothesis, we can write $ X $ as a union of open sets $ U_\alpha $, such that $ f|_{U_\alpha} $, is continuous for each $ \alpha $. Let $ V $ be an open set in $ Y $. Then
		\begin{equation}
			f^{-1}(V)\cap U_\alpha = (f|_{U_\alpha})^{-1}(V) \nonumber
		\end{equation}
		because both expressions represent the set of those points $ x $ lying in $ U_\alpha $ for which $ f(x)\in V $. Since $ f|_U $ is continuous, this set is open in $ U_\alpha $, and hence open in $ X $. But
		\begin{equation}
			f^{-1}(V)=\bigcup_{\alpha}(f^{-1}(V)\cap U_\alpha) \nonumber
		\end{equation}
		so that $ f^{-1}(V) $ is also open in $ X $.
	\end{enumerate}
\end{proof}	
\begin{Theorem}[The pasting lemma]
	Let $ X = A \cup B $, where $ A $ and $ B $ are closed in $ X $. Let $ f:A \to Y $ and $ g:B \to Y $ be continuous. If $ f(x)=g(x) $ for every $ x \in A \cap B $, then $ f $ and $ g $ combine to give a continuous function $ h:X \to Y $, defined by setting $ h(x)=f(x) $ if $ x \in A $, and $ h(x)=g(x) $ if $ x \in B $.
\end{Theorem}
\begin{proof}
	Let $ C $ be a closed subset of $ Y $. Now
	\begin{equation}
		h^{-1}(C)=f^{-1}(C)\cup g^{-1}(C) \nonumber
	\end{equation}
	The fact that $ f $ and $ g $ are continuous make each term of the right-hand side closed in their respectively domain, and hence they are closed in $ X $, which makes $ h^{-1}(C) $ closed in $ X $.
\end{proof}
\begin{Theorem}[Maps into products]
	Let $ f:A \to X \times Y $ given by the equation
	\begin{equation}
		f(a)=(f_1(a),f_2(a))\nonumber
	\end{equation}
	Then $ f $ is continuous iff the functions $ f_1 $ and $ f_2 $ are continuous.
\end{Theorem}
The mapping $ f_1 $ and $ f_2 $ are called the \textit{coordinate functions} of $ f $.
\begin{proof}
	Let $ \pi_1:X \times Y \to X $ and $ \pi_2 :X \times Y \to Y $ be projections onto the first and second factors, respectively. These maps are continuous. For $ \pi_1^{-1}(U)=U \times Y $ and $ \pi_2^{-1}(V)=X \times V $, and these sets are open iff $ U $ and $ V $ are open. Note that for each $ a \in A$
	\begin{equation}
		f_1(a)=\pi_1(f(a))\quad \text{and}\quad f_2(a)=\pi_2(f(a)) \nonumber
	\end{equation}
	If $ f $ is continuous, then $ f_1 $ and $ f_2 $ are composition of continuous functions and therefore continuous. Conversely, suppose that $ f_1 $ and $ f_2 $ are continuous. We show that for each basis element $ U \times V $ for the topology of $ X \times Y $, its inverse image $ f^{-1}(U \times V) $ is open. A point $ a $ is in $ f^{-1}(U \times V) $ iff $ f(a)\in U \times V $, therefore
	\begin{equation}
		f^{-1}(U \times V) = f_1^{-1}(U)\cap f_2^{-1}(V)\nonumber
	\end{equation}
	Since both terms of the right-hand side are open, so is their intersection.
\end{proof}
\section{The Product Topology}	
\begin{Definition}
	Let $ J $ be an index set. Given a set $ X $, we define a \textit{$ J $-tuple} of elements of $ X $ to be a function $ x:J \to X $. If $ \alpha $ is an element of $ J $, the value of $ x $ at $ \alpha $ is usually denoted by $ x_\alpha $ and called the $ \alpha $th \textit{coordinate} of $ x $. The function $ x $ itself is denoted by
	\begin{equation}
		(x_\alpha)_{\alpha \in J} \nonumber
	\end{equation}
	The set of all $ J $-tuples of elements of $ X $ is denoted by $ X^J $.
\end{Definition}
\begin{Definition}
	Let $ \{A_\alpha \}_{\alpha \in J} $ be an indexed family of sets; let $ X=\bigcup_{\alpha \in J}A_\alpha $. The \textit{cartesian product} of this indexed family, denoted by
	\begin{equation}
		\prod_{\alpha \in J}A_\alpha \nonumber
	\end{equation}
	is defined to be the set of all $ J $-tuples $ (x_\alpha)_{\alpha \in J} $ of elements of $ X $ such that $ x_\alpha \in A_\alpha $ for each $ \alpha \in J $. That is, it is the set of all functions
	\begin{equation}
		x:J \to \bigcup_{\alpha \in J}A_\alpha \nonumber
	\end{equation}
	such that $ x(\alpha)\in A_\alpha $ for each $ \alpha \in J $.
\end{Definition}
\begin{Definition}
	Let $ \{X_\alpha \}_{\alpha \in J} $ be an indexed family of topological spaces. Let us take as a basis for a topology on the product space
	\begin{equation}
		\prod_{\alpha \in J}X_\alpha \nonumber
	\end{equation}
	the collection of all sets of the form
	\begin{equation}
		\prod_{\alpha \in J}U_\alpha \nonumber
	\end{equation}
	where $ U_\alpha $ is open in $ X_\alpha $. The topology generated by this basis is called the \textit{box topology}.
\end{Definition}	
\begin{Definition}
	The \textit{projection mapping} associated with the index $ \beta $ is defined as a function assigning to each element of the product space its $ \beta $th coordinate.
	\begin{equation}
		\pi_\beta:\prod_{\alpha \in J}X_\alpha \to X_\beta ,\quad \pi_\beta((x_\alpha)_{\alpha \in J}) = x_\beta \nonumber
	\end{equation}
\end{Definition}	
\begin{Definition}
	Let $ \mathscr{S}_\beta $ denote the collection
	\begin{equation}
		\mathscr{S}_\beta = \{\pi_\beta^{-1}(U_\beta)|U_\beta \text{ open in }X_\beta \} \nonumber
	\end{equation}
	and let $ \mathscr{S} $ denote the union of these collections,
	\begin{equation}
		\mathscr{S}=\bigcup_{\beta \in J}\mathscr{S}_\beta \nonumber
	\end{equation}
	The topology generated by the subbasis $ \mathscr{S} $ is called the \textit{product topology}. In this topology $ \prod_{\alpha \in J}X_\alpha $ is called a \textit{product space}.
\end{Definition}
\begin{Theorem}[Comparison of the Box and Product Topologies]
	The boc topology on $ \prod X_\alpha $ has as basis all sets of the form $ \prod U_\alpha $, where $ U_\alpha $ is open in $ X_\alpha $ for each $ \alpha $. The product topology on $ \prod X_\alpha $ has as basis all sets of the form $ \Pi U_\alpha $, where $ U_\alpha $ is open in $ X_\alpha $ for each $ \alpha $ and $ U_\alpha $ equals $ X_\alpha $ except for finitely many values of $ \alpha $.
\end{Theorem}
For finite products $ \prod_{\alpha =1}^{n}X_\alpha $ the two topologies are the same. Generally the box topology is finer than the product topology.
\begin{Theorem}
	Suppose the topology on each space $ X_\alpha $ is given by a basis $ \mathscr{B}_\alpha $. The collection of all sets of the form
	\begin{equation}
		\prod_{\alpha \in J}B_\alpha \nonumber
	\end{equation}
	where $ B_\alpha \in \mathscr{B}_\alpha $ for each $ \alpha $, will serve as a basis for the box topology on $ \prod_{\alpha \in J}X_\alpha $.\newpara
	The collection of all sets of the same form, where $ B_\alpha \in \mathscr{B}_\alpha $ for finitely many indices $ \alpha $ and $ B_\alpha = X_\alpha $ for all the remaining indices, will serve as a basis for the product topology $ \prod_{\alpha \in J}X_\alpha $.
\end{Theorem}	
\begin{proof}
	\exercise
\end{proof}
\begin{Theorem}
	Let $ A_\alpha $ be a subspace of $ X_\alpha $, for each $ \alpha \in J $. Then $ \prod A_\alpha $ is a subspace of $ \prod X_\alpha $ if both products are given the box topology, or if both products are given the product topology.
\end{Theorem}
\begin{proof}
	\exercise
\end{proof}
\begin{Theorem}
	If each space $ X_\alpha $ is Hausdorff space, then $ \prod X_\alpha $ is a Hausdorff space in both the box and product topologies.
\end{Theorem}
\begin{proof}
	\exercise
\end{proof}
\begin{Theorem}
	Let $ \{X_\alpha \} $ be an indexed family of spaces; let $ A_\alpha \subset X_\alpha$ for each $ \alpha $. If $ \prod X_\alpha $ is given either the product or the box topology, then
	\begin{equation}
		\prod \bar{A}_\alpha = \overline{\prod A_\alpha} \nonumber
	\end{equation}
\end{Theorem}
\begin{proof}
	Let $ x=(x_\alpha) $ be a point of $ \prod \bar{A}_\alpha $; we show that $ x \in \overline{\prod A_\alpha} $. Let $ U = \prod U_\alpha $ be a basis element for either the box or product topology that contains $ x $. Since $ x_\alpha \in \bar{A}_\alpha$, we can choose a point $ y_\alpha \in U_\alpha \cap A_\alpha $ for each $ \alpha $. Then $ y=(y_\alpha) $ belongs to both $ U $ and $ \prod A_\alpha $. Since $ U $ is arbitrary, it follows that $ x $ belongs to the closure of $ \prod A_\alpha $. \newpara
	Conversely, suppose $ x=(x_\alpha) $ lies in the closure of $ \prod A_\alpha $, in either topology. We show that for any given index $ \beta $, we have $ x_\beta \in \bar{A}_\beta $. Let $ V_\beta $ be an arbitrary open set of $ X_\beta $ containing $ x_\beta $. Since $ \pi_\beta^{-1}(V_\beta) $ is open in $ \prod X_\alpha $ in either topology, it contains a point $ y=(y_\alpha) $ if $ \prod A_\alpha $. Then $ y_\beta $ belongs to $ V_\beta \cap A_\beta $. It follows that $ x_\beta \in \bar{A}_\beta $.
\end{proof}
\begin{Theorem}
	Let $ f:A \to \prod_{\alpha \in J}X_\alpha $ be given by the equation
	\begin{equation}
		f(a)=(f_\alpha(a))_{\alpha \in J} \nonumber
	\end{equation}
	where $ f_\alpha :A \to X_\alpha$ for each $ \alpha $. Let $ \prod X_\alpha $ have the product topology. Then the function $ f $ is continuous iff each function $ f_\alpha $ is continuous.
\end{Theorem}	
\begin{proof}
	Let $ \pi_\beta $ be the projection of the product onto its $ \beta $th factor. The function $ \pi_\beta $ is continuous, for if $ U_\beta $ is open in $ X_\beta $, the set $ \pi_{\beta}^{-1}(U_\beta) $ is a subbasis element for the product topology on $ X_\alpha $. Now suppose that $ f:A \to \prod X_\alpha $ is continuous. The function $ f_\beta $ equals the composite $ \pi_\beta \circ f $; being the composite of two continuous functions, it is continuous. \newpara
	Conversely, suppose that each coordinate function $ f_\alpha $ is continuous. To prove that $ f $ is continuous, it suffices to prove that the inverse image under $ f $ of each subbasis element is open in $ A $; A typical subbasis element for the product topology on $ \prod X_\alpha $ is a set of the form $ \pi_\beta^{-1}(U_\beta) $, where $ \beta $ is some index and $ U_\beta $ is open in $ X_\beta $. Now
	\begin{equation}
		f^{-1}(\pi_\beta^{-1}(U_\beta))=f^{-1}_\beta (U_\beta) \nonumber
	\end{equation}
	because $ f_\beta = \pi_\beta \circ f $. Since $ f_\beta $ is continuous, this set is open in $ A $, as desired.
\end{proof}
The theorem only holds on product topology, and we'll present a counterexample below.
\begin{Example}
	Consider $ \Real^\omega $, the countably infinite product of $ \Real $ with itself. Recall that
	\begin{equation}
		\Real^\omega = \prod_{n\in \Zahlen_+} X_n \nonumber
	\end{equation}
	where $ X_n=\Real $ for each $ n $. Let us defined a function $ f:\Real \to \Real^\omega $ by the equation
	\begin{equation}
		f(t)=(t,t,t,\cdots)\nonumber
	\end{equation}
	the $ n $th coordinate function of $ f $ is the function $ f_n(t)=t $. Each of the coordinate functions is continuous; therefore, the function $ f $ is continuous if given the product topology. But $ f $ is not continuous if $ \Real^\omega $ is given the box topology. Consider, for example. the basis element
	\begin{equation}
		B=(-1,1)\times (-\frac{1}{2},\frac{1}{2})\times (-\frac{1}{3},\frac{1}{3})\times \cdots \nonumber
	\end{equation}
	for the box topology. We assert that $ f^{-1}(B) $ is not open in $ \Real $. If $ f^{-1}(B) $ were open in $ \Real $, it would contain some interval $ (-\delta,\delta) $ about the point $ 0 $. This would mean that $ f((-\delta,\delta))\subset B $, so that, applying $ \pi_n $ to both sides of the inclusion,
	\begin{equation}
		f_n((-\delta,\delta))=(-\delta,\delta)\subset (-1/n,1/n) \nonumber
	\end{equation}
	for all $ n $, a contradiction.
\end{Example}

\section{The Metric Topology}	
\begin{Definition}
	A \textit{metric} on a set $ X $ is a function
	\begin{equation}
		d:X \times X \to \Real \nonumber
	\end{equation}
	having the following properties:
	\begin{enumerate}
		\item $ d(x,y)\geqslant 0 $ for all $ x,y \in X $; equality holds iff $ x=y $.
		\item $ d(x,y)=d(y,x) $ for all $ x,y \in X $.
		\item (Triangle inequality) $ d(x,y)+d(y,z)\geqslant d(x,z) $, for all $ x,y,z \in X $.
	\end{enumerate}
Given a metric $ d $ on $ X $, the number $ d(x,y) $ is often called the \textit{distance} between $ x $ and $ y $ in the metric $ d $.
\end{Definition}	
\begin{Definition}
	If $ d $ is a metric on the set $ X $, then the collection of all $ \varepsilon $-balls $ B_d(x,\varepsilon) $, for $ x \in X $ and $ \varepsilon >0$, is a basis for a topology on $ X $, called the \textit{metric topology} induced by $ d $.
\end{Definition}
	
\begin{Definition}
	If $ X $ is a topological space, $ X $ is said to be \textit{metrizable} if there exists a metric $ d $ on the set $ X $ that induces the topology of $ X $. A \textit{metric space} is a metrizable space $ X $ together with a specific metric $ d $ that gives the topology of $ X $.
\end{Definition}	
\begin{Definition}
	Let $ X $ be a metric space with metric $ d $. A subset $ A $ of $ X $ is said to be \textit{bounded} if there is some number $ M $ such that
	\begin{equation}
		d(a_1,a_2)\leqslant M \nonumber
	\end{equation}
	for every pair $ a_1,a_2 $ of points of $ A $. If $ A $ is bounded and nonempty, the \textit{diameter} of $ A $ is defined to be the number
	\begin{equation}
		d(A)=\sup \{d(a_1,a_2)|a_1,a_2 \in A \} \nonumber
	\end{equation}
\end{Definition}
\begin{Theorem}
	Let $ X $ be a metric space with metric $ d $. Define $ \bar{d}:X \times X \to \Real $ by the equation
	\begin{equation}
		\bar{d}(x,y)=\min \{d(x,y),1 \} \nonumber
	\end{equation}
	Then $ \bar{d} $ is a metric that induces the same topology as $ d $ and called the \textit{standard bounded metric} corresponding to $ d $.
\end{Theorem}
\begin{proof}
	We only need to check the triangle equality for $ \bar{d} $, and check the case either $ d(x,y)\geqslant 1 $ or $ d(y,z)\geqslant 1 $ as well as $ d(x,y)<1 $ and $ d(y,z)<1 $. Now we note that in any metric space, the collection $ \varepsilon $-balls with $ \varepsilon<1 $ forms a basis for the metric topology, for every basis element containing $ x $ contains such an $ \varepsilon $-ball centered at $ x $. It follows that $ d $ and $ \bar{d} $ induce the same topology on $ X $, because the collection of $ \varepsilon $-balls with $ \varepsilon<1 $ under these two metrices are the same collection.
\end{proof}
\begin{Definition}
	Given $ \vec{x}=(x_1,\cdots,x_n) $ in $ \Real^n $, we define the \textit{norm} of $ \vec{x} $ by the equation
	\begin{equation}
		\norm{\vec{x}} = (x_1^2+\cdots +x_n^2)^{1/2} \nonumber
	\end{equation}
	and we define the \textit{euclidean metric} $ d $ on $ \Real^n $ by the equation
	\begin{equation}
		d(\vec{x},\vec{y})=\norm{\vec{x}-\vec{y}} \nonumber
	\end{equation}
	We define the \textit{square metric} $ \rho $ by the equation
	\begin{equation}
		\rho(\vec{x},\vec{y}) = \max \{|x_1-y_1|,\cdots, |x_n-y_n|\} \nonumber
	\end{equation}
\end{Definition}	
\begin{Lemma}
	Let $ d $ and $ d^\prime $ be two metrices on the set $ X $; let $ \mathscr{T} $ and $ \mathscr{T}^\prime $ be the topology they induce, respectively. Then $ \mathscr{T}^\prime $ is finer than $ \mathscr{T} $ iff for each $ x \in X $ and each $ \varepsilon>0 $, there exists a $ \delta >0 $ such that
	\begin{equation}
		B_{d^\prime}(x,\delta) \subset B_{d}(x,\varepsilon) \nonumber
	\end{equation}
\end{Lemma}
\begin{Theorem}
	The topologies on $ \Real^n $ induced by the euclidean metric $ d $ and the square metric $ \rho $ are the same as the product topology on $ \Real^n $.
\end{Theorem}
\begin{proof}
	Let $ \vec{x} $ and $ \vec{y} $ be two points of $ \Real^nn  $. It is easy to check that
	\begin{equation}
		\rho(\vec{x},\vec{y})\leqslant d(\vec{x},\vec{y}) \leqslant \sqrt{n} \rho(\vec{x},\vec{y}) \nonumber
	\end{equation}
	The first inequality shows that $ B_d(\vec{x},\varepsilon) \subset B_\rho(\vec{x},\varepsilon)$ for all $ \vec{x} $ and $ \varepsilon $. Similarly, the second inequality snows that $ B_\rho (\vec{x},\varepsilon/\sqrt{n}) \subset B_d(\vec{x},\varepsilon)$,and then the two metric topologies are the same. \newpara
	Now we show that the product topology is the same as that given by the metric $ \rho $. First, let
	\begin{equation}
		B=(a_1,b_1)\times \cdots \times (a_n,b_n) \nonumber
	\end{equation}
	be a basis element for the product topology, and $ \vec{x}=(x_1,\cdots,x_n) $ be an element of $ B $. For each $ i $, there is an $ \varepsilon_i $ such that
	\begin{equation}
		(x_i-\varepsilon_i,x_i+\varepsilon_i)\subset (a_i,b_i) \nonumber
	\end{equation}
	choose $ \varepsilon = \min \{\varepsilon_1,\cdots,\varepsilon_n \} $. Then $ B_\rho(\vec{x},\varepsilon)\subset B $. As a result, the $ \rho $-topology is finer than the product topology. \newpara
	Conversely, let $ B_\rho(\vec{x},\varepsilon) $ be a basis element for the $ \rho $-topology. Given the element $ \vec{y}\in B_\rho(\vec{x},\varepsilon) $, we need to find a basis element $ B $ for the product topology such that
	\begin{equation}
		\vec{y}\in B \subset B_\rho(\vec{x},\varepsilon) \nonumber
	\end{equation}
	But this is trivial, for
	\begin{equation}
		B_\rho(\vec{x},\varepsilon)=(x_1-\varepsilon,x_1+\varepsilon)\times \cdots \times  (x_n-\varepsilon,x_n+\varepsilon) \nonumber
	\end{equation}
	is itself a basis element for the product topology.
\end{proof}	
\begin{Definition}
	Given an index set $ J $, and given points $ \vec{x}=(x_\alpha)_{\alpha \in J} $ and $ \vec{y}=(y_\alpha)_{\alpha \in J}$ of $ \Real^J $, let us define a metric $ \bar{\rho} $ on $ \Real^J $ by the equation
	\begin{equation}
		\bar{\rho}(\vec{x},\vec{y})=\sup \{\bar{d}(x_\alpha,y_\alpha)|\alpha \in J \}
	\end{equation}
	where $ \bar{d} $ is the standard bounded metric on $ \Real $. The metric $ \bar{\rho} $ is called the \textit{uniform metric} on $ \Real^J $, and the topology it induces is called the \textit{uniform topology}.
\end{Definition}
\begin{Theorem}
	The uniform topology on $ \Real^J $ is finer than the product topology and coarser than the box topology; these three topologies are all different if $ J $ is infinite.
\end{Theorem}
\begin{proof}
Suppose that we are given a point $ \vec{x} =(x_\alpha)_{\alpha \in J}$ and a product topology basis element $ \prod U_\alpha $ about $ \vec{x} $. Let $ \alpha_1,\cdots,\alpha_n $ be the indices for which $ U_\alpha \neq \Real $. Then for each $ i $, choose $ \varepsilon_i >0 $ so that the $ \varepsilon_i $-ball centered at $ x_\alpha $ in the $ \bar{d} $ matrix is contained in $ U_{\alpha_i} $; this is possible because $ U_{\alpha_i} $ is open in $ \Real $. Let $ \varepsilon=\min \{\varepsilon_1,\cdots,\varepsilon_n \} $; then the $ \varepsilon $-ball centered at $ \bar{\rho} $ metric is contained in $ \prod U_\alpha $. For if $ \vec{z} $ is a point of $ \Real^J $ such that $ \bar{\rho}(\vec{x},\vec{z})<\varepsilon $, then $ \bar{d}(\vec{x},\vec{z})<\varepsilon $ for all $ \alpha $, so that $ \vec{z}\in \prod U_\alpha $. It follows that the uniform topology is finer than the product topology. These two topologies are different since any point in a open set in the uniform topology does not have a neighborhood in the product topology that is contained in this open set(any neighborhood in the product topology is unbounded).   \newpara

On the other hand, let $ B $ be the $ \varepsilon $-ball centered at $ \vec{x} $ in the $ \bar{\rho} $ metric. Then the box neighborhood
\begin{equation}
	U=\prod (x_\alpha - \frac{1}{2}\varepsilon,x_\alpha + \frac{1}{2}\varepsilon) \nonumber
\end{equation}
of $ \vec{x} $ is contained in $ B $. For if $ \vec{y}\in U $, then $ \bar{d}(x_\alpha,y_\alpha)<\frac{1}{2}\varepsilon $ for all $ \alpha $, so that $ \bar{\rho}(\vec{x},\vec{y})\leqslant \frac{1}{2}\varepsilon $. Now consider the $ \varepsilon $-ball in the box topology for the point $ \vec{x} $. Clearly it is open in the box topology, but it's not open in the uniform topology, for the point $ (x_1+\varepsilon/2,x_2+2\varepsilon/3,x_3+3\varepsilon/4,\cdots) $ does not have any neighborhood that is contained in the previously mentioned $ \varepsilon $-ball in the uniform topology.\newpara

\end{proof}
\begin{Theorem}
	Let $ \bar{d}(a,b)=\min \{|a-b|,1 \} $ be the standard bounded metric on $ \Real $. If $ \vec{x} $ and $ \vec{y} $ are two points of $ \Real^\omega $, define
	\begin{equation}
		D(\vec{x},\vec{y}) =\sup \{ \frac{\bar{d}(x_i,y_i)}{i}\} \nonumber
	\end{equation}
	Then $ D $ is a metric that induces the product topology on $ \Real^\omega $.
\end{Theorem}
\begin{proof}
	The properties of a metric are satisfied trivially except for the triangle inequality, which is proved by noting that for all $ i $,
	\begin{equation}
		\frac{\bar{d}(x_i,z_i)}{i}\leqslant \frac{\bar{d}(x_i,y_i)}{i}+	\frac{\bar{d}(y_i,z_i)}{i} \leqslant D(\vec{x},\vec{y})+D(\vec{y},\vec{z}) \nonumber
	\end{equation}
	so that
	\begin{equation}
		\sup \{\frac{\bar{d}(x_i,z_i)}{i} \}\leqslant D(\vec{x},\vec{y})+D(\vec{y},\vec{z}) \nonumber
	\end{equation}
	
	Now we prove that $ D $ gives the product topology. First, let $ U $ be open in the metric topology and let $ \vec{x}\in U $; we find an open set $ V $ in the product topology such that $ \vec{x}\in V \subset U $. Choose an $ \varepsilon $-ball $ B_D(\vec{x},\varepsilon) $ lying in $ U $. Then choose $ N $ large enough that $ 1/N < \varepsilon $. Finally, let $ V $ be the basis element for the product topology
	\begin{equation}
		V=(x_1-\varepsilon,x_1+\varepsilon)\times \cdots \times (x_N-\varepsilon,x_N+\varepsilon)\times \Real \times \Real \times \cdots \nonumber
	\end{equation}
	We assert that $ V \subset B_D(\vec{x}\varepsilon) $: Given any $ \vec{y}\in\Real^\omega $,
	\begin{equation}
		\frac{\bar{d}(x_i,y_i)}{i}\leqslant \frac{1}{N} \quad \text{for}\quad i \geqslant N \nonumber
	\end{equation}
	Therefore,
	\begin{equation}
		D(\vec{x},\vec{y})\leqslant \max\{\frac{\bar{d}(x_1,y_1)}{1},\cdots,\frac{\bar{d}(x_N,y_N)}{N},\frac{1}{N} \} \nonumber
	\end{equation}
	If $ \vec{y} $ is in $ V $, this expression is less than $ \varepsilon $, so that $ V \subset B_D(\vec{x},\varepsilon) $, as desired. \newpara
	Conversely, consider a basis element
	\begin{equation}
		U=\prod_{i\in \Zahlen_+}U_i \nonumber
	\end{equation}
	for the product topology, where $ U_i $ is open in $ \Real $ for $ i=\alpha_1,\cdots,\alpha_n $ and $ U_i=\Real $ for all other indices $ i $. Given $ \vec{x}\in U $, we find an open set $ V $ of the metric topology such that $ \vec{x}\in V \subset U $. Choose an interval $ (x_i-\varepsilon_i,x_i+\varepsilon_i) $ in $ \Real $ centered about $ x_i $ and lying in $ U_i $ for $ i=\alpha_1,\cdots,\alpha_n $; choose each $ \varepsilon_i \leqslant 1 $. Then define
	\begin{equation}
		\varepsilon=\min\{\varepsilon_i/i|i=\alpha_1,\cdots,\alpha_n \} \nonumber
	\end{equation}
	We assert that
	\begin{equation}
		\vec{x}\in B_D(\vec{x},\varepsilon)\subset U \nonumber
	\end{equation}
	Let $ \vec{y} $ be a point of $ B_D(\vec{x},\varepsilon) $. Then for all $ i $,
	\begin{equation}
		\frac{\bar{d}(x_i,y_i)}{i}\leqslant D(\vec{x},\vec{y}) < \varepsilon \nonumber
	\end{equation}
	Now if $ i=\alpha_1,\cdots,\alpha_n $, then $ \varepsilon \leqslant \varepsilon_i/i $, so that $ \bar{d}(x_i,y_i)<\varepsilon_i \leqslant 1 $; it follows that $ |x_i-y_i|<\varepsilon_i $. Therefore, $ \vec{y}\in \prod U_i $, as desired.
\end{proof}
\begin{Proposition}
	If $ A $ is a subspace of the topological space $ X $ and $ d $ is a metric for $ X $, then the restriction of $ d $ to $ A \times A $ is a metric for the topology of $ A $.
\end{Proposition}
The Hausdorff axiom is satisfied by every metric topology. If $ x $ and $ y $ are distinct points of the metric space $ (X,d) $, we let $ \varepsilon = \frac{1}{2}d(x,y) $; then the triangle inequality implies that $ B_d(x,\varepsilon) $ and $ B_d(y,\varepsilon) $ are disjoint. \newpara
In general, countable products of metrizable spaces are metrizable;
\begin{proof}
	\exercise
\end{proof}
\begin{Lemma}[The Sequence Lemma]
	Let $ X $ be a topological space; let $ A \subset X $. If there is a sequence of points of $ A $ converging to $ x $, then $ x \in \bar{A} $; the converse holds if $ X $ is metrizable.
\end{Lemma}
\begin{proof}
	Suppose that $ x_n \to x $, where $ x_n \in A $. Then every neighborhood of $ x $ contains a point of $ A $, so $ x \in \bar{A} $. Conversely, suppose that $ X $ is metrizable and $ x \in \bar{A} $. Let $ d $ be a metric for the topology of $ X $. For each positive integer $ n $, take the neighborhood $ B_d(x,1/n) $ of $ x $, and choose $ x_n $ to be a point of its intersection with $ A $. Then any open set $ U $ containing $ x $ contains an $ \varepsilon $-ball centered at $ x $; if we choose $ N $ so that $ 1/N<\varepsilon $, then $ U $ contains $ x_i $ for all $ i\geqslant N $.
\end{proof}
\begin{Theorem}
	Let $ f:X \to Y $. If the function $ f $ is continuous, then for every convergent sequence $ x_n \to x $ in $ X $, the sequence $ f(x_n) $ converges to $ f(x) $. The converse holds if $ X $ is metrizable.
\end{Theorem}
\begin{proof}
	Assume that $ f $ is continuous. Given $ x_n \to x $, we wish to show that $ f(x_n)\to f(x) $. Let $ V $ be a neighborhood of $ f(x) $. Then $ f^{-1}(V) $ is a neighborhood of $ x $, and so there is an $ N $ such that $ x_n \in f^{-1}(V) $ for $ n \geqslant N $. Then $ f(x_n)\in V $ for $ n \geqslant N $. \newpara
	To prove the converse, assume that the convergent sequence condition is satisfied. Let $ A $ be a subset of $ X $; we show that $ f(\bar{A})\subset \overline{f(A)} $. If $ x \in \bar{A} $, then there is a sequence $ x_n $ of points of $ A $ converging to $ x $. By assumption the sequence $ f(x_n) $ converges to $ f(x) $. Since $ f(x_n)\in f(A) $, the preceding lemma implies that $ f(x)\in \overline{f(A)} $, and hence $ f(\bar{A})\subset \overline{f(A)} $, as desired.
\end{proof}
\begin{Definition}
	Let $ f_n:X \to Y $ be a sequence of functions from the set $ X $ to the metric space $ Y $. Let $ d $ be the metric for $ Y $. We say that the sequence $ (f_n) $ \textit{converges uniformly} to the function $ f:X \to Y $ if given $ \varepsilon>0 $, there exists and integer $ N $ such that
	\begin{equation}
		d(f_n(x),f(x))<\varepsilon \nonumber
	\end{equation}
	for all $ n>N $ and all $ x \in X $.
\end{Definition}
\begin{Theorem}[Uniform Limit Theorem]
	Let $ f_n:X \to Y $ be a sequence of continuous functions from the topological space $ X $ to the metric space $ Y $. If $ (f_n) $ converges uniformly to $ f $, then $ f $ is continuous.
\end{Theorem}
\begin{proof}
	Let $ V $ be open in $ Y $; let $ x_0 $ be a point of $ f^{-1}(V) $. We wish to find a neighborhood $ U $ of $ x_0 $ such that $ f(U)\subset V $. \newpara
	Let $ y_0=f(x_0) $. First choose $ \varepsilon $ so that the $ \varepsilon $-ball $ B(y_0,\varepsilon) $ is contained in $ V $. Then, using uniform convergence, choose $ N $ so that for all $ n \geqslant N $ and all $ x \in X $,
	\begin{equation}
		d(f_n(x),f(x))<\varepsilon/3 \nonumber
	\end{equation}
	Finally, using continuity of $ f_N $, choose a neighborhood $ U $ of $ x_0 $ such that $ f_N $ carries $ U $ into the $ \varepsilon/3 $-ball in $ Y $ centered at $ f_N(x_0) $. \newpara
	We claim that $ f $ carries $ U $ into $ B(y_0,\varepsilon) $ and hence into $ V $, as desired. For this purpose, note that if $ x \in U $, then
	\begin{align}
		d(f(x),f_N(x))&<\varepsilon/3\quad \text{by choice of }N \nonumber\\
		d(f_N(x),f_N(x_0))&<\varepsilon/3\quad \text{by choice of }U \nonumber\\
		d(f_N(x_0),f(x_0))&<\varepsilon/3\quad \text{by choice of }N \nonumber
	\end{align}
	Adding and using the triangle inequality, we see that $ d(f(x),f(x_0))<\varepsilon $, as desired.
\end{proof}
\begin{Theorem}
	Let $ \Real^X $ denote the space of all functions $ f:X \to \Real $ and let $ \bar{\rho} $ denote the uniform metric. A sequence of functions $ f_n:X \to \Real $ converges uniformly to $ f $ iff the sequence $ (f_n) $ converges to $ f $ when they are considered as elements of the metric space $ (\Real^X,\bar{\rho}) $.
\end{Theorem}
\begin{proof}
	\exercise
\end{proof}
\begin{Example}
	Let's show that $ \Real^\omega $ in the box topology is not metrizable. \newpara
	We shall show that the sequence lemma does not hold for $ \Real^\omega $. Let $ A $ be the subset of $ \Real^\omega $ consisting of those points all of whose coordinates are positive.
	\begin{equation}
		A=\{(x_1,x_2,\cdots)| \forall i\in \Zahlen_+(x_i>0) \} \nonumber
	\end{equation}
	Let $ \vec{0} $ be the origin in $ \Real^\omega $, or the point $ (0,0,\cdots) $. In the box topology, $ \vec{0} $ belongs to $ \bar{A} $, for any basis element containing $ \vec{0} $ intersects $ A $.\\
	However, we assert that there is no sequence of points of $ A $ converging to $ \vec{0} $. For let $ (\vec{a}_n) $ be a sequence of points of $ A $, where
	\begin{equation}
		\vec{a}_n=(x_{1n},x_{2n},\cdots,x_{in},\cdots) \nonumber
	\end{equation}
	Every coordinate $ x_{in} $ is positive, so we can construct a basis element $ B^\prime $ for the box topology on $ \Real $ by setting
	\begin{equation}
		B^\prime = (-x_{1n},x_{1n})\times (-x_{2n},x_{2n})\times \cdots \nonumber
	\end{equation}
	Then $ B^\prime $ contains the origin $ \vec{0} $, but it contains no member of the sequence $ (\vec{a}_n) $; the point $ \vec{a}_n $ cannot belong to $ B^\prime $ because its $ i $th coordinate $ x_{in} $ does not belong to the interval $ (-x_{in},x_{in}) $. Hence the sequence $ (\vec{a}_n) $ cannot converge to $ \vec{0} $ in the box topology.
\end{Example}
\begin{Example}
	We show that an uncountable product of $ \Real $ with itself is not metrizable. \newpara
	Let $ J $ be an uncountable index set; we show that $ \Real^J $ does not satisfy the sequence lemma (in the product topology).\\
	Let $ A $ be the subset of $ \Real^J $ consisting of all points $ (x_\alpha) $ such that $ x_\alpha =1 $ for all but finitely many values of $ \alpha $. Let $ \vec{0} $ be the "origin" in $ \Real^J $. \\
	We assert that $ \vec{0} $ belongs to the closure of $ A $. Let $ \prod U_\alpha $ be a basis element containing $ \vec{0} $. Then $ U_\alpha \neq \Real $ for only finitely many values of $ \alpha $, say for $ \alpha=\alpha_1,\cdots,\alpha_n $. Let $ (x_\alpha) $ be the point of $ A $ defined by letting $ x_\alpha = 0 $ for $ \alpha = \alpha_1,\cdots, \alpha_n $ and $ x_\alpha =1 $ for all other values of $ \alpha $; then $ (x_\alpha)\in A \cap \prod U_\alpha $, as desired. \newpara
	But there is no sequence of points of $ A $ converging to $ \vec{0} $. For let $ \vec{a}_n $ be a sequence of points of $ A $. Given $ n $, let $ J_n $ denote the subset of $ J $ consisting of those indices $ \alpha $ for which the $ \alpha $th coordinate of $ \vec{a}_n $ is different from $ 1 $. The union of all the sets $ J_n $ is a countable union of finite sets and therefore countable. Because $ J $ itself is uncountable, there is an index in $ J $, say $ \beta $, that does not lie in any of the sets $ J_n $. This means that for each of the points $ \vec{a}_n $, its $ \beta $th coordinate equals $ 1 $. \newpara
	Now let $ U_\beta $ be the open interval $ (-1,1) \subset \Real$, and let $ U $ be the open set $ \pi_\beta^{-1}(U_\beta) $ in $ \Real^J $. The set $ U $ is a neighborhood of $ \vec{0} $ that contains none of the points $ \vec{a}_n $; therefore, the sequence $ \vec{a}_n $ cannot converge to $ \vec{0} $.
\end{Example}
	
\section{The Quotient Topology}	
\begin{Definition}
	A mapping between two topological spaces is said to be closed(resp. opened) if for each closed(resp. opened) set in its domain the image of it is closed(resp. opened).
\end{Definition}
\begin{Definition}
	Let $ X $ and $ Y $ be topological spaces; let $ p:X \to Y $ be a surjective map. The map $ p $ is said to be a \textit{quotient map} provided a subset $ U $ of $ Y $ is open in $ Y $ iff $ p^{-1}(U) $ is open in $ X $. It follows that if $ p:X \to Y $ is a surjective continuous map that is either open or closed, then $ p $ is a quotient map. However, there are quotient maps that are neither open or closed.
\end{Definition}	
\begin{Definition}
	A subset $ C $ of $ X $ is \textit{saturated}(with respect to the surjective map $ p:X \to Y $) if $ C $ contains every set $ p^{-1}(\{y\}) $ that it intersects. Thus $ C $ is saturated if it equals the complete inverse image of a subset of $ Y $.
\end{Definition}
\begin{Example}
	Let $ X $ be the subspace $ [0,1]\cup [2,3] $ of $ \Real $, and let $ Y $ be the subspace $ [0,2] $ of $ \Real $. The map $ p:X \to Y $ defined by
	\begin{equation}
		p(x)=\begin{cases}
		x &\quad \text{for }x\in [0,1]\\
		x-1 &\quad \text{for }x\in [2,3]
		\end{cases}\nonumber
	\end{equation}
	is readily seen to be surjective, continuous, and closed. Therefore it is a quotient map. It is not, however, an open map; the image of the open set $ [0,1] $ of $ X $(its complement is the empty set; or think that $ [0,1] $ does not contain all of its boundary points in $ X$) is not open in $ Y $. \newpara
	Note that if $ A $ is the subspace $ [0,1)\cup [2,3] $ of $ X $, then the map $ q:A \to Y $ obtained by restricting $ p $ is continuous and surjective, but it is not a quotient map. For the set $ [2,3] $ is open in $ A $ and is saturated with respect to $ q $(it equals the inverse of $ [1,2] $ in $ Y $), but its image is not open in $ Y $.
\end{Example}
\begin{Definition}
	If $ X $ is a space and $ A $ is a set and if $ p:X \to A $ is a surjective map, then there exists exactly one topology $ \mathscr{T} $ on $ A $ relative to which $ p $ is a quotient map; it is called the \textit{quotient topology} induced by $ p $.
\end{Definition}
The topology $ \mathscr{T} $ is defined by letting it consist of those subsets $ U $ of $ A $ such that $ p^{-1}(U) $ is open in $ X $. Now we check the condition for the forming the topology $ \mathscr{T} $. The sets $ \varnothing $ and $ A $ and open because $ p^{-1} (\varnothing)=\varnothing$ and $ p^{-1}(A)=X $. The other two conditions follow from the equations
\begin{equation}
	p^{-1}(\bigcup_{\alpha \in J}U_\alpha) = \bigcup_{\alpha \in J}p^{-1}(U_\alpha),\qquad p^{-1}(\bigcap_{i=1}^{n}U_i)=\bigcap_{i=1}^{n}p^{-1}(U_i) \nonumber
\end{equation}
\begin{Definition}
	Let $ X $ be a topological space, and let $ X^\ast $ be a partition of $ X $ into disjoint subsets whose union is $ X $. Let $ p:X \to X^\ast $ be the surjective map that carries each point of $ X $ to the element of $ X^\ast $ containing it. In the quotient topology induced by $ p $, the space $ X^\ast $ is called a \textit{quotient space} of $ X $.
\end{Definition}
Given $ X^\ast $, there is an equivalence relation on $ X $ of which the elements of $ X^\ast $ are the equivalence classes. One can think of $ X^\ast $ as having been obtained by "identifying" each pair of equivalent points. For this reason, the quotient space $ X^\ast $ is often called an \textit{identification space}, or a \textit{decomposition space}, of the space $ X $. \newpara
We can describe the topology of $ X^\ast $ in another way. A subset $ U $ of $ X^\ast $ is a collection of equivalence classes, and the set $ p^{-1}(U) $ is just the union of equivalence classes belonging to $ U $. Thus the typical open set of $ X^\ast $ is a collection of equivalence classes whose union is an open set of $ X $.
\begin{Example}
	Let $ X $ be the closed unit ball
	\begin{equation}
		\{x \times y|x^2+y^2\leqslant 1 \} \nonumber
	\end{equation}
	in $ \Real^2 $, and let $ X^\ast $ be the partition of $ X $ consisting of all the one-point sets $ \{x \times y \} $ for which $ x^2+y^2<1 $, along with the set $ S^1=\{x\times y|x^2+y^2=1 \} $. One can show that $ X^\ast $ is homeomorphic with the subspace of $ \Real^3 $ called the unit 2-sphere, defined by
	\begin{equation}
		S^2=\{(x,y,z)|x^2+y^2+z^2=1 \} \nonumber
	\end{equation}
\end{Example}	
\begin{proof}
	We will prove a similar statement. I believe that this proof can be modified slightly to apply to the original problem.\newpara
	 "There is a equivalence relation on a $ n $-dimensional closed unit ball $ D^n $:
	\begin{equation}
		(x \sim y) \Leftrightarrow (x=y)\lor(\norm{x}=\norm{y}=1) \nonumber
	\end{equation}
	Prove that $ D^n/\sim $ is homeomorphic to $ S^n $, the unit sphere in $ \Real^n $."\newpara
	Define the function $ \theta:\Real^n \to S^n $
	\begin{equation}
		\theta(x_1, \ldots, x_n)=\bigg(\frac{S-1}{S+1}, \frac{2x_1}{S+1},\cdots,\frac{2x_n}{S+1}\bigg) \nonumber
	\end{equation}
	where $ S=\sum x_i^2 $. This map is also known as the \textit{stereographic projection}(more accurately, its inverse). Note that the image consists of all points on the sphere except $ (1,0,0,\cdots,0) $. Also it is injective.\newpara
	Now we define a mapping $ g:B^n \to \Real^n $ from the open disk to $ \Real^n $
	\begin{equation}
		g(x)=\begin{cases}
		\tan(\frac{\pi}{2}\norm{x})\frac{x}{\norm{x}}&\quad \text{if }x\neq 0 \\
		0 &\quad \text{if }x=0 
		\end{cases}\nonumber
	\end{equation}
	This a homeomorphism. It's inverse is given by scalar multiplication by $ \arctan $, which is $ x=\frac{2}{\pi}\arctan(\norm{y})\frac{y}{\norm{y}} $. Finally we define our function $ D^n \to S^n $
	\begin{equation}
		f(x)=\begin{cases}
		 \theta(g(x)) &\quad \text{if }  \norm{x}<1 \\
		 (1,0,\cdots,0) &\quad \text{otherwise}
		\end{cases}\nonumber
	\end{equation}
	In particular, $ f $ induces the mapping $ \bar{f}:B^n/\sim \to S^n $
	\begin{equation}
		\bar{f}([x])=f(x)\nonumber
	\end{equation}
	and $ \bar{f} $ is a homeomorphism. This mapping is closed, injective and surjective. Since $ B^n $ is compact, so is $ B^n / \sim $ and thus $ \bar{f} $ is also closed and therefore its inverse is continuous.
\end{proof}
\begin{Theorem}
	Let $ p:X \to Y $ be a quotient map; let $ A $ be a subspace of $ X $ that is saturated with respect to $ p $; let $ q:A \to p(A) $ be the map obtained by restricting $ p $.
	\begin{enumerate}
		\item If $ A $ is either open or closed in $ X $, then $ q $ is a quotient map.
		\item If $ p $ is either an open map or a closed map, then $ q $ is a quotient map.
	\end{enumerate}
\end{Theorem}
\begin{proof}
	We first verify the following two equations:
	\begin{align}
		q^{-1}(V)&=p^{-1}(V) \qquad &\text{if }V\subset p(A)\nonumber\\
		p(U \cap A)&=p(U)\cap p(A)\qquad&\text{if }U \subset X\nonumber
	\end{align}
	Then suppose $ A $ is open or $ p $ is open. Given the subset $ V $ of $ p(A) $, we assume that $ q^{-1}(V) $ is open in $ A $ and show that $ V $ is open in $ p(A) $. Finally, replace all "opened" with "closed" to complete the other part of the proof.
\end{proof}
\begin{Theorem}
	Let $ p:X \to Y $ be a quotient map. Let $ Z $ be a space and let $ g:X \to Z $ be a map that is constant on each set $ p^{-1}(\{y\}) $, for $ y \in Y $. Then $ g $ induces a map $ f:Y \to Z $ such that $ f \circ p = g $. The induced map $ f $ is continuous iff $ g $ is continuous; $ f $ is a quotient map iff $ g $ is a quotient map.
	\begin{equation}
			\begin{tikzcd}[column sep = large, row sep=large]
		X \arrow[dr,"g"] \arrow[d,"p"]\\
		Y \arrow[r,dotted,"f"] &Z 
		\end{tikzcd} \nonumber
	\end{equation}
\end{Theorem}
\begin{proof}
	For each $ y \in Y $, the set $ g(p^{-1}(\{y \})) $ is a one-point set in $ Z $(since $ g $ is constant on it). If we let $ f(y) $ denote this point, then we have defined a map $ f:Y \to Z $ such that for each $ x\in X $, $ f(p(x))=g(x) $. If $ f $ is continuous, then $ g=f \circ p $ is continuous. Conversely, suppose $ g $ is continuous. Given an open set $ V $ of $ Z $, $ g^{-1}(V) $ is open in $ X $. But $ g^{-1}(V)=p^{-1}(f^{-1}(V)) $; because $ p $ is a quotient map, it follows that $ f^{-1}(V) $ is open in $ Y $. Hence $ f $ is continuous. \newpara
	If $ f $ is a quotient map, then $ g $ is the composite of two quotient maps and is thus a quotient map. Conversely, suppose that $ g $ is a quotient map. Since $ g $ is surjective, so is $ f $. Let $ V $ be a subset of $ Z $; we show that $ V $ is open in $ Z $ if $ f^{-1}(V) $ is open in $ Y $. Now the set $ p^{-1}(f^{-1}(V)) $ is open in $ X $ because $ p $ is continuous. Since this set equals $ g^{-1}(V) $, the latter is open in $ X $. Then because $ g $ is a quotient map, $ V $ is open in $ Z $.
\end{proof}	
\begin{Corollary}
	Let $ g:X \to Z $ be a surjective continuous map. Let $ X^\ast $ be the following collection of subsets of $ X $:
	\begin{equation}
		X^\ast = \{g^{-1}(\{z\})|z \in Z \} \nonumber
	\end{equation}
	Give $ X^\ast $ the quotient topology.
	\begin{enumerate}
		\item The map $ g $ induces a bijective continuous map $ f:X^\ast \to Z $, which is a homeomorphism iff $ g $ is a quotient map.\begin{equation}
			\begin{tikzcd}[column sep = large, row sep=large]
X \arrow[dr,"g"] \arrow[d,"p"]\\
X^\ast \arrow[r,"f"] &Z 
\end{tikzcd} \nonumber
		\end{equation}
		\item If $ Z $ is Hausdorff, so is $ X^\ast $.
	\end{enumerate}
\end{Corollary}
\begin{proof}
	By the preceding theorem, $ g $ induces a continuous map $ f:X^\ast \to Z $; it is clear that $ f $ is bijective. Suppose that $ f $ is a homeomorphism. Then both $ f $ and the projection map $ p:X \to X^\ast $ are quotient maps, so that their composite $ q $ is a quotient map. Conversely, suppose that $ g $ is a quotient map. Then it follows from the preceding theorem that $ f $ is a quotient map. Being bijective, $ f $ is thus a hoemomorphism. \newpara
	Suppose $ Z $ is Hausdorff. Given distinct points of $ X^\ast $, their images under $ f $ are distinct and thus possess disjoint neighborhoods $ U $ and $ V $. Then $ f^{-1}(U) $ and $ f^{-1}(V) $ are disjoint neighborhoods of the two given points of $ X^\ast $.
\end{proof}

\chapter{Connectedness and Compactness}
\section{Connected Spaces}
\begin{Definition}
	Let $ X $ be a topological space. A \textit{separation} of $ X $ is a pair $ U ,V$ of disjoint nonempty open subsets of $ X $ whose union is $ X $. The space $ X $ is said to be \textit{connected} if there does not exist a separation of $ X $.
\end{Definition}
Equivalently, a space $ X $ is connected iff the only subsets of $ X $ that are both open and closed in $ X $ are the empty set and $ X $ itself.
\begin{Lemma}
	If $ Y $ is a subspace of $ X $, a separation of $ Y $ is a pair of disjoint nonempty sets $ A $ and $ B $ whose union is $ Y $, neither of which contains a limit point of the other. The space $ Y $ is connected if there exists no separation of $ Y $.
\end{Lemma}	
\begin{proof}
	Suppose $ A $ and $ B $ form a separation of $ Y $. Then $ A $ is both open and closed in $ Y $. The closure of $ A $ in $ Y $ is the set $ \bar{A} \cap Y$. Since $ A $ is closed, $ A=\bar{A} \cap Y $. Since $ \bar{A} $ is the union of $ A $ and its limit points, $ B $ contains no limit points of $ A $. A similar argument shows that $ A $ contains no limit points of $ B $. \newpara
	Conversely, suppose that $ A $ and $ B $ are disjoint nonempty sets whose union is $ Y $, neither of which contains a limit point of the other. Then $ \bar{A}\cap B = \varnothing $ and $ A \cap \bar{B}=\varnothing $, and we conclude that $ \bar{A}\cap Y = A $ and $ \bar{B}\cap Y = B $. Thus both $ A $ and $ B $ are closed in $ Y $, and they are both open because they are each other's complement in $ Y $.
\end{proof}
\begin{Lemma}
	If the sets $ C $ and $ D $ form a separation of $ X $, and if $ Y $ is a connected subspace of $ X $, then $ Y $ lies entirely within either $ C $ or $ D $.
\end{Lemma}	
\begin{proof}
	Since $ C $ and $ D $ are both open in $ X $, the sets $ C \cap Y $ and $ D \cap Y $ are open in $ Y $. These two sets are disjoint and their union is $ Y $; if they were both nonempty, they would constitute a separation of $ Y $. Therefore, one of them is empty. Hence $ Y $ must lie entirely in $ C $ or in $ D $.
\end{proof}
\begin{Theorem}
	The union of a collection of connected subspaces of $ X $ that have a point in common is connected.
\end{Theorem}
\begin{proof}
	Let $ \{A_\alpha \} $ be a collection of connected subspaces of a space $ X $; let $ p $ be a point of $ \bigcap A_\alpha $. We prove that the space $ Y= \bigcup A_\alpha $ is connected. Suppose that $ Y= C \cup D $ is a separation of $ Y $. The point $ p $ is in one of the sets $ C $ or $ D $; suppose $ p \in C $. Since $ A_\alpha $ is connected, it must lie entirely in either $ C $ or $ D $, and it cannot lie in $ D $ because it contains the point $ p $ of $ C $. Hence $ A_\alpha \subset C $ for every $ \alpha $, so that $ \bigcup A_\alpha \subset C $, contradicting the fact that $ D $ is nonempty.
\end{proof}	
\begin{Theorem}
	Let $ A $ be a connected subspace of $ X $. If $ A \subset B \subset \bar{A} $, then $ B $ is also connected. Said differently: If $ B $ is formed by adjoining to the connected subspace $ A $ some or all of its limit points, then $ B $ is connected.
\end{Theorem}
\begin{proof}
	Let $ A $ be connected and $ A \subset B \subset \bar{A} $. Suppose that $ B=C \cup D $ is a separation of $ B $. Then the set $ A $ must lie entirely in $ C $ or in $ D $; suppose that $ A \subset C $. Then $ \bar{A}\subset \bar{C} $; since $ \bar{C} $ and $ D $ are disjoint, $ B $ cannot intersect $ D $. This contradicts the fact that $ D $ is a nonempty subset of $ B $.
\end{proof}
\begin{Theorem}
	The image of a connected space under a continuous map is connected.
\end{Theorem}
\begin{proof}
	Let $ f:X \to Y $ be a continuous map; let $ X $ be connected. We wish to prove the image space $ Z=f(X) $ is connected. Since the map obtained from $ f $ by restricting its range to the space $ Z $ is also continuous, it suffices to consider the case of a continuous surjective map
	\begin{equation}
		g:X \to Z \nonumber
	\end{equation}
	Suppose that $ Z = A \cup B $ is a separation of $ Z $ into two disjoint nonempty sets open in $ Z $. Then $ g^{-1}(A) $ and $ g^{-1}(B) $ are disjoint sets whose union is $ X $; they are open in $ X $ because $ g $ is continuous, and nonempty because $ g $ is surjective. Therefore, they form a separation of $ X $, contradicting the assumption that $ X $ is connected.
\end{proof}
\begin{Theorem}
	We prove the theorem first for the product of two connected spaces $ X $ and $ Y $. Choose a base point $ a \times b $ in the product $ X \times Y $. Now $ X \times b $ is connected, being homeomorphic with $ X $, and $ x \times Y $ is connected, being homeomorphic with $ Y $. As a result, the space
	\begin{equation}
		T_x = (X \times b)\cup (x \times Y) \nonumber
	\end{equation}
	is connected, being the union of two connected spaces that have the point $ x \times b $ in common. Now from the union $ \bigcup_{x \in X}T_x $ of all these spaces. This union is connected because it is the union of a collection of connected spaces that have the point $ a \times b $ in common, Since this union equals $ X \times Y $, the space $ X \times Y $ is connected. The proof for any finite product of connected spaces follows by induction, using the fact that $ X_1 \times \cdots \times X_n $ is homeomorphic with $ (X_1 \times \cdots \times X_{n-1})\times X_n $.
\end{Theorem}
\begin{Example}
	Consider the cartesian product $ \Real^\omega $ in the box topology. We can write $ \Real^\omega $ as the union of the set $ A $ consisting of all bounded sequences of real numbers, and the set $ B $ of all unbounded sequences. These sets are disjoint, and each is open in the box topology. For if $ \vec{a} $ is a point of $ \Real^\omega $, the open set
	\begin{equation}
		U=(a_1-1,a_1+1)\times (a_2-1,a_2+1)\times \cdots \nonumber
	\end{equation}
	consists entirely of bounded sequences if $ \vec{a} $ is bounded, and of unbounded sequences if $ \vec{a} $ is unbounded. Thus, even though $ \Real $ is connected, $ \Real^\omega $ is not connected in the box topology.
\end{Example}
\begin{Example}
	Consider $ \Real^\omega $ in the product topology. Assuming that $ \Real $ is connected, we show that $ \Real^\omega $ is connected. Let $ \tilde{\Real^n} $  denote the subspace of $ \Real^\omega $ consisting of all sequences $ \vec{x}=(x_1,\cdots) $ such that $ x_i=0 $ for $ i>n $. The space $ \tilde{\Real^n} $ is clearly homeomorphic to $ \Real^n $, so that it is connected. It follows that the space $ \Real^\infty $ that is the union of the spaces $ \tilde{\Real^n} $ is connected, for these spaces have the point $ \vec{0}=(0,0,\cdots) $ in common. We show that the closure of $ \Real^\infty $ equals all of $ \Real^\omega $, for which it follows that $ \Real^\omega $ is connected as well. \newpara
	Let $ \vec{a}=(a_1,a_2,\cdots) $ be a point of $ \Real^\omega $. Let $ U=\prod U_i $ be a basis element for the product topology that contains $ \vec{a} $. We show that $ U $ intersects $ \Real^n $. There is an integer $ N $ such that $ U_i = \Real $ for $ i>N $. Then the point
	\begin{equation}
		\vec{x}=(a_1,\cdots,a_n,0,0,\cdots) \nonumber
	\end{equation}
	of $ \Real^\infty $ belongs to $ U $, since $ a_i \in U_i $ for all $ i $, and $ 0 \in U_i $ for $ i>N $.
\end{Example}
\section{Connected Subspaces of the Real Line}	
\begin{Definition}
	A simply ordered set $ L $ having more than one element is called a \textit{linear continuum} if the following hold:
	\begin{enumerate}
		\item $ L $ has the least upper bound property.
		\item If $x<y  $, there exists $ z $ such that $ x<z<y $.
	\end{enumerate}
\end{Definition}
\begin{Theorem}
	If $ L $ is a linear continuum in the order topology, then $ L $ is connected, and so are intervals and rays in $ L $.
\end{Theorem}
\begin{proof}
	Suppose $ Y $ is a convex subspace of $ L $ and is the union of the disjoint nonempty sets $ A $ and $ B $, each of which is open in $ Y $. Choose $ a \in A $ and $ b \in B $; suppose for convenience that $ a<b $. The interval $ [a,b] $ of points of $ L $ is contained in $ Y $. Hence $ [a,b] $ is the union of the disjoint sets
	\begin{equation}
		A_0 = A \cap [a,b]\quad\text{and}\quad B_0 = B\cap [a,b] \nonumber
	\end{equation}
	each of which is open in $ [a,b] $ in the subspace topology, which is the same as the order topology. The sets $ A_0 $ and $ B_0 $ are nonempty because $ a \in A_0 $ and $ b \in B_0 $. Thus, $ A_0 $ and $ B_0 $ constitute a separation of $ [a,b] $. Let $ c= \sup A_0 $. \newpara
	Suppose that $ c \in B_0 $. Then $ c \neq a $, so either $ c=b $ or $ a<c<b $. In either case, it follows from the fact that $ B_0 $ is open in $ [a,b] $ that there is some interval of the form $ (d,c] $ contained in $ B_0 $. If $ c=b $, we have a contradiction at once, for $ d $ is a smaller upper bound on $ A_0 $ than $ c $. If $ c<b $, we note that $ (c,b] $ does not intersect $ A_0 $. Then
	\begin{equation}
		(d,b] = (d,c] \cup (c,b] \nonumber
	\end{equation}
	does not intersect $ A_0 $. Again, $ d $ is a smaller upper bound on $ A_0 $ than $ c $, contrary to construction. \newpara
	The case when $ c $ belongs to $ A_0 $ can be proved similarly.
\end{proof}
\begin{Corollary}
	The real line $ \Real $ is connected and so are intervals and rays in $ \Real $.
\end{Corollary}
\begin{Theorem}[Intermediate Value Theorem]
	Let $ f:X \to Y $ be a continuous map, where $ X $ is a connected space and $ Y $ is an ordered set in the order topology. If $ a $ and $ b $ are two points of $ X $ and if $ r $ is a point of $ Y $ lying between $ f(a) $ and $ f(b) $, then there exists a point $ c $ of $ X $ such that $ f(c)=r $.
\end{Theorem}
\begin{proof}
	Assume the hypotheses of the theorem. The sets
	\begin{equation}
		A=f(X)\cap (-\infty,r)\quad \text{and}\quad B=f(X)\cap (r,+\infty) \nonumber
	\end{equation}
	are disjoint and nonempty. Each is open in $ f(X) $ because they are the intersection of an open ray in $ Y $ with $ f(X) $. If there were no point $ c $ of $ X $ such that $ f(c)=r $, then $ f(X) $ would be the union of the sets $ A $ and $ B $, and this contradicts to the fact that the image of a connected space under a continuous map is connected.
\end{proof}
\begin{Definition}
	Given points $ x $ and $ y $ of the space $ X $, a \textit{path} in $ X $ from $ x $ to $ y $ is a continuous map $ f:[a,b]\to X $ of some closed interval in the real line into $ X $, such that $ f(a)=x $ and $ f(b)=y $. A space $ X $ is said to be \textit{path connected} if every pair of points of $ X $ can be joined by a path in $ X $.
\end{Definition}
Every path-connected space $ X $ is connected, but the converse does not hold.
\begin{Example}
	Let $ S $ denote the following subset of the plane
	\begin{equation}
		S=\{x \times \sin(1/x)|0<x \leqslant 1 \} \nonumber
	\end{equation}
	Because $ S $ is the image of the connected set $ (0,1] $ under a continuous map, $ S $ is connected. Therefore, its closure $ \bar{S} $ in $ \Real^2 $ is also connected. The set $ \bar{S} $ is called the \textit{topologist's sine curve}. However, $ \bar{S}  $ is not path connected. \newpara
	Suppose there is a path $ f:[a,c]\to \bar{S} $ beginning at the origin and ending at a point of $ S $. The set of those $ t $ for which $ f(t)\in 0 \times [-1,1] $ is closed, so it has a largest element $ b $. Then $ f:[b,c]\to \bar{S} $ is a path that maps $ b $ into the vertical interval $ 0 \times [-1,1] $ and maps the other points of $ [b,c] $ to points of $ S $. \newpara
	Replace $ [b,c] $ by $ [0,1] $ for convenience; let $ f(t)=(x(t),y(t)) $. Then $ x(0)=0 $, while $ x(t)>0 $ and $ y(t)=\sin(1/x) $ for $ t>0 $. We show there is a sequence of points $ t_n \to 0 $ such that $ y(t_n)=(-1)^n $. Then the sequence $ y(t_n) $ does not converge, contradicting continuity of $ f $.\newpara
	To find $ t_n $, we proceed as follows: Given $ n $, choose $ u $ with $ 0<u<x(1/n) $ such that $ \sin(1/u)=(-1)^n $. Then use the intermediate value theorem to find $ t_n $ with $ 0<t_n<1/n $ such that $ x(t_n)=u $.
\end{Example}	
\section{Components and Local Connectedness}	
\begin{Definition}
	Given $ X $, define an equivalence relation on $ X $ by setting $ x \sim y $ if there is a connected subspace of $ X $ containing both $ x $ and $ y $. The equivalence classes are called the \textit{components}(or the \textit{connected components}) of $ X $.
\end{Definition}	
\begin{Theorem}
	The components of $ X $ are connected disjoint subspaces of $ X $ whose union is $ X $, such that each nonempty connected subspace of $ X $ intersects only one of them.
\end{Theorem}
\begin{Definition}
	We define another equivalence relation on the space $ X $ by defining $ x \sim y $ if there is a path in $ X $ from $ x $ to $ y $. The equivalence classes are called the \textit{path components} of $ X $.
\end{Definition}
\begin{Theorem}
	The path components of $ X $ are path-connected disjoint subspaces of $ X $ whose union is $ X $, such that each nonempty path-connected subspace of $ X $ intersects only one of them.
\end{Theorem}	
\begin{Example}
	If $ \Quoziente $ is the subspace of $ \Real $ consisting of the rational numbers, then each component of $ \Quoziente $ consists of a single point. None of the components of $ \Quoziente $ are open in $ \Quoziente $.
\end{Example}
\begin{Definition}
	A space $ X $ is said to ve \textit{locally connected at $ x $} if for every neighborhood $ U $ of $ x $, there is a connected neighborhood $ V $ of $ x $ contained in $ U $. If $ X $ is locally connected at each of its points, it is said simply to be \textit{locally connected}. Similarly, a space $ X $ is said to be \textit{locally path connected at $ x $} if for every neighborhood $ U $ of $ x $, there is a path-connected neighborhood $ V $ of $ x $ contained in $ U $. If $ X $ is locally path connected at each of its points, then it is said to be \textit{locally path connected}.
\end{Definition}
\begin{Theorem}
	A space $ X $ is locally connected iff for every open set $ U $ of $ X $, each component of $ U $ is open in $ X $.
\end{Theorem}
\begin{proof}
	Suppose that $ X $ is locally connected; let $ U $ be an open set in $ X $; let $ C $ be a component of $ U $. If $ x $ is a point of $ C $, we can choose a connected neighborhood $ V $ of $ x $ such that $ V \subset U $. Since $ V $ is connected, it must lie entirely in the component $ C $ of $ U $. Therefore, $ C $ is open in $ X $. \newpara
	Conversely, suppose that components of open sets in $ X $ are open. Given a point $ x $ of $ X $ and a neighborhood $ U $ of $ x $, let $ C $ be the component of $ U $ containing $ x $. Now $ C $ is connected; since it is open in $ X $ by hypothesis, $ X $ is locally connected at $ x $.
\end{proof}
\begin{Theorem}
	A space $ X $ is locally path connected iff for every open set $ U $ of $ X $, each path component of $ U $ is open in $ X $.
\end{Theorem}
\begin{Theorem}
	If $ X $ is a topological space, each path component of $ X $ lies in a component of $ X $. If $ X $ is locally path connected, then the components and the path components of $ X $ are the same.
\end{Theorem}
\begin{proof}
	Let $ C $ be a component of $ X $; let $ x $ be a point of $ C $; let $ P $ be the path component of $ X $ containing $ x $. Since $ P $ is connected, $ P \subset C $. We wish to show that if $ X $ is locally path connected, $ P=C $. Suppose that $ P \subsetneq C$. Let $ Q $ denote the union of all the path components of $ X $ that are different from $ P $ and intersect $ C $; each of them necessarily lies in $ C $, so that
	\begin{equation}
		C = P \cup Q \nonumber
	\end{equation}
	Because $ X $ is locally path connected, each path component of $ X $ is open in $ X $. Therefore, $ P $ and $ Q $ are open in $ X $, so they constitute a separation of $ C $. This contradicts the fact that $ C $ is connected.
\end{proof}

\section{Compact Spaces}
\begin{Definition}
	A space $ X $ is said to be \textit{compact} if every open covering $ \mathscr{A} $ of $ X $ contains a finite subcollection that also covers $ X $.
\end{Definition}
\begin{Definition}
	If $ Y $ is a subspace of $ X $, a collection $ \mathscr{A} $ of subsets of $ X $ is said to \textit{cover} $ Y $ if the union of its elements contains $ Y $.
\end{Definition}
\begin{Lemma}
	Let $ Y $ be a subspace of $ X $. Then $ Y $ is compact iff every covering of $ Y $ by sets open in $ X $ contains a finite subcollection covering $ Y $.
\end{Lemma}
\begin{Theorem}
	Every closed subspace of a compact space is compact.
\end{Theorem}
\begin{proof}
	Let $ Y $ be a closed subspace of the compact space $ X $. Given a covering $ \mathscr{A} $ of $ Y $ by sets open in $ X $, let us form an open covering $ \mathscr{B} $ of $ X $ by adjoining to $ \mathscr{A} $ the single open set $ X \setminus Y $, that is,
	\begin{equation}
		\mathscr{B}=\mathscr{A}\cup \{X \setminus Y \} \nonumber
	\end{equation}
	Some finite subcollection of $ \mathscr{B} $ covers $ X $. If this subcollection contains the set $ X \setminus Y $, discard $ X \setminus Y $; otherwise, leave the subcollection alone. The resulting collection is a finite subcollection of $ \mathscr{A} $ that covers $ Y $.
\end{proof}	
\begin{Theorem}
	Every compact subspace of a Hausdorff space is closed.
\end{Theorem}
\begin{proof}
	Let $ Y $ be a compact subspace of the Hausdorff space $ X $. We show that $ X \setminus Y $ is open. \newpara
	Let $ x_0 $ be a point of $ X \setminus Y $. We show there is a neighborhood of $ x_0 $ that is disjoint from $ Y $. For each point $ y $ of $ Y $, let us choose disjoint neighborhoods $ U_y $ and $ V_y $ of the points $ x_0 $ and $ y $, respectively. The collection $ \{V_y|y\in Y \} $ is a covering of $ Y $ by sets open in $ X $; therefore, finitely many of them $ V_{y1},\cdots,V_{yn} $ cover $ Y $. The open set
	\begin{equation}
		V=V_{y1}\cup \cdots \cup V_{yn} \nonumber
	\end{equation}
	contains $ Y $, and it is disjoint from the open set
	\begin{equation}
		U=U_{y1}\cap \cdots \cap U_{yn} \nonumber
	\end{equation}
	formed by taking the intersection of the corresponding neighborhoods of $ x_0 $. Then $ U $ is a neighborhood of $ x_0 $ disjoint from $ Y $, as desired.
\end{proof}
\begin{Theorem}
	The image of a compact space under a continuous map is compact.
\end{Theorem}
\begin{proof}
	Let $ f:X \to Y $ be continuous; let $ X $ be compact. Let $ \mathscr{A} $ be a covering of the set $ f(X) $ by sets open in $ Y $. The collection
	\begin{equation}
		\{f^{-1}(A)|A \in \mathscr{A} \} \nonumber
	\end{equation}
	is a collection of sets covering $ X $; these sets are open in $ X $ because $ f $ is continuous. Hence finitely many of them, say
	\begin{equation}
		f^{-1}(A_1),\cdots,f^{-1}(A_n) \nonumber
	\end{equation}
	cover $ X $. Then the sets $ A_1,\cdots,A_n $ cover $ f(X) $.
\end{proof}
\begin{Theorem}
	Let $ f:X \to Y $ be a bijective continuous function. If $ X $ is compact and $ Y $ is Hausdorff, then $ f $ is a homeomorphism.
\end{Theorem}
\begin{proof}
	We shall prove that images of closed sets of $ X $ under $ f $ are closed in $ Y $; this will prove continuity of the map $ f^{-1} $. If $ A $ is closed in $ X $, then $ A $ is compact, and $ f(A) $ is compact. Since $ Y $ is Hausdorff, $ f(A) $ is closed in $ Y $.
\end{proof}
\begin{Lemma}[The Tube Lemma]
	Consider the product space $ X \times Y $, where $ Y $ is compact. If $ N $ is an open set of $ X \times Y $ containing the slice $ x_0 \times Y $ of $ X \times Y $, then $ N $ contains some \textit{tube} $ W \times Y $ about $ x_0 \times Y $, where $ W $ is a neighborhood of $ x_0 $ in $ X $.
\end{Lemma}
\begin{proof}
	Let us cover $ x_0 \times Y $ by basis elements $ U \times V $ for the topology of $ X \times Y $ lying in $ N $. The space $ x_0 \times Y $ is compact, being homeomorphic to $ Y $. Therefore, we can cover $ x_0 \times Y $ by finitely many such basis elements
	\begin{equation}
		U_1 \times V_1 ,\cdots ,U_n \times V_n \nonumber
	\end{equation}
	Now we assume that each of the basis elements $ U_i \times V_i $ actually intersects $ x_0 \times Y $, since otherwise that basis element would be superfluous; we could discard it from the finite collection and still have a covering of $ x_0 \times Y $. Define
	\begin{equation}
		W = U_1 \cap \cdots \cap U_n \nonumber
	\end{equation}
	The set $ W $ is open, and it contains $ x_0 $ because each set $ U_i \times V_i $ intersects $ x_0 \times Y $. \newpara
	We assert that the sets $ U_i \times V_i $, which were chosen to cover the slice $ x_0 \times Y $, actually cover the tube $ W \times Y $. Let $ x \times y $ be a point of $ W \times Y $. Consider the point $ x_0 \times y $ of the slice $ x_0 \times Y $ having the same $ y $-coordinate as this point. Now $ x_0 \times y $ belongs to $ U_i \times V_i $ for some $ i $, so that $ y \in V_i $. But $ x \in U_j $ for every $ j $ since $ x \in W $. Therefore, we have $ x \times y \in U_i \times V_i $, as desired.\newpara
	Since all the sets $ U_i \times V_i $ lie in $ N $, and since they cover $ W \times Y $, the tube $ W \times Y $ lies in $ N $ also.
\end{proof}
\begin{Theorem}
	The product of finitely many compact spaces in compact.
\end{Theorem}
\begin{proof}
	Let $ X $ and $ Y $ be compact spaces. Let $ \mathscr{A} $ be an open covering of $ X \times Y $. Given $ x_0 \in X $, the slice $ x_0 \times Y $ is compact and may therefore be covered by finitely many elements $ A_1,\cdots,A_m $ of $ \mathscr{A}  $. Their union $ N= A_1 \cup \cdots \cup A_m $ is an open set containing $ x_0 \times Y $; by the preceding lemma, the open set $ N $ contains a tube $ W \times Y $ about $ x_0 \times Y $, where $ W $ is open in $ X $. Then $ W \times Y $ is covered by finitely many elements $ A_1,\cdots,A_m $ of $ \mathscr{A} $. \newpara
	Thus, for each $ x \in X $, we can choose a neighborhood $ W_x $ of $ x $ such that the tube $ W_x \times Y $ can be covered by finitely many elements of $ \mathscr{A} $. The collection of all the neighborhoods $ W_x $ is an open covering of $ X $; therefore by compactness of $ X $, there exists a finite subcollection
	\begin{equation}
		\{W_1,\cdots,W_k \} \nonumber
	\end{equation}
	covering $ X $. The union of the tubes
	\begin{equation}
		W_1 \times Y,\cdots,W_k \times Y \nonumber
	\end{equation}
	is all of $ X \times Y $; since each may be covered by finitely many elements of $ \mathscr{A} $, so may $ X \times Y $ be covered. The theorem itself follows by induction.
\end{proof}
\begin{Definition}
	A collection $ \mathscr{C} $ of subsets of $ X $ is said to have te \textit{finite intersection property} if for every finite subcollection
	\begin{equation}
		\{C_1,\cdots,C_n \} \nonumber
	\end{equation}
	of $ \mathscr{C} $, the intersection $ C_1 \cap \cdots \cap C_n $ is nonempty.
\end{Definition}
\begin{Theorem}
	Let $ X $ be a topological space. Then $ X $ is compact iff for every collection $ \mathscr{C} $ of closed sets in $ X $ having the finite intersection property, the intersection $ \bigcap_{C \in \mathscr{C}C} $ of all the elements of $ \mathscr{C} $ is nonempty.
\end{Theorem}
\begin{proof}
	Given a collection $ \mathscr{A} $ of subsets of $ X $, let
	\begin{equation}
		\mathscr{C}=\{X \setminus A|A \in \mathscr{A} \} \nonumber
	\end{equation}
	be the collection of their complements. Then the following statements hold:
	\begin{enumerate}
		\item $ \mathscr{A} $ is a collection of open sets iff $ \mathscr{C} $ is a collection of closed sets. 
		\item The collection $ \mathscr{A} $ covers $ X $ iff the intersection $ \bigcap_{C \in \mathscr{C}}C $ of all the elements of $ \mathscr{C} $ is empty.
		\item The finite subcollection $ \{A_1,\cdots,A_n \} $ of $ \mathscr{A} $ covers $ X $ iff the intersection of the corresponding elements $ C_i = X \setminus A_i $ of $ \mathscr{C} $ is empty.
	\end{enumerate}
The first statement is trivial, while the second and third follow from DeMorgan's law
\begin{equation}
	X \setminus(\bigcup_{\alpha \in J}A_\alpha) = \bigcap_{\alpha \in J}(X \setminus A_\alpha) \nonumber
\end{equation}
The contrapositive statement of compactness is that "Given any collection $ \mathscr{A} $ of open sets, if no finite subcollection of $ \mathscr{A} $ covers $ X $, then $ \mathscr{A} $ does not cover $ X $." Letting $ \mathscr{C} $ be, as earlier, the collection $ \{X\setminus A|A \in \mathscr{A} \} $ and applying $ (1)-(3) $, we see that this statement is equivalent to the condition of our theorem.
\end{proof}
\section{Compact Subspaces of the Real Line}
\begin{Theorem}
	Let $ X $ be a simply ordered set having the least upper bound property. In the order topology, each closed interval in $ X $ is compact.
\end{Theorem}
\begin{proof}
	Given $ a<b $, let $ \mathscr{A} $ be a covering of $ [a,b] $ by sets open in $ [a,b] $ in the subspace topology. We wish to prove the existence of a finite subcollection of $ \mathscr{A} $ covering $ [a,b] $.
	\begin{enumerate}
		\item We prove that if $ x $ is a point of $ [a,b] $ different from $ b $, then there is a point $ y>x $ of $ [a,b] $ such that the interval $ [x,y] $ can be covered by at most two elements of $ \mathscr{A} $. If $ x$ has an immediate successor in $ X $, let $ y $ be this immediate successor. Then $ [x,y] $ consists of the two points $ x $ and $ y $, so that it can be covered by at most two elements of $ \mathscr{A} $. If $ x $ has no immediate successor in $ X $, choose an element $ A $ of $ \mathscr{A} $ containing $ x $. Since $ A $ is open and $ x \neq b $, there is an interval of the form $ [x,c) $ for some $ c $ in $ [a,b] $. Choose a point $ y $ in $ (x,c) $; then the interval $ [x,y] $ is covered by the single element $ A $ of $ \mathscr{A} $.
		\item Let $ C $ be the set of all points $ y>a $ of $ [a,b] $ such that the interval $ [a,y] $ can be covered by finitely many elements of $ \mathscr{A} $. Applying step $ 1 $ to the case $ x=a $, we see that $ C $ is not empty. Let $ c $ be the least upper bound of $ C $. Then $ a<c \leqslant b $.
		\item We show that $ c $ belongs to $ C $; that is, we show that the interval $ [a,c] $ can be covered by finitely many elements of $ \mathscr{A} $. Choose an element $ A $ of $ \mathscr{A} $ containing $ c $; since $ A $ is open, it contains an interval of the form $ (d,c] $ for some $ d $ in $ [a,b] $. If $ c $ is not in $ C $, there must be a point $ z $ of $ C $ lying in the interval $ (d,c) $, because otherwise $ d $ would be a smaller upper bound on $ C $ than $ c $. Since $ z $ is in $ C $, the interval $ [a,z] $ can be covered by finitely many, say $ n $, elements of $ \mathscr{A} $. Now $ [z,c] $ lies in the single element $ A $ of $ \mathscr{A} $(since $ A $ contains $ (d,c] $), hence $ [a,c]=[a,z]\cup[z,c] $ can be covered by $ n+1 $ elements of $ \mathscr{A} $. Thus $ c $ is in $ C $, contrary to assumption.
		\item We show that $ c=b $, and our theorem is proved. Suppose that $ c<b $. Applying step $ 1 $ to the case $ x=c $, we conclude that there exists a point $ y>c $ of $ [a,b] $ such that the interval $ [c,y] $ can be covered by finitely many elements of $ \mathscr{A} $. We proved in step $ 3 $ that $ c $ is in $ C $, so $ [a,c] $ can be covered by finitely many elements of $ \mathscr{A} $. Therefore, the interval $ [a,y]=[a,c]\cup[c,y] $ can also be covered by finitely many elements of $ \mathscr{A} $. This means that $ y $ is in $ C $, contradicting the fact that $ c $ is an upper bound on $ C $.
	\end{enumerate}
\end{proof}
\begin{Corollary}
	Every closed interval in $ \Real $ is compact.
\end{Corollary}

\begin{Theorem}
	A subspace $ A $ of $ \Real^n $ is compact iff it is closed and is bounded in the euclidean metric $ d $ or the square metric $ \rho $.
\end{Theorem}
\begin{proof}
	It will suffice to consider only the metric $ \rho $; the inequalities
	\begin{equation}
		\rho(x,y) \leqslant d(x,y)\leqslant \sqrt{n}\rho(x,y) \nonumber
	\end{equation}
	imply that $ A $ is bounded under $ d $ iff it is bounded under $ \rho $. \newpara
	Suppose that $ A $ is compact. Then it is closed. Consider the collection of open sets
	\begin{equation}
		\{B_\rho(\vec{0},m)|m \in \Zahlen_+ \} \nonumber
	\end{equation}
	whose union is all of $ \Real^n $. Some finite subcollection covers $ A $. It follows that $ A \subset B_{\rho}(\vec{0},M) $ for some $ M $. Therefore, for any two points $ x $ and $ y $ of $ A $, we have $ \rho(x,y)\leqslant 2M $. Thus $ A $ is bounded under $ \rho $. \newpara
	Conversely, suppose that $ A $ is closed and bounded under $ \rho $; suppose that $ \rho(x,y)\leqslant N $ for every pair $ x,y $ of points of $ A $. Choose a point $ x_0 $ of $ A $, and let $ \rho(x_0,\vec{0}) =b$ . The triangle inequality implies that $ \rho(x,\vec{0})\leqslant N+b $ for every $ x $ in $ A $. If $ P=N+b $, then $ A $ is a subset of the cube $ [-P,P]^n $, which is compact. Being closed, $ A $ is also compact.
\end{proof}
\begin{Theorem}[Extreme Value Theorem]
	Let $ f:X \to Y $ be continuous, where $ Y $ is an ordered set in the order topology. If $ X $ is compact, the there exist points $ c $ and $ d $ in $ X $ such that $ f(c)\leqslant f(x)\leqslant f(d) $ for every $ x \in X $.
\end{Theorem}	
\begin{proof}
	Since $ f $ is continuous and $ X $ is compact, $ A=f(X) $ is compact. We show that $ A $ has a largest element $ M $ and a smallest element $ m $. \newpara
	If $ A $ has no largest element, then the collection
	\begin{equation}
		\{(-\infty,a)|a \in A \} \nonumber
	\end{equation}
	forms a covering of $ A $. Since $ A $ is compact, some finite subcollection
	\begin{equation}
		\{(-\infty,a_1),\cdots,(-\infty,a_n) \} \nonumber
	\end{equation}
	covers $ A $. If $ a_i $ is the largest of the elements $ a_1,\cdots,a_n $, then $ a_i $ belongs to none of these sets, contrary to the fact that they cover $ A $. A similar argument shows that $ A $ has a smallest element.
\end{proof}
\begin{Definition}
	Let $ (X,d) $ be a metric space; let $ A $ be a nonempty subset of $ X $. For each $ x \in X $, we define the \textit{distance from $ x $ to $ A $} by the equation
	\begin{equation}
		d(x,A)=\inf\{d(x,a)|ain A \}  \nonumber
	\end{equation}
\end{Definition}
For a fixed $ A $, the function $ d(x,A) $ is a continuous function of $ x $.
\begin{Lemma}[The Lebesgue Number Lemma]
	Let $ \mathscr{A} $ be an open covering of the metric space $ (X,d) $. If $ X $ is compact, there is a $ \delta>0 $ such that for each subset of $ X $ having diameter less than $ \delta $, there exists an element of $ \mathscr{A} $ containing it.
\end{Lemma}
The number $ \delta $ is called a \textit{Lebesgue number} for the covering $ \mathscr{A} $.
\begin{proof}
	Let $ \mathscr{A} $ be an open covering of $ X $ and assume that $ X $ is not one of its elements.\newpara
	Choose a finite subcollection $ \{A_1,\cdots,A_n \} $ of $ \mathscr{A} $ that covers $ X $. For each $ i $, set $ C_i = X \setminus A_i $, and define $ f:X \to \Real $ by letting $ f(x) $ be the average of the numbers, that is
	\begin{equation}
		f(x)=\frac{1}{n}\sum_{i=1}^{n}d(x,C_i) \nonumber
	\end{equation}
	We show that $ f(x) >0$ for all $ x $. Given $ x \in X $, choose $ i $ so that $ x \in A_i $. Then choose $ \varepsilon $ so that the $ \varepsilon $-neighborhood of $ x $ lies in $ A_i $. Then $ d(x,C_i)\geqslant \varepsilon $, so that $ f(x)\geqslant \varepsilon/n $. Alternatively, suppose that $ f(x)=0 $ for some $ x $. Then it follows that $ x $ has to be belong to all $ C_i $. However, since $ A_i $ form a finite covering of $ X $, it turns out that the intersection of all $ C_i $ must be empty.\newpara
	Since $ f $ is continuous. it has a minimum value $ \delta $; we show that $ \delta $ is our required Lebesgue number. Let $ B $ be a subset of $ X $ of diameter less than $ \delta $. Choose a point $ x_0 \in B $; then $ B $ lies in the $ \delta $-neighborhood of $ x_0 $. Now
	\begin{equation}
		\delta \leqslant f(x_0) \leqslant d(x_0,C_m) \nonumber
	\end{equation}
	where $ d(x_0,C_m) $ is the largest of the numbers $ d(x_0,C_i) $. Then the $ \delta $-neighborhood of $ x_0 $ is contained in the element $ A_m=X \setminus C_m $ of the covering $ \mathscr{A} $.
\end{proof}
\begin{Theorem}[Uniform Continuity Theorem]
	Let $ f:X \to Y $ be a continuous map of the compact metric space $ (X,d_X) $ to the metric space $ (Y,d_Y) $. Then $ f $ is uniformly continuous.
\end{Theorem}	
\begin{proof}
	Given $ \varepsilon>0 $, take the open covering of $ Y $ by balls $ B(y,\varepsilon/2) $. Let $ \mathscr{A} $ be the open covering of $ X $ by the inverse images of these balls under $ f $. Choose $ \delta $ to be a Lebesgue number for the covering $ \mathscr{A} $. Then if $ x_1 $ and $ x_2 $ are two points of $ X $ such that $ d_X(x_1,x_2)<\delta $, the two-point set $ \{x_1,x_2\} $ has diameter less than $ \delta $, so that its image $ \{f(x_1),f(x_2) \} $ lies in some ball $ B_(y,\varepsilon/2) $. Then $ d_Y(f(x_1),f(x_2) <\varepsilon$, as desired.
\end{proof}
\begin{Definition}
	If $ X $ is a space, a point $ x $ of $ X $ is said to be an \textit{isolated point} of $ X $ if the one-point set $ \{x\} $ is open in $ X $.
\end{Definition}
\begin{Theorem}
	Let $ X $ be a nonempty compact Hausdorff space. If $ X $ has no isolated points, then $ X $ is uncountable.
\end{Theorem}
\begin{proof}
	\begin{enumerate}
		\item We show first that given any nonempty open set $ U $ of $ X $ and any point $ x $ of $ X $, there exists a nonempty open set $ V $ contained in $ U $ such that $ x \notin \bar{V} $.\newpara
		Choose a point $ y $ of $ U $ different from $ x $; this is possible if $ x $ is in $ U $ because $ x $ is not an isolated point of $ X $ and it is possible if $ x $ is not in $ U $ simply because $ U $ is nonempty. Now choose disjoint open sets $ W_1 $ and $ W_2 $ about $ x $ and $ y $, respectively. Then the set $ V=W_2 \cap U $ is the desired open set; it is contained in $ U $, it is nonempty because it contains $ y $, and its closure does not contain $ x $(if not so, then $ x $ must be a boundary point, and in this case every neighborhood of $ x $ intersects $ V $, contradicts our construction). 
		\item We show that given $ f:\Zahlen_+ \to X $, the function $ f $ is not surjective. It follows that $ X $ is uncountable. \newpara
		Let $ x_n=f(n) $. Apply step $ 1 $ to the nonempty open set $ U=X $ to choose a nonempty open set $ V_1\subset X $ such that $ \bar{V_1} $ does not contain $ x_1 $. In general, given $ V_{n-1} $ open and nonempty, choose $ V_n $ to be a nonempty open set such that $ V_n \subset V_{n-1} $ and $ \bar{V_n} $ does not contain $ x_n $. Consider the nested sequence formed by the closure of $ V_i $. Because $ X $ is compact, there is a point $ x \in \bigcap \bar{V_n} $. Now $ x $ cannot equal $ x_n $ for any $ n $, since $ x $ belongs to $ \bar{V_n} $ and $ x_n $ does not.
	\end{enumerate}
\end{proof}
\begin{Corollary}
	Every closed interval in $ \Real $ is uncountable.
\end{Corollary}
	
	
\section{Limit Point Compactness}	
\begin{Definition}
	A space $ X $ is said to be \textit{limit point compact} if every infinite subset of $ X $ has a limit point. Sometimes this property is also called the "Fr\'echet compactness"  or the "Bolzano-Weierstrass property".
\end{Definition}
\begin{Theorem}
	Compactness implies limit point compactness, but not conversely.
\end{Theorem}
\begin{proof}
	Let $ X $ be a compact space. Given a subset $ A $ of $ X $, we prove the contrapositive--if $ A $ has no limit point, then $ A $ must be finite. \newpara
	Suppose $ A $ has no limit point. Then $ A $ contains all its limit points, so that $ A $ is closed. Furthermore, for each $ a \in A $ we can choose a neighborhood $ U_a $ of $ a $ such that $ U_a $ intersects $ A $ in the point $ a $ alone. The space $ X $ is covered by the open set $ X \setminus A $ and the open sets $ U_a $; being compact, it can be covered by finitely many of these sets. Since $ X \setminus A $ does not intersect $ A $, and each set $ U_a $ contains only one point of $ A $, the set $ A $ must be finite.
\end{proof}
\begin{Example}
	Let $ Y $ consist of two points; give $ Y $ the trivial topology. Then the space $ X = \Zahlen_+ \times Y $ is limit point compact, for every nonempty subset of $ X $ has a limit point. It is not compact, for the covering of $ X $ by the open sets $ U_n = \{n\}\times Y $ has no finite subcollection covering $ X $.
\end{Example}
\begin{Definition}
	The topological space $ X $ is said to be \textit{sequentially compact} if every sequence of points of $ X $ has a convergent subsequence.
\end{Definition}	
\begin{Theorem}
Let $ X $ be a metrizable space. Then the following are equivalent:
\begin{enumerate}
	\item $ X $ is compact.
	\item $ X $ is limit point compact.
	\item $ X $ is sequentially compact.
\end{enumerate}
\end{Theorem}
\begin{proof}
	We show that $ (2)\Rightarrow(3) $. Assume that $ X $ is limit point compact. Given a sequence $ (x_n) $, consider the set $ A=\{x_n|n \in \Zahlen_+ \} $. If the set $ A $ is finite, then there is a point $ x $ such that $ x=x_n $ for infinitely many values of $ n $. In this case, the sequence itself converges trivially. On the other hand, if $ A $ is infinite, then $ A $ has a limit point $ x $. We define a subsequence of $ (x_n) $ converging to $ x $ as follows: First choose $ n_1 $ so that
	\begin{equation}
		x_{n_1}\in B(x,1) \nonumber
	\end{equation}
	Then suppose that the positive integer $ n_{i-1} $ is given. Because the ball $ B(x,1/i) $ intersects $ A $ in infinitely many points, we can choose an index $ n_i > n_{i-1} $ such that
	\begin{equation}
		x_{n_i} \in B(x,1/i) \nonumber
	\end{equation}
	Then the subsequence $ x_{n_1},x_{n_2},\cdots $ converges to $ x $. \newpara
	Finally, we show that $ (3)\Rightarrow(1) $.
	\begin{enumerate}
		\item We show that if $ X $ is sequentially compact, then the Lebesgue number lemma holds for $ X $. Let $ \mathscr{A} $ be an open covering of $ X $. We assume that there is no $ \delta>0 $ such that each set of diameter less than $ \delta $ has an element of $ \mathscr{A} $ containing it, and derive a contradiction. Our assumption relies in particular that for each positive integer $ n $, there exists a set of diameter less than $ 1/n $ that is not contained in any element of $ \mathscr{A} $; let $ C_n $ be such a set. Choose a point $ x_n \in C_n $, for each $ n $. By hypothesis, some subsequence $ (x_{n_1}) $ of the sequence $ (x_n) $ converges, say to the point $ a $. Now $ a $ belongs to some element $ A $ of the collection $ \mathscr{A} $; because $ A $ is open, we may choose an $ \varepsilon>0 $ such that $ B(a,\varepsilon)\subset A $. If $ i $ is large enough that $ 1/n_{i} <\varepsilon/2$, then the set $ C_{n_i} $ lies in the $ \varepsilon/2 $ neighborhood of $ x_{n_i} $; if $ i $ is large enough that $ d(x_{n_i},a) < \varepsilon/2 $, then $ C_{n_i} $ lies in the $ \varepsilon $-neighborhood of $ a $. But this means that $ C_{n_i}\subset A $, contrary to hypothesis. 
		\item We show that if $ X $ is sequentially compact, then given $ \varepsilon>0 $, there exists a finite covering of $ X $ by open $ \varepsilon $-balls. We will assume that there exists an $ \varepsilon>0 $ such that $ X $ cannot be covered by finitely many $ \varepsilon $-balls and use it to derive a contradiction. Construct a sequence of points $ x_n $ of $ X $ as follows: First, choose $ x_1 $ to be any point of $ X $. Noting that the ball $ B_(x_1,\varepsilon) $ is not all of $ X $(otherwise $ X $ could be covered by a single ball), choose $ x_2 $ to be a point of $ X $ not in $ B(x_1,\varepsilon) $. In general, given $ x_1,\cdots,x_n $, choose $ x_{n+1} $ to be a point not in the union
		\begin{equation}
			B(x_1,\varepsilon)\cup \cdots \cup B(x_n, \varepsilon) \nonumber
		\end{equation}
		using the fact that these balls do not cover $ X $. Note that by construction $ d(x_{n+1},x_i)\geqslant \varepsilon $ for $ i=1,\cdots,n $. Therefore, the sequence $ (x_n) $ can have no convergent subsequence; in fact, any ball of radius $ \varepsilon/2 $ can contain $ x_n $ for at most one value of $ n $.
		\item Finally, we show that if $ X $ is sequentially compact, then $ X $ is compact. Let $ \mathscr{A} $ be an open covering of $ X $. Because $ X $ is sequentially compact, the open covering $ \mathscr{A} $ has a Lebesgue number $ \delta $. Let $ \varepsilon=\delta/3 $; use sequential compactness of $ X $ to find a finite covering of $ X $ by open $ \varepsilon $-balls. Each of these balls has diameter at most $ 2\delta/3 $, so it lies in an element of $ \mathscr{A} $. Choosing one such element of $ \mathscr{A} $ for each of these $ \varepsilon $-balls, we obtain a finite subcollection of $ \mathscr{A} $ that covers $ X $.
	\end{enumerate}
\end{proof}

\section{Local Compactness}
\begin{Definition}
	A space $ X $ is said to be \textit{locally compact at $ x $} if there is some compact subspace $ C $ of $ X $ that contains a neighborhood of $ x $. If $ X $ is locally compact at each of its points, $ X $ is said to be \textit{locally compact}.
\end{Definition}
\begin{Theorem}
	Let $ X $ be a space. Then $ X $ is locally compact Hausdorff iff there exists a space $ Y $ satisfying the following conditions:
	\begin{enumerate}
		\item $ X $ is a subspace of $ Y $.
		\item The set $ Y \setminus X $ consists of a single point.
		\item $ Y $ is a compact Hausdorff space.
	\end{enumerate}
If $ Y $ and $ Y^\prime $ are two spaces satisfying these conditions, then there is a homeomorphism of $ Y $ with $ Y^\prime $ that equals the identity map on $ X $.
\end{Theorem}
\begin{proof}
	\begin{enumerate}
		\item We first verify uniqueness. Let $ Y $ and $ Y^\prime $ be two spaces satisfying these conditions. Define $ h:Y \to Y^\prime $ by letting $ h $ map the single point $ p $ of $ Y \setminus X $ to the point $ q^\prime $ of $ Y^\prime \setminus X $, and letting $ h $ equal the identity on $ X $. We show that if $ U $ is open in $ Y $, then $ h(U) $ is open in $ Y^\prime $. Symmetry then implies that $ h $ is a homeomorphism. \newpara
		First consider the case where $ U $ does not contain $ p $. Then $ h(U)=U $. Since $ U $ is open in $ Y $ and is contained in $ X $, it is open in $ X $. Because $ X $ is open in $ Y^\prime $, the set $ U $ is also open in $ Y^\prime $, as desired. \newpara
		Second, suppose that $ U $ contains $ p $. Since $ C=Y \setminus U $ is closed in $ Y $, it is compact as a subspace of $ Y $. Because $ C $ is contained in $ X $, it is a compact subspace of $ X $. Then because $ X $ is a subspace of $ Y^\prime $, the space $ C $ is also a compact subspace of $ Y^\prime $. Because $ Y^\prime $ is Hausdorff, $ C $ is closed in $ Y^\prime $, so that $ h(U)=Y^\prime \setminus C $ is open in $ Y^\prime $, as desired.
		\item Now we suppose $ X $ is locally compact Hausdorff and construct the space $ Y $. Let us take some object that is not a point of $ X $, denote it by the symbol $ \infty $ and adjoin it to $ X $, forming the set $ Y=X \cup \{\infty \} $. Topologize $ Y $ by defining the collection of open sets of $ Y $ to consist of $ (1) $ all sets $ U $ that are open in $ X $, and $ (2) $ all sets of the form $ Y \setminus C $, where $ C $ is a compact subspace of $ X $. \newpara
		The empty set is a set of type $ (1) $, and the space $ Y $ is a set of type $ (2) $. Checking that the intersection of two open sets is open involves three cases:
		\begin{align}
			U_1 &\cap U_2 \qquad &\text{is of type }(1)\nonumber\\
			(Y \setminus C_1)\cap (Y \setminus C_2)&= Y \setminus(C_1 \cup C_2)\qquad &\text{is of type }(2)\nonumber\\
			U_1 \cap (Y \setminus C_1)&=U_1 \cap (X \setminus C_1)\qquad &\text{is of type }(1)\nonumber
		\end{align}
	because $ C_1 $ is closed in $ X $. Similarly, one check that the union of any collection of open sets is open
	\begin{align}
		\bigcup U_\alpha &= U \qquad &\text{is of type }(1)\nonumber\\
		\bigcup(Y \setminus C_\beta)=Y \setminus(\bigcap C_\beta)&=Y \setminus C\qquad &\text{is of type }(2)\nonumber\\
		(\bigcup U_\alpha)\cup(\bigcup(Y \setminus C_\beta))=U \cup (Y \setminus C)&=Y \setminus(C \setminus U)\nonumber
	\end{align}
which is of type $ (2) $ because $ C \setminus U $ is a closed subspace of $ C $ and therefore compact. \newpara
Now we show that $ X $ is a subspace of $ Y $. Given any open set of $ Y $, we show its intersection with $ X $ is open in $ X $. If $ U $ is of type $ (1) $, then $ U \cap X = U $; if $ Y \setminus C $ is of type $ (2) $, then $ (Y \setminus C)\cap X = X \cap C $; both of these sets are open in $ X $. Conversely, any set open in $ X $ is a set of type $ (1) $ and therefore open in $ Y $ by definition. \newpara
To show that $ Y $ is compact, let $ \mathscr{A} $ be an open covering of $ Y $. The collection $ \mathscr{A} $ must contain an open set of type $ (2) $, say $ Y \setminus C $, since none of the open sets of type $ (1) $ contain the point $ \{\infty \} $. Take all the members of $ \mathscr{A} $ different from $ Y \setminus C $ and intersect them with $ X $; they form a collection of open sets of $ X $ covering $ C $. Because $ C $ is compact, finitely many of them cover $ C $; the corresponding finite collection of elements of $ \mathscr{A} $ will, along with the element $ Y \setminus C $, cover all of $ Y $. \newpara
To show that $ Y $ is Hausdorff, let $ x $ and $ y $ be two points of $ Y $. If both of them lie in $ X $, there are disjoint sets $ U $ and $ V $ open in $ X $ containing them, respectively. On the other hand, if $ x \in X $ and $ y=\infty $, we can choose a compact set $ C $ in $ X $ containing a neighborhood $ U $ of $ x $. Then $ U $ and $ Y \setminus C $ are disjoint neighborhoods of $ x $ and $ \infty $, respectively, in $ Y $.
\item Finally, we prove the converse. Suppose a space $ Y $ satisfying conditions $ (1)-(3) $ exists. Then $ X $ is Hausdorff because it is a subspace of the Hausdorff space $ Y $. Given $ x \in X $, we show $ X $ is locally compact at $ x $. Choose disjoint open sets $ U $ and $ V $ of $ Y $ containing $ x $ and the single point of $ Y \setminus X $, respectively. Then the set $ C = Y \setminus V $ is closed in $ Y $, so it is a compact subspace of $ Y $. Since $ C $ lies in $ X $, it is also compact as a subspace of $ X $; it contains the neighborhood $ U $ of $ x $.
	\end{enumerate}
\end{proof}
\begin{Definition}
	If $ Y $ is a compact Hausdorff space and $ X $ is a proper subspace of $ Y $ whose closure equals $ Y $, then $ Y $ is said to be a \textit{compactification} of $ X $. If $ Y \setminus X $ equals a single point, then $ Y $ is called the \textit{one-point compactification} of $ X $.
\end{Definition}	
\begin{Theorem}
	Let $ X $ be a Hausdorff space. Then $ X $ is locally compact iff given $ x $ in $ X $, and given a neighborhood $ U $ of $ x $, there is a neighborhood $ V $ of $ x $ such that $ \bar{V} $ is compact and $ \bar{V}\subset U $.
\end{Theorem}
\begin{proof}
	The set $ C = \bar{V} $ is the desired compact set containing a neighborhood of $ x $. To prove the converse, suppose $ X $ is locally compact; let $ x $ be a point of $ X $ and let $ U $ be a neighborhood of $ x $. Take the one-point compactification $ Y $ of $ X $, and let $ C $ be the set $ Y \setminus U $. Then $ C $ is closed in $ Y $, so that $ C $ is a compact subspace of $ Y $. Choose disjoint open sets $ V $ and $ W $ containing $ x $ and $ C $, respectively. Then the closure $ \bar{V} $ of $ V $ in $ Y $ is compact; furthermore, $ \bar{V} $ is disjoint from $ C $, so that $ \bar{V}\subset U $, as desired.
\end{proof}	
\begin{Corollary}
	Let $ X $ be locally compact Hausdorff; let $ A $ be a subspace of $ X $. If $ A $ is closed in $ X $ or open in $ X $, then $ A $ is locally compact.
\end{Corollary}
\begin{proof}
	Suppose that $ A $ is closed in $ X $. Given $ x \in A $, let $ C $ be a compact subspace of $ X $ containing the neighborhood $ U $ of $ x $ in $ X $. Then $ C \cap A $ is closed in $ C $ and thus compact, and it contains the neighborhood $ U \cap A $ of $ x $ in $ A $.\newpara
	Suppose now that $ A $ is open in $ X $. Given $ x \in A $, we apply the preceding theorem to choose a neighborhood $ V $ of $ x $ in $ X $ such that $ \bar{V} $ is compact and $ \bar{V}\subset A $. Then $ C = \bar{V} $ is a compact subspace of $ A $ containing the neighborhood $ V $ of $ x $ in $ A $.
\end{proof}
\begin{Corollary}
	A space $ X $ is homeomorphic to an open subspace of a compact Hausdorff space iff $ X $ is locally compact Hausdorff.
\end{Corollary}


\section{Nets}
\begin{Definition}
	A \textit{directed set} $ J $ is a set with a partial order $ \preceq $ such that for each pair $ \alpha $, $ \beta $ of elements of $ J $, there exists an element $ \gamma $ of $ J $ having the property that $ \alpha \preceq \gamma $ and $ \beta \preceq \gamma $.
\end{Definition}
\begin{Definition}
	A subset $ K $ of $ J $ is said to be \textit{cofinal} in $ J $ if for each $ \alpha \in J $, there exists $ \beta \in K $ such that $ \alpha \preceq \beta $.
\end{Definition}
If $ J $ is a directed set and $ K $ is cofinal in $ J $, then $ K $ is a directed set.
\begin{Definition}
	Let $ X $ be a topological space. A \textit{net} in $ X $ is a function $ f $ from a directed set $ J $ into $ X $. If $ \alpha \in J $, we usually denote $ f(\alpha) $ by $ x_\alpha $. We denote the net $ f $ itself by the symbol $ (x_\alpha)_{\alpha \in J} $, or by $ (x_\alpha) $ if the index set is understood.
\end{Definition}
\begin{Definition}
	The net $ (x_\alpha) $ is said to \textit{converge} to the point $ x $ of $ X $ (written $ x_\alpha \to x $) if for each neighborhood $ U $ of $ x $, there exists $ \alpha \in J $ such that
	\begin{equation}
		\alpha \preceq \beta \Rightarrow x_\beta \in U \nonumber
	\end{equation}
\end{Definition}
Now we can see that the concept of net is a direct generalization of sequences when the directed set is chosen to be $ \Zahlen_+ $.
\begin{Proposition}
	If $ (x_\alpha)_{\alpha \in J}\to x \in X $ and $ (y_\alpha)_{\alpha \in J}\to y \in Y $, then $ (x_\alpha \times y_\alpha) \to x \times y \in X \times Y $.
\end{Proposition}
\begin{Theorem}
	Let $ A \in X $. Then $ x \in \bar{A} $ iff there is a net of points of $ A $ converging to $ x $.
\end{Theorem}
\begin{proof}
	The implication $ \Leftarrow $ is trivial since it implies that $ x $ is a limit point of $ A $. To prove the implication $ \Rightarrow $, consider the collection of all neighborhoods of $ x $ partially ordered by reverse inclusion. On this directed set there is a net on itself. Since $ x\in \bar{A} $, $ x $ is a limit point of $ A $. Then we can select one point from each neighborhood of $ x $, then this net converges to $ x $.
\end{proof}
\begin{Theorem}
	Let $ f:X \to Y $. Then $ f $ is continuous iff for every convergent net $ (x_\alpha) $ in $ X $ converging to $ x $ the net $ (f(x_\alpha)) $ converges to $ f(x) $.
\end{Theorem}
\begin{proof}
	Consider the same directed set from our last theorem.
\end{proof}
\begin{Definition}
	Let $ f:J \to X $ be a net in $ X $; let $ f(\alpha)=x_\alpha $. If $ K $ is a directed set and $ g:K \to J $ is a function such that 
	\begin{enumerate}
		\item $ i \preceq j \Rightarrow g(i)\preceq g(j) $.
		\item $ g(K) $ is cofinal in $ J $.
	\end{enumerate}
then the composition $ f \circ g:K \to X $ is called a \textit{subnet} of $ (x_\alpha) $.
\end{Definition}
\begin{Proposition}
	If the net $ (x_\alpha) $ converges to $ x $, so does any subnet.
\end{Proposition}
\begin{Definition}
	Let $ (x_\alpha)_{\alpha \in J} $ be a net in $ X $. We say that $ x $ is an \textit{accumulation point} of the net $ (x_\alpha) $ if for each neighborhood $ U $ of $ x $, the set of those $ \alpha $ for which $ x_\alpha \in U $ is cofinal in $ J $.
\end{Definition}
\begin{Lemma}
	The net $ (x_\alpha) $ has the point $ x $ as an accumulation point iff some subnet of $ (x_\alpha) $ converges to $ x $.
\end{Lemma}
\begin{proof}
	\exercise
\end{proof}
\begin{Theorem}
	$ X $ is compact iff for every net in $ X $ has a convergent subnet.
\end{Theorem}
\begin{proof}
	\exercise
\end{proof}



\chapter{Countability and Separation Axioms}	
\section{The Countability Axioms}	
\begin{Definition}
	A space $ X $ is said to have a \textit{countable basis at $ x $} if there is a countable collection $ \mathscr{B} $ of neighborhoods of $ x $ such that each neighborhood of $ x $ contains at least one of the elements of $ \mathscr{B} $. A space that has a countable basis at each of its points is said to satisfy the \textit{first countability axiom}, or to be \textit{first-countable}.
\end{Definition}
\begin{Theorem}
	Let $ X $ be a topological space.
	\begin{enumerate}
		\item Let $ A $ be a subset of $ X $. If there is a sequence of points of $ A $ converging to $ x $, then $ x \in \bar{A} $; the converse holds if $ X $ is first-countable.
		\item Let $ f:X \to Y $. If $ f $ is continuous, then for every convergent sequence $ x_n \to x $ in $ X $, the sequence $ f(x_n) $ converges to $ f(x) $. The converse holds if $ X $ is first-countable.
	\end{enumerate}
\end{Theorem}
\begin{Definition}
	If a space $ X $ has a countable basis for its topology, then $ X $ is said to satisfy the \textit{second countability axiom}, or to be \textit{second-countable}. The second axiom implies the first.
\end{Definition}
\begin{Theorem}
	A subspace of a first-countable space is first countable, and a countable product of first-countable spaces is first-countable. A subspace of a second-countable space is second-countable, and a countable-product of second-countable spaces is second-countable.
\end{Theorem}	
\begin{proof}
	The proof is trivial -- just consider the basis for subspace topology and product topology.
\end{proof}
\begin{Definition}
	A subset $ A $ of a space $ X $ is said to be \textit{dense} in $ X $ if $ \bar{A}=X $.
\end{Definition}
\begin{Theorem}
	Suppose that $ X $ has a countable basis. Then:
	\begin{enumerate}
		\item Every open covering of $ X $ contains a countable subcollection covering $ X $.
		\item There exists a countable subset of $ X $ that is dense in $ X $.
	\end{enumerate}
\end{Theorem}
\begin{proof}
	Let $ \{B_n \} $ be a countable basis for $ X $.
	\begin{enumerate}
		\item Since $ \{B_n \} $ itself covers $ X $, one can choose elements from an open covering $ \mathscr{A} $ of $ X $ such that each $ A_n $ contains the basis element $ B_n $, and the result is a countable open covering of $ X $. 
		\item For each nonempty basis element $ B_n $, choose a point $ x_n $. Let $ D $ be the set consisting of the points $ x_n $. Then $ D $ is dense in $ X $: Given any point $ x $ of $ X $, every basis element containing $ x $ intersects $ D $, so $ x $ belongs to $ \bar{D} $.
	\end{enumerate}
\end{proof}
\begin{Definition}
	A space for which every open covering contains a countable subcovering is called a \textit{Lindel\"of space}.
\end{Definition}
\begin{Definition}
	A space having a countable dense subset is often said to be \textit{separable}.
\end{Definition}
Each of these properties is equivalent to the second countability axiom when the space is metrizable.
	
	
	
	
	
	
	
	
	
	
	
	
	
	
	
	
	
	
	
	
	
	
	
	
	
	
	
	
	
\end{document}