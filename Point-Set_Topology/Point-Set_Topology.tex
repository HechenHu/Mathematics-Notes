%%%%%%%%%%%%%%%%%%%%%%%%%%%%%%%%%%%%%%%%%%%%%%%%%%%
%% LaTeX book template                           %%
%% Author:  Amber Jain (http://amberj.devio.us/) %%
%% License: ISC license                          %%
%%%%%%%%%%%%%%%%%%%%%%%%%%%%%%%%%%%%%%%%%%%%%%%%%%%

\documentclass[a4paper,11pt]{book}
\usepackage[T1]{fontenc}
\usepackage[utf8]{inputenc}
\usepackage{lmodern}
%%%%%%%%%%%%%%%%%%%%%%%%%%%%%%%%%%%%%%%%%%%%%%%%%%%%%%%%%
% Source: http://en.wikibooks.org/wiki/LaTeX/Hyperlinks %
%%%%%%%%%%%%%%%%%%%%%%%%%%%%%%%%%%%%%%%%%%%%%%%%%%%%%%%%%
\usepackage{/Users/HechenHu/Development/NoteTaking/Mathematics-Notes/Customized}
%%%%%%%%%%%%%%%%%%%%%%%%%%%%%%%%%%%%%%%%%%%%%%%%
% Chapter quote at the start of chapter        %
% Source: http://tex.stackexchange.com/a/53380 %
%%%%%%%%%%%%%%%%%%%%%%%%%%%%%%%%%%%%%%%%%%%%%%%%

\makeatletter

\renewcommand{\@chapapp}{}% Not necessary...

\newenvironment{chapquote}[2][2em]
{\setlength{\@tempdima}{#1}%
	\def\chapquote@author{#2}%
	\parshape 1 \@tempdima \dimexpr\textwidth-2\@tempdima\relax%
	\itshape}
{\par\normalfont\hfill--\ \chapquote@author\hspace*{\@tempdima}\par\bigskip}
\makeatother

%%%%%%%%%%%%%%%%%%%%%%%%%%%%%%%%%%%%%%%%%%%%%%%%%%%
% First page of book which contains 'stuff' like: %
%  - Book title, subtitle                         %
%  - Book author name                             %
%%%%%%%%%%%%%%%%%%%%%%%%%%%%%%%%%%%%%%%%%%%%%%%%%%%

% Book's title and subtitle
\title{\Huge \textbf{Point-Set Topology}}
% Author
\author{\textsc{Hechen Hu}}

\begin{document}
	\frontmatter
	\maketitle
	%%%%%%%%%%%%%%%%%%%%%%%%%%%%%%%%%%%%%%%%%%%%%%%%%%%%%%%%%%%%%%%%%%%%%%%%
	% Auto-generated table of contents, list of figures and list of tables %
	%%%%%%%%%%%%%%%%%%%%%%%%%%%%%%%%%%%%%%%%%%%%%%%%%%%%%%%%%%%%%%%%%%%%%%%%
	
	\tableofcontents
	\mainmatter
	
	\chapter{Basic Set Theory}
	\section{Classification of Relations}
	\begin{Definition}
		An \textit{equivalence relation} is a relation that satisfy the following properties: \\
		\newline
		$ aRa $ (Reflexivity);\\
		$ aRb \Rightarrow bRa $ (Symmetry); \\
		$ (aRb)\land (bRc) \Rightarrow aRc $ (Transitivity).\\
		\newline
		An equivalence relation is denoted by the special symbol $ \sim $. $ a\sim b $ means $ a $ is \textit{equivalent} to $ b $.	
	\end{Definition}
	\begin{Definition}
		Let $ R(\sim) $ be an equivalence relation on $ A $. If $ a \in A $, the \textit{equivalence class} of $ a $ (denoted $ \bar{a} $) is the class of all those elements of $ A $ that are equivalent to $ a $. The class of all equivalence classes in $ A $ is denoted $ A/R $ and called the \textit{quotient class} of $ A $ by $ R $.
	\end{Definition}
	\begin{Theorem}
		Two equivalence classes are either disjoint or equal.
	\end{Theorem}
	
	\begin{Definition}
		A \textit{partial ordering} on a set $ X^2 $ is a relation $ R $ that have the following properties: \\
		\newline
		$ aRa $ (Reflexivity);\\
		$ (aRb)\land (bRc) \Rightarrow aRc $ (Transitivity).\\
		$ (aRb) \land (bRa )\Rightarrow (a=b) $ (Anti-symmetry);	
	\end{Definition}
	
	We often write $ a\preceq b $ and say that $ b $ \textit{follows} $ a $.
	If the condition
	\begin{equation}
	\forall a \forall b((aRb)\lor(bRa) \nonumber
	\end{equation}
	holds in addition to transitivity and anti-symmetry defining a partial ordering relation(this means any two elements of $ X $ is comparable), the relation $ R $ is called an \textit{ordering}, and the set $ X $ is said to be \textit{linearly ordered}.
	\begin{Definition}
		A relation $ \prec $ is called a \textit{strict partial order} if it's nonreflexive and transitive.
	\end{Definition}
	\begin{Theorem}[The Maximum Principle]
		Let $ A $ be a set and $ \prec $ be a strict partial order on $ A $. Then there exists a maximal simply ordered(linearly ordered) subset of $ A $.
	\end{Theorem}
	\begin{Lemma}[Zorn's Lemma]
		Let $ A $ be a set that is strictly partially ordered. If every simply ordered subset of $ A $ has an upper bound in $ A $, then $ A $ has a maximal element.
	\end{Lemma}
	\begin{Definition}
		If $ X $ is a set and $ < $ is an order relation on $ X $, and if $ a<b $, the notation $ (a,b) $ to denote the set
		\begin{equation}
		\{x|a<x<b \} \nonumber
		\end{equation}
		it is called an \textit{open interval} in $ X $. If this set is empty, $ a $ is called the \textit{immediate predecessor} of $ b $, and $ b $ is called the \textit{immediate successor} of $ a $.
	\end{Definition}
	\begin{Definition}
		Suppose that $ A $ and $ B $ are two sets with order relations $ <_A $ and $ <_B $ respectively. We say that $ A $ and $ B $ have the same \textit{order type} if there is a bijective correspondence between them that preserves order, that is, if $ f:A \to B $ is a bijection
		\begin{equation}
		a_1 <_A a_2 \Rightarrow f(a_1)<_B f(a_2) \nonumber
		\end{equation}
	\end{Definition}
	\begin{Example}
		The interval $ (1,1) $ of real numbers has the same order type as $ \Real $, for the function $ f:(-1,1)\to \Real $ given by
		\begin{equation}
		f(x)=\frac{x}{1-x^2} \nonumber
		\end{equation}
		is an order-preserving bijection.
	\end{Example}
	\begin{Definition}
		Suppose $ A $ and $ B $ are two sets with order relations $ <_A $ and $ <_B $ respectively. Define an order relation $ < $ on $ A \times B $ by defining
		\begin{equation}
		a_1 \times b_1 < a_2 \times b_2
		\end{equation}
		if $ a_1 <_A a_2 $, or if $ a_1 = a_2 $ and $ b_1 <_B b_2 $. It is called the \textit{dictionary order relation} on $ A \times B $.
	\end{Definition}
	
	\subsubsection{Functions and Their Graphs}
	\begin{Definition}
		A relation $ R $ is said to be functional if
		\begin{equation}
		(xRy_1)\land(xRy_2)\Rightarrow (y_1=y_2) \nonumber
		\end{equation}
		and it is called a \textit{function}.\\
		$ R\subset X \times Y $ is a \textit{mapping from $ X $ into $ Y $}, or a \textit{function from $ X $ into $ Y $}.\\
	\end{Definition}
	\section{Sets and Operations on Them}
	
	\subsection{Naive Set Theory}
	1$^0$.\quad A set may consist of any distinguishable objects($x\in A\Rightarrow\exists!x\in A$) \\
	2$^0$.\quad A set is unambiguously determined by the collection of objects that comprise it. \\
	3$^0$.\quad Any property defines the set of objects having that property($A = \{x|P(x)\}\Rightarrow P(A)$). \\ \\
	However, this will lead to Russell's Paradox: \\
	Let's have$P(M) \coloneqq M\notin M$ \\
	Consider the class $K = \{M|P(M)\}$. If so $K$ is not a set, since whether $P(K)$ is true or false, contradiction arises.\\
	
	\subsection{ZFC: Zermelo-Fraenkel Axioms and Axiom of Choice}
	1$^0$.\quad \textbf{(Axiom of Extensionality)} Sets $ A $ and $B$ are equal iff they have the same elements. $ (A=B)\Leftrightarrow(\forall x((x\in A)\Leftrightarrow(x \in B))) $\\
	2$^0$. \quad \textbf{(Axiom of Seperation)} To any set $A$ and any property P there corresponds a set $B$ whose elements are those elements of $A$, and only those, having property P(if $A$ is a set, then $B=\{x\in A|P(x)\}$ is also a set). \\
	3$^0$. \quad \textbf{(Union Axiom)} For any set $ \mathscr{M} $ whose elements are sets there exists a set $\bigcup M$, called the union of $ M $ and consisting of those elements and only those that belong to some element of $ \mathscr{M} $ $ (x\in \bigcup \mathscr{M} \Leftrightarrow \exists X((X\in \mathscr{M})\land (x\in X))) $ \\
	Similarly, the intersection of the set $ \mathscr{M} $ is defined as:
	\begin{equation}
	\bigcap \mathscr{M}  \coloneqq \{x\in \bigcup \mathscr{M} | \forall X((X\in \mathscr{M})\Rightarrow (x\in X))\} \nonumber
	\end{equation}
	4$^0$ \quad \textbf{(Pairing Axiom)} For any sets $ X $ and $ Y $ there exists a set $ Z $ such that $ X $ and $ Y $ are its only elements. \\
	5$^0$ \quad \textbf{(Power Set Axiom)} For any set $ X $ there exists a set $ P(X)$ having each subset of $ X $ as an element, and having no other elements. \\
	\newline
	\begin{Definition}
		The \textit{successor} $X^+$ of the set $ X $ is $X^+ = X\cup \{X\}$.\\
	\end{Definition} 
	\begin{Definition}
		An \textit{inductive} set is a set that $\varnothing$ is one of its elements and the successor of each of its elements aso belongs to it.
	\end{Definition}
	6$^0$ \quad \textbf{(Axiom of Infinity)} There exist inductive sets (Example: $\mathbb{N}_0$).\\ 
	7$^0$ \quad \textbf{(Axiom of Replacement)} Let $ F(x,y) $ be a statement(a formula) such that for every $ x_0 \in X$ there exists a unique object $ y_0 $ such that $ F(x_0,y_0) $ is true. Then the objects $ y $ for which there exists an element $ x\in X $ such that $ F(x,y) $ is true form a set.\\
	And finally, an axiom that is independent of ZF. \\
	\newline
	8$^0$ \quad \textbf{(Axiom of Choice/Zermelo's Axiom)}  Given a collection of disjoint nonempty sets, there exists another set consisting of exactly one element from each element of the original set.
	\begin{Definition}
		A choice function is a function $ f $, defined on a collection $ X $ of nonempty sets, such that for every set $ A $ in $ X $, $ f(A) $ is an element of $ A $.
	\end{Definition}
\begin{Corollary}
	There exists a choice function for any collection of nonempty sets.
\end{Corollary}

	
	
	\subsection{The \textit{Cardinality} of a Set(\textit{Cardinal Numbers})}
	\begin{Definition}
		The set $ X $ is said to be \textit{equipollent} to the set $ Y $ if there exists a bijective mapping of $ X $ onto $ Y $(then $ X\sim Y $).
	\end{Definition}
	\begin{Definition}
		\textit{Cardinality} is a measure of the number of elements of the set. If $ X \sim Y $, we write $ \card X = \card Y $.
	\end{Definition}
	If $ X $ is equipollent to some subset of $ Y $, we say $ \card X\leqslant \card Y $, thus
	\begin{equation}
	(\card X \leqslant \card Y)\coloneqq \exists Z\subset Y (\card X = \card Z) \nonumber
	\end{equation}
	A set is called \textit{finite} if it is not equipollent to any proper subset of itself; otherwise it is called \textit{infinite}. \\
	It has the  properties below:\\
	\newline
	1$^0  \quad (cardX\leqslant \card Y)\land(\card Y \leqslant \card Z)\Rightarrow(\card X \leqslant \card Z)$.\\
	2$^0 \quad (\card X \leqslant \card Y)\land(\card Y \leqslant \card X)\Rightarrow(\card X = \card Y)$(The Schröder–Bernstein theorem).\\
	3$^0 \quad \forall X\forall Y(\card X \leqslant \card Y)\lor(\card Y \leqslant \card X)$(Cantor's theorem).\\
	\newline
	We say $ \card X < \card Y $ if $ (\card X \leqslant \card Y) \land (\card X \neq \card Y)$.\\
	let $ \varnothing $ be the empty set and $ P(X) $ the set of all subsets(thus, the power set) of the set $ X $. Then:
	\begin{Theorem}
		$ \card X < \card P(X) $
	\end{Theorem}
	\begin{proof}
		The assertion is obvious for the empty set, and we shall assume that $ X\neq \varnothing $.\\
		Since $ P(X) $ contains all the one-element subsets of $ X $, $ \card X \leqslant \card P(X)$.\\
		Suppose, contrary to the assertion, that there exists a bijective mapping $ f : X \to P(X) $. Let set $ A = \{x\in X:x\notin f(x)\}$ consisting of the elements $ x \in X $ that do not belong to the set $ f(x)\in P(X) $ assigned to them by the bijection. Because $ A \in P(X) $, there exists $ a\in X $ such that $ f(a) = A $. For the element $ a $ the relation $ a \in A $ or $ a \notin A $ is impossible by the definition of $ A $(Similar to Russell's Paradox).
	\end{proof}
	
	
	
	\subsection{Operations on Sets}
	\begin{tabular}{| l | l | l |}
		\hline
		Notation &Meaning &Definition \\ \hline
		$A\subset B$ &$A$ is a subset of $B$ &$\forall x ((x\in A) \Rightarrow(x\in B))$ \\ \hline
		$A\subsetneq B$ &$A$ is a proper subset of $B$ &$A \neq B \land A \subset B$ \\ \hline
		$A = B$ &$A$ equals to $B$ &$(A \subset B)\land(B \subset A)$ \\ \hline
		$\varnothing$ &Empty Set & $\{x|x\neq x\}$ \\ \hline
		$A \cup B$ &The union of $ A $ and $ B $ &$ \{x|x\in A \lor x\in B\} $ \\
		$ A \cap B $ &The intersection of $ A $ and $ B $ &$ \{x|x\in A \land x\in B\} $ \\ \hline
		$ A\setminus B $ &The difference between $ A $ and $ B $ &$ \{x|x\in A \land x\notin B\} $ \\ \hline
		$ C_M A $ &The complement of $A$ in M &$ \{x|x\in M \land x\notin A\} where A\subset M $ \\ \hline
		$ A \times B $ &The Cartesian Product of $ A $ and $ B $ &$ \{(x,y)|x\in A \land y\in B\} $ \\ \hline
		$ A^2 $ &$ A \times A $ & \\
		\hline
	\end{tabular}
	\newline \\
	In the ordered pair$ z = (x_1,x_2) $ where$ Z = X_1 \times X_2 ,z\in Z,x_1 \in X_1, x_2 \in X_2$, $ x_1 $ is called the \textit{first projection} of the pair $ z $ and denoted  pr$_1 z $ while $ x_2 $ is called the \textit{second projection} of the pair $ z $ and denoted  pr$_2 z $.
	
	
	
	
	
	
	
	\section{Countable and Uncountable Sets}
	
	\begin{Definition}
		A set $ X $ is \textit{countable} if it is equipollent with the set $ \Natural $ of natural numbers, that is, $ \card X = \card \Natural $.
	\end{Definition}
	
	\begin{Proposition}
		An infinite subset of a countable set is countable.
	\end{Proposition}
	\begin{proof}
		Let's consider a countable set $ E $. There is a minimal element of $ E_1\coloneqq E $, which we assign to $ 1 \in \Natural $ and denote $ e_1 \in E $. $ E $ is infinite, so $ E_2 \coloneqq E \setminus e_1 $ is not empty. Following the principle of induction, we can construct a injective mapping from $ \{1,2...\} $ to $ \{e_1,e_2,...\} $. \newpara
		Now we have to prove that this mapping is also surjective. Suppose the contrary, that an element $ e \in E$ does not have a natural number assigned to it. The set $ K=\{n \in E | n \leqslant e\} $ is finite, since it's a subset of $ \Natural $ bounded both from below and above. According to our previous construction, we assign $ 1 $ to $ \min K $, denoted as $ e_1 $, and we can acquire a sequence $ e_1, e_2,...e_{k=\card K} $. But $ e_{k=\card K} $ is $ \max K $, and because $ e\in K \land (\forall n\in K (n\leqslant e))$, $ e = \max K$. Therefore $ e = e_k $, or otherwise it will contradict the uniqueness of maximal element.
	\end{proof}
	
	\begin{Proposition}
		The Union of the sets of a finite or countable system of countable sets is also a countable set.
	\end{Proposition}
	\begin{proof}
		Let $ X_1,X_2...,X_n,... $ is a countable system of sets and each set $ X_m = \{x^1_m,...,x^n_m,...\} $ is itself countable. Since $ \forall m\in \Natural (\card (X=\bigcup_{n \in \Natural}X_n) \geqslant X_m)$, $ X $ is an infinite set. The ordered pair $ (m,n) $ identifies the element $ x^n_m \in X_m $. We can construct a mapping, like $ f : \Natural \times \Natural \to \Natural \coloneqq (m,n) \to \frac{(m+n-2)(m+n-1)}{2}+m $, such that it is bijective. Thus $ X $ is countable. Then because $ \card X \leqslant \card \Natural $ and the fact that $ X $ is infinite, we conclude that $ \card X = \card \Natural $.
	\end{proof}
	
	If it is known that a set is either finite or countable, we say it is \textit{at most countable}($ \card X \leqslant \Natural $).
	
	\begin{Corollary}
		$ \card \Zahlen = \card \Natural $
	\end{Corollary}
	
	\begin{Corollary}
		$ \card \Natural^2 = \card \Natural $(The direct product of countable sets is countable).
	\end{Corollary}
	
	\begin{Corollary}
		$ \card \Quoziente = \card \Natural $, that is, the set of rational numbers is countable.
	\end{Corollary}
	\begin{proof}
		Let $ (m,n) $ denote a rational number $ \frac{m}{n} $. It is known that the pair $ (m,n) $ and $ (m^\prime, n^\prime) $ define the same number iff they are proportional. Thus $ \Quoziente $ is equipollent to some infinite subset of the set $ \Zahlen \times \Zahlen $. Since $ \card \Zahlen^2 = \card \Natural $, we can conclude that $ \card \Quoziente = \card \Natural $.
	\end{proof}
	
	\begin{Corollary}
		The set of algebraic numbers is countable.
	\end{Corollary}
	\begin{proof}
		It can be observed that $ \card \Quoziente \times \Quoziente = \card \Natural  $. By the principle of induction, $ \forall k \in \Natural (\card \Quoziente^k = \card \Natural) $. Let $ r \in \Quoziente^k $ be an ordered set $ (r_1,r_2,...,r_k) $ consists of $ k $ rational numbers. \newpara
		An algebraic equation of degree $ k $ with rational coefficient can be writtne in the reduced form $ x^k + r_1x^{k-1}+ \cdots + r_k = 0 $. Thus there are as many different algebraic equations of degree $ k $ as there are different ordered sets $ (r_1,...,r_k) $ of rational numbers, that is, a countable set. \newpara
		The algebraic equation with rational coefficients (of arbitrary degree) is the union of sets consisting of algebraic equation (of a fixed degree) which is countable, and this union is countable. Each such equation has only a finite number of roots. Hence the set of algebraic numbers is at most countable. But it is infinite, and therefore countable.
	\end{proof}
	
	\subsection{The Cardinality of the Continuum}
	\begin{Definition}
		The set $ \Real $ of real numbers is also called the \textit{number continuum}(from Latin \textit{continuum}, meaning continuous, or solid), and its cardinality the \textit{cardinality of the continuum}.
	\end{Definition}
	
	\begin{Theorem}[Cantor]
		$ \card \Natural < \card \Real $
	\end{Theorem}
	\begin{proof}[Proof by Nested Interval Lemma]
		It is sufficient to show that even $ [0,1] $ in an uncountable set. \newpara
		Assume it is countable, that is, can be written as a sequence $ x_1,x_2,...,x_n,.... $. Take $ x_1 $ on $ I_0 = [0,1] $, and find $ I_1 $ such that $ x_1 \notin I_1 $. Then construct the nested interval $ I_n $ such that $ x_{n+1} \notin I_{n+1} $ and $ |I_n| > 0 $. It follows the nested interval lemma that there exist a point $ c \in [0,1]$ belonging to all $ I_n $. But by our construction, $ c \in \Real $ and $ c $ cannot be any point of the sequence $ x_1,x_2,...,x_n,.... $.
	\end{proof}
	
	\begin{proof}[Proof by Cantor's Diagonal Argument]
		Let's first consider an the set $ L $ and write out the infinite sequence of distinct binary numbers in it which has the form: 
		\begin{align}
		&s1 =	(0,	0,	0,	0,	0,	0,	0,	...) \\
		&s2 =	(1,	1,	1,	1,	1,	1,	1,	...) \\
		&s3 =	(0,	1,	0,	1,	0,	1,	0,	...)\\
		&s4 =	(1,	0,	1,	0,	1,	0,	1,	...)\\
		&s5 =	(1,	1,	0,	1,	0,	1,	1,	...)\\
		&s6 =	(0,	0,	1,	1,	0,	1,	1,	...)\\
		&s7 =	(1,	0,	0,	0,	1,	0,	0,	...)\\
		&... \\		
		\end{align}
		We then constrcut a number $ s $ such that its first digit is the complementary (swapping 0s for 1s and vice versa) of the first digit of $ s_1 $ and etc.
		\begin{align}
		&s1 =	(\mathbf{0},	0,	0,	0,	0,	0,	0,	...) \\
		&s2 =	(1,	\mathbf{1},	1,	1,	1,	1,	1,	...) \\
		&s3 =	(0,	1,	\textbf{0},	1,	0,	1,	0,	...)\\
		&s4 =	(1,	0,	1,	\textbf{0},	1,	0,	1,	...)\\
		&s5 =	(1,	1,	0,	1,	\textbf{0},	1,	1,	...)\\
		&s6 =	(0,	0,	1,	1,	0,	\textbf{1},	1,	...)\\
		&s7 =	(1,	0,	0,	0,	1,	0,	\textbf{0},	...)\\
		&... \\		
		&s = (\textbf{1},\textbf{0},\textbf{1},\textbf{1},\textbf{1},\textbf{0},\textbf{1},..)
		\end{align}
		By construction $ s $ differs from $ s_n $ at the $ n $th digit, so $ s $ is not in this sequence, and thus $ L $ is uncountable. \newpara
		We can now define a mapping $ f : L \to \Real $.$ f(s_n) = r_n\in \Real $ means that $ s_n $ and $ r_n $ have the same digit while $ r_n $ is under base 10 and $ s_n $ is under base 2. For $ s_n \neq s_m \Rightarrow (r_n=f(s_n)) \neq (r_m=f(s_m)) $, $ f $ is injective, and with the fact that all $ s_n $ corresponds to a $ r_n $ together give us $ \card f(L) = \card L $. Since $ f(L) $ is a subset of $ \Real $, we can see that $ \Real $ is also uncountable.
	\end{proof}
The proof above illustrates the theorem below.
\begin{Definition}
	Let $ X $ denote the two element set $ \{ 0,1\} $. Then $ X^\omega $ is uncountable.
\end{Definition}
	The cardinality of $ \Real $ is often denotes as $ \mathfrak{c} $.
	\begin{Corollary}
		$ \Quoziente \neq \Real $, and so irrational numbers exist.
	\end{Corollary}
	
	\begin{Corollary}
		There exist transcendental numbers, since the set of algebraic numbers is countable.
	\end{Corollary}
	
	\begin{Example}
		The cardinality of $ P(X) $, which is the power set of $ X $, satisfy that if $ \card X = n $, $ \card P(X) = 2^{n} $.
	\end{Example}
	\begin{proof}
		We can use the principle of induction to complete the proof. If $ n= 1 $, $ X = \{x\} $, then $ P(X) =\{\varnothing, X\} $, then $ \card P(X) = 2^{1} $. \newpara
		Now if $ n \in \Natural \Rightarrow \card P(X) = 2^{n}  $, let $ X $ be a set that has $ x $ as one of its elements and has the cardinality of $ n+1 $. Therefore $ Y = X \setminus \{x\} $ has $ n $ elements. We can divide $ P(X) $ into two parts: the ones containing $ x $ and the ones don't. If $ x\in A \subset P(X) $, then $ A \setminus \{x\} \subset P(Y) $ and vice versa. Thus we can set up a bijection between $ P(Y) $ and the elements in $ P(X) $ that contains $ x $. Similarly, we can clearly see that a bijection between the subsets of $ P(X) $ that does not contains $ x $ and $ P(Y) $. Thus $ \card P(X) = 2^n + 2^n = 2^{n+1} $, and we complete the proof.
	\end{proof}
	We'll use a script letter to denote the collection of sets, for example, $ \mathscr{A} $ for collection of sets and $ A $ for individual sets in it.
	\begin{Definition}
		A \textit{partition} of a set $ A $, besides the definition we have when studying Riemann Sum, can be defined as a collection of disjoint nonempty subsets of $ A $ whose union is all of $ A $.
	\end{Definition}
\begin{Theorem}
	Given any partition $ \mathscr{D} $ of $ A $, there is exactly one equivalence relation on $ A $ from which it is derived.
\end{Theorem}
\begin{Example}
	Defined two points in the plane to be equivalent if they lie at the same distance from the origin. The collection of equivalence classes consists of all circles centered at the origin, along with the set consisting of the origin alone.
\end{Example}
\begin{Definition}
	Any set together with a order relation $ < $ that satisfy both the following properties
	\begin{enumerate}
		\item $ < $ has the least upper bound property.
		\item if $ x<y $, then there exists an element $ z $ such that $ x<z<y $.
	\end{enumerate}
is called a \textit{linear continuum}.
\end{Definition}
\begin{Definition}
	Let $ \mathscr{A} $ be a nonempty collection of sets. An \textit{indexing function} for $ \mathscr{A} $ is a surjective function $ f $ from some set $ J $, called the \textit{index set}, to $ \mathscr{A} $. The collection $ \mathscr{A} $, together with the indexing function is called an \textit{indexed family of sets}. Given $ \alpha \in J $, the set $ f(\alpha) $ is denoted $ A_\alpha $. The indexed family itself is denoted by
	\begin{equation}
		\{A_\alpha \}_{\alpha \in J} \nonumber
	\end{equation}
\end{Definition}
\begin{Theorem}[Principle of Recursive Definition]
	Let $ A $ be a set. Given a formula that defined $ h(1) $ as a unique element of $ A $, and for $ i>1 $ defines $ h(i) $ uniquely as an element of $ A $ in terms of the values of $ h $ for positive integers less than $ i $, this formula determines a unique function $ h:\Natural \to A $.
\end{Theorem}
\begin{Definition}
	A set $ A $ with an order relation $ < $ is said to be \textit{well-ordered} if every nonempty subset of it has a smallest element.
\end{Definition}
\begin{Theorem}
	Any subset of a well-ordered set is well-ordered. The cartesian product of two well-ordered sets is well-ordered.
\end{Theorem}
\begin{Theorem}
	Every nonempty finite ordered set has the order type of a section of $ \Natural $, so it's well-ordered.
\end{Theorem}
\begin{Theorem}[Well-ordering theorem, proved by Zermelo]
	If $ A $ is a set, there exists an order relation on $ A $ that is a well-ordering.
\end{Theorem}
\begin{Corollary}
	There exists an uncountable well-ordered set.
\end{Corollary}
\begin{Definition}
	Let $ X $ be a well-ordered set. Given $ \alpha \in X $, let $ S_\alpha $ denote the set
	\begin{equation}
		S_\alpha = \{x|x \in X \land x< \alpha \} \nonumber
	\end{equation}
	It is called the \textit{section} of $ X $ by $ \alpha $.
\end{Definition}
\begin{Lemma}
	There exists a well-ordered set $ A $ having a largest element $ \Omega $, such that the section $ S_\Omega $ of $ A $ by $ \Omega $ is uncountable but every other section of $ A $ is countable.
\end{Lemma}
\begin{proof}
	We begin with an uncountable well-ordered set $ B $. Let $ C $ be the well-ordered set $ \{1,2 \}\times B $ in the dictionary order, then some section of $ C $ is uncountable. Let $ \Omega $ be the smallest element of $ C $ for which the section of $ C $ by $ \Omega $ is uncountable, then let $ A $ consist of this section along with $ \Omega $.
\end{proof}
The set $ S_\Omega $ is called a \textit{minimal uncountable well-ordered set}, and the well-ordered set $ A=S_\Omega \cup \{\Omega \} $ by $ \bar{S}_\Omega $.
\begin{Theorem}
	If $ A $ is a countable subset of $ S_\Omega $, then $ A $ has an upper bound in $ S_\Omega $.
\end{Theorem}	
\begin{proof}
	Let $ A $ be a countable subset of $ S_\Omega$. For each $ a \in A $, the section $ S_a $ is countable. Therefore, the union $ B=\bigcup_{\alpha \in A}S_a $ is also countable. Since $ S_\Omega \neq B $, let $ x $ be a point of $ S_\Omega $ that is not in $ B $, and then $ x $ is an upper bound for $ A $.
\end{proof}
	\chapter{Topological Spaces and Continuous Functions}
	\section{Definition for Topological Spaces}
	\begin{Definition}
		A \textit{topology} on a set $ X $ is a collection $ \mathscr{T} $ of subsets of $ X $ having the following properties:
		\begin{enumerate}
			\item $ \varnothing $ and $ X $ are in $ \mathscr{T} $.
			\item The union of the elements of any subcollection of $ \mathscr{T} $ is in $ \mathscr{T} $.
			\item The intersection of the elements of any finite subcollection of $ \mathscr{T} $ is in $ \mathscr{T} $.
		\end{enumerate}
	A set $ X $ for which a topology $ \mathscr{T} $ has been specified is called a \textit{topological space}. A subset $ U $ of $ X $ is an \textit{open set} of $ X $ if $ U $ belongs to the collection $ \mathscr{T} $. Then a topological space is a set $ X $ with a collection of subsets of $ X $, called open sets, such that $ X $ and $ \varnothing $ are both open and arbitrary unions and finite intersections of open sets are open.
	\end{Definition}
\begin{Definition}
	If $ X $ is any set, the collection of all subsets of $ X $ is a topology on $ X $ and called the \textit{discrete topology}. The collection consisting of $ X $ and $ \varnothing $ is also a topology on $ X $ and is called the \textit{indiscrete topology} or the \textit{trivial topology}.
\end{Definition}
\begin{Definition}
	Let $ X $ be a set; let $ \mathscr{T}_f $ be the collection of all subsets $ U $ of $ X $ such that $ X \setminus U $ either is finite or is all of $ X $. Then $ \mathscr{T}_f $ is a topology on $ X $ and called the \textit{finite complement topology}.
\end{Definition}
\begin{Definition}
	Suppose $ \mathscr{T} $ and $ \mathscr{T}^\prime $ are two topologies on a given set $ X $. If $ \mathscr{T}^\prime \supset \mathscr{T} $, we say that $ \mathscr{T}^\prime $ is \textit{finer} than $ \mathscr{T} $; If $ \mathscr{T}^\prime $ properly contains $ \mathscr{T} $, we say that $ \mathscr{T}^\prime $ is \textit{strictly finer} then $ \mathscr{T} $. We also say that $ \mathscr{T} $ is \textit{coarser} than $ \mathscr{T}^\prime $, or \textit{strictly coarser}, in these two respective situations. We say $ \mathscr{T} $ is \textit{comparable} with $ \mathscr{T}^\prime $ if either $ \mathscr{T}^\prime \supset \mathscr{T}  $ or $ \mathscr{T} \supset \mathscr{T}^\prime  $
\end{Definition}
\section{Basis for Topology}
	\begin{Definition}
		If $ X $ is a set, a \textit{basis} for a topology on $ X $ is a collection $ \mathscr{B} $ of subsets of $ X $(called \textit{basis elements}) such that
		\begin{enumerate}
			\item For each $ x \in X $, there is at least one basis element $ B $ containing $ x $.
			\item If $ x $ belongs to the intersection of two basis elements $ B_1 $ and $ B_2 $, then there is a basis element $ B_3 $ containing $ x $ such that $ B_3 \subset B_1 \cap B_2$.
		\end{enumerate}
	If $ \mathscr{B} $ satisfies these two conditions, then we define the \textit{topology $ \mathscr{T} $ generated by $ \mathscr{B} $} as follows: A subset $ U $ of $ X $ is said to be open in $ X $ if for each $ x\in U $, there is a basis element $ B \in \mathscr{B} $ and $ x \in B $ and $ B \subset U $.
	\end{Definition}
\begin{Example}
	If $ X $ is any set, the collection of all one-point subsets of $ X $ is a basis for the discrete topology on $ X $.
\end{Example}
\begin{Lemma}
	Let $ X $ be a set; Let $ \mathscr{B} $ be a basis for a topology $ \mathscr{T} $ on $ X $. Then $ \mathscr{T} $ equals toe collection of all unions of elements of $ \mathscr{B} $.
\end{Lemma}
	\begin{proof}
		Given a collection of elements of $ \mathscr{B} $, they are also elements of $ \mathscr{T} $. Because $ \mathscr{T} $ is a topology, their union is in $ \mathscr{T} $. Conversely, given $ U \in \mathscr{T} $, choose for each $ x\in U $ an element $ B_x $ of $ \mathscr{B} $ such that $ x \in B_x \subset U $. Then $ U = \bigcup_{x\in U}B_x $, so $ U $ equals a union of elements of $ \mathscr{B} $.
	\end{proof}
\begin{Lemma}
	Let $ X $ be a topological space. Suppose that $ \mathscr{C} $ is a collection of open sets of $ X $ such that for each open set $ U $ of $ X $ and each $ x $ in $ U $, there is an element $ C $ of $ \mathscr{C} $ such that $ x \in C \subset U $. Then $ \mathscr{C} $ is a basis for the topology of $ X $.
\end{Lemma}	
\begin{proof}
	First we show that $ \mathscr{C} $ is a basis. Given $ x \in X $, since $ X $ is open, there is by hypothesis an element $ C $ of $ \mathscr{C} $ such that $ x\in C \subset X $. Now let $ x \in C_1 \cap C_2 $, where $ C_1 $ and $ C_2 $ are elements of $ \mathscr{C} $. The intersection of them is open, and there exists by hypothesis an element $ C_3 $ in $ \mathscr{C} $ such that $ x \in C_3 \subset C_1 \cap C_2 $. \newpara
	Let $ \mathscr{T} $ be the collection of open sets of $ X $; we will show that the topology $ \mathscr{T}^\prime $ generated by $ \mathscr{C} $ equals the topology. First, note that if $ U $ belongs to $ \mathscr{T} $ and if $ x \in U $, then there is by hypothesis an element $ C $ of $ \mathscr{C} $ such that $ x \in C \subset U $. It follows that $ U $ belongs to the topology $ \mathscr{T}^\prime $ by definition. Conversely, if $ W $ belongs to the topology $ \mathscr{T} $, then $ W $ equals a union of elements of $ \mathscr{C} $ by the preceding lemma. Since each element of $ \mathscr{C} $ belongs to $ \mathscr{T} $ and $ \mathscr{T} $ is a topology, $ W $ also belongs to $ \mathscr{T} $.
\end{proof}
\begin{Lemma}
	Let $ \mathscr{B} $ and $ \mathscr{B}^\prime $ be bases for the topologies $ \mathscr{T} $ and $ \mathscr{T}^\prime $, respectively, on $ X $. Then the following are equivalent:
	\begin{enumerate}
		\item $ \mathscr{T}^\prime $ is finer than $ \mathscr{T} $.
		\item For each $ x \in X $ and each basis element $ B\in \mathscr{B} $ containing $ x $, there is a basis element $ B^\prime \in \mathscr{B}^\prime $ such that $ x \in B^\prime \subset B $.
	\end{enumerate}
\end{Lemma}
\begin{proof}
	First, we prove that the second condition implies the first one. Given an element $ U $ of $ \mathscr{T} $, we wish to show that $ U \in \mathscr{T}^\prime $. Let $ x \in U $. Since $ \mathscr{B} $ generates $ \mathscr{T} $, there is a element $ B \in \mathscr{B} $ such that $ x \in B \subset U $. Condition $ (2) $ tells us there exists an element $ B^\prime \in \mathscr{B}^\prime $ such that $ x \in B^\prime \subset B $. Then $ x \in B^\prime \subset U $, so $ U \in \mathscr{T}^\prime $ by definition.
	\newpara
	Then we prove that the first condition implies the second. We are given $ x \in X $ and $ B \in \mathscr{B} $. Now $ B $ belongs to $ \mathscr{T} $ by definition and $ \mathscr{T}\subset \mathscr{T}^\prime $ by condition $ (1) $; therefore, $ B \in \mathscr{T}^\prime $. Since $ \mathscr{T}^\prime $ is generated by $ \mathscr{B}^\prime $, there is an element $ B^\prime \in \mathscr{B}^\prime $ such that $ x \in B^\prime \subset B $.
\end{proof}
\begin{Definition}
	If $ \mathscr{B} $ is the collection of all intervals in the real line, the topology generated by $ \mathscr{B} $ is called the \textit{standard topology} on the real line. If $ \mathscr{B}^\prime $ is the collection of all half-open intervals $ [a,b) $ where $ a<b $, the topology generated by $ \mathscr{B}^\prime $ is called the \textit{lower limit topology} on $ \Real $. When $ \Real $ is given the lower limit topology, it's denoted $ \Real_{l} $. Let $ K $ denote the set of all numbers of the form $ 1/n $ for $ n\in \Natural $, and let $ \mathscr{B}^{\prime\prime} $ be the collection of all open intervals $ (a,b) $, along with all sets of the form $ (a,b)\setminus K $. The topology generated by $ \mathscr{B}^{\prime\prime} $ will be called the \textit{K-topology} on $ \Real $. When $ \Real $ is given this topology, it's denoted by $ \Real_{K} $.
\end{Definition}
\begin{Lemma}
	The topologies of $ \Real_l $ and $ \Real_K $ are strictly finer than the standard topology on $ \Real $, but are not comparable with one another.
\end{Lemma}
\begin{proof}
	Let $ \mathscr{T} $, $ \mathscr{T}^\prime $, and $ \mathscr{T}^{\prime\prime} $ be the topologies of $ \Real $, $ \Real_l $, and $ \Real_K $ respectively. Given a basis element $ (a,b) $ for $ \mathscr{T} $ and a point $ x $ of $ (a,b) $, the basis element $ [x,b) $ for $ \mathscr{T}^\prime $ contains $ x $ and lies in $ (a,b) $. On the other hand, given the basis element $ [x,d) $ for $ \mathscr{T}^\prime $, there is no open interval $ (a,b) $ that contains $ x $ and lies in $ [x,d) $, and thus $ \mathscr{T}^\prime $ is strictly finer than $ \mathscr{T} $. \newpara
	A similar argument applies to $ \Real_K $. Given a basis element $ (a,b) $ for $ \mathscr{T} $ and a point $ x\in(a,b) $, this same interval is a basis for $ \mathscr{T}^{\prime\prime} $ that contains $ x $. On the other hand, given the basis element $ B=(-1,1)\setminus K $ for $ \mathscr{T}^{\prime\prime} $ and the point $ 0 $ of $ B $, there is no open interval that contains $ 0 $ and lies in $ B $. \newpara
	Now we show that $ \Real_l $ and $ \Real_K $ are not comparable. For any basis element in $ \Real_l  $ that has $ 0 $ as its lower limit, it always contains number of the form $ 1/n $, thus not any subset of sets of the form $ (a,b)\setminus K $. The rest of this argument is trivial.
\end{proof}
\begin{Definition}
	A \textit{subbasis} $ \mathscr{S} $ for a topology on $ X $ is a collection of subsets of $ X $ whose union equals $ X $. The \textit{topology generated by the subbasis} $ \mathscr{S} $ is defined to be the collection $ \mathscr{T} $ of all unions of finite intersections of elements of $ \mathscr{S} $.
\end{Definition}	
For the purpose of checking whether $ \mathscr{T} $ is a topology, it's sufficient to show that the collection $ \mathscr{B} $ of all finite intersections of elements of $ \mathscr{S} $ is a basis, for then the collection $ \mathscr{T} $ of all unions of elements of $ \mathscr{B} $ is a topology. Given $ x \in X $, it belongs to an element of $ \mathscr{S} $ and hence to an element of $ \mathscr{B} $; to check the second condition, let
\begin{equation}
	B_1 = S_1 \cap \cdots \cap S_m\quad \text{and}\quad B_2=S^\prime_1 \cap \cdots \cap S^\prime_n \nonumber
\end{equation}
to be two elements of $ \mathscr{B} $. Their intersection
\begin{equation}
	B_1 \cap B_2=(S_1 \cap \cdots \cap S_m)\cap (S^\prime_1 \cap \cdots \cap S^\prime_n) \nonumber
\end{equation}
is also a finite intersection of elements of $ \mathscr{S} $, so it belongs to $ \mathscr{B} $.
	
	
	
	
	
	
	
	
	
	
	
	
	
	
	
	
	
	
	
	
\end{document}