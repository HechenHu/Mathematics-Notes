\documentclass{article}
\usepackage[margin=1in]{geometry} 
\usepackage{amsmath,amsthm,amssymb, graphicx, multicol, array}
\usepackage{/Users/HechenHu/Development/NoteTaking/Mathematics-Notes/Customized}

\newenvironment{problem}[2][Problem]{\begin{trivlist}
\item[\hskip \labelsep {\bfseries #1}\hskip \labelsep {\bfseries #2.}]}{\end{trivlist}}

\begin{document}
	
	\title{Problem Set}
	\author{Hechen Hu}
	\maketitle
	
\section{Mathematical Analysis}
\begin{problem}{1}
	These questions deal with measure.
	\begin{enumerate}
		\item Show that a set of measure zero has no interior points.
		\item Show that not having interior points by no means guarantees that a set is of measure zero.
		\item Construct a set having measure zero whose closure is the entire space $ \Real^n $.
	\end{enumerate}
\end{problem}
\begin{problem}{2}
	These questions deal with Riemann integrals.
	\begin{enumerate}
		\item Construct the analogue of the Dirichlet function in $ \Real^n $ and show that a bounded function $ f:I \to \Real $ equal to zero at almost every point of the interval $ I $ may still fail to belong to $ \mathcal{R}(I) $. 
		\item Show that if $ f \in \mathcal{R}(I) $ and $ f(x)=0 $ at almost all points of the interval $ I $, then $ \int_{I}f(x)dx = 0 $.
	\end{enumerate}
\end{problem}
\begin{problem}{3}(The Brunn-Minkowski inequality)
	Given two nonempty sets $ A,B \subset \Real^n $, we form their vector sum in the sense of Minkowski $ A+B \coloneqq \{a+b|a \in A,b \in B \} $. Let $ V(E) $ denote the content of a set $ V(E)\subset \Real^n $.
	\begin{enumerate}
		\item Verify that if $ A $ and $ B $ are standard $ n $-dimensional intervals(parallelepipeds), then
		\begin{equation}
			V^{1/n}(A+B)\geqslant V^{1/n}(A)+V^{1/n}(B) \nonumber
		\end{equation}
		\item Now prove the preceding inequality for arbitrary measurable compact sets $ A $ and $ B $.
	\end{enumerate}
\end{problem}
\begin{problem}{4}
	These questions deal with functions and sets.
	\begin{enumerate}
		\item Construct a subset of the square $ I \subset \Real^2 $ such that on the one hand its intersection with any vertical line and any horizontal line consists of at most one point, while on the other hand its closure equals $ I $.
		\item Construct a function $ f:I \to \Real $ for which both of the iterated integrals that occur in Fubini's theorem exist and equal, yet $ f \notin \mathcal{R}(I) $.
	\end{enumerate}
\end{problem}
\begin{problem}{5}
	Consider the sequence of integrals
	\begin{equation}
		F_0(x)=\int_{0}^{x}f(y)dy\qquad F_n(x)=\int_{0}^{x}\frac{(x-y)^n}{n!}f(y)dy \quad n\in \Natural \nonumber
	\end{equation}
	where $ f \in C(\Real,\Real) $.
	\begin{enumerate}
		\item Verify that $ F_n^\prime(x)=F_{n-1}(x) $, $ F_n^{(k)}(0)=0 $ if $ k \leqslant n $, and $ F_n^{n+1}(x)=f(x) $.
		\item Show that
		\begin{equation}
			\int_{0}^{x}dx_1 \int_{0}^{x_1}dx_2 \cdots \int_{0}^{x_{n-1}}f(x_n)dx_n = \frac{1}{n!}\int_{0}^{x}(x-y)^n f(y)dy \nonumber
		\end{equation}
	\end{enumerate}
\end{problem}



\section{Topology}
\begin{problem}{1}
Let $ X $ be a set.
\begin{enumerate}
		\item If $ \{\mathscr{T}_\alpha \} $ is a family of topologies on $ X $, show that $ \bigcap \mathscr{T}_\alpha $ is a topology on $ X $. Is $ \bigcup \mathscr{T}_\alpha $ a topology on $ X $?
		\item Let $ \{\mathscr{T}_\alpha \} $ be a family of topologies on $ X $. Show that there is a unique smallest topology on $ X $ containing all the collections $ \mathscr{T}_\alpha $, and a unique largest topology contained in all $ \mathscr{T}_\alpha $.
		\item If $ X=\{a,b,c \} $, let
		\begin{equation}
			\mathscr{T}_1 = \{\varnothing, X, \{ a\},\{a,b\} \}\quad \text{and}\quad \mathscr{T}_1 = \{\varnothing, X, \{ a\},\{b,c\} \} \nonumber
		\end{equation}
		Find the smallest topology containing $ \mathscr{T}_1 $ and $ \mathscr{T}_2 $, and the largest topology contained in $ \mathscr{T}_1 $ and $ \mathscr{T}_2 $.
\end{enumerate}
\end{problem}
\begin{problem}{2}
	Show that $ X $ is Hausdorff iff the diagonal $ \Delta = \{x \times x | x \in X \} $ is closed in $ X \times X $.
\end{problem}
\begin{problem}{3}(Kuratowski)
	Consider the collection of all subsets $ A $ of the topological space $ X $. The operations of closure $ A \to \bar{A} $ and complementation $ A \to X \setminus A $ are functions from this collection to itself.
	\begin{enumerate}
		\item Show that starting with a given $ A $, one can form no more than $ 14 $ distinct sets by applying these two operations successively.
		\item Find a subset $ A $ of $ \Real $ (in its usual topology) for which the maximum of $ 14 $ is obtained.
	\end{enumerate}
\end{problem}
\begin{problem}{4}
	Find a function $ f:\Real \to \Real $ that is continuous at precisely one point.
\end{problem}
\begin{problem}{5}
	Let $ X $ be the subset of $ \Real^\omega $ consisting of all sequences $ \vec{x} $ such that $ \sum x_i^2 $ converges. Then the formula
	\begin{equation}
		d(\vec{x},\vec{y}) = \begin{bmatrix}
\sum_{i=1}^{\infty}(x_i-y_i)^2
		\end{bmatrix}^{1/2}\nonumber
	\end{equation}
	defines a metric on $ X $. On $ X $ we have the three topologies inherits from the box, uniform, and product topologies on $ \Real^\omega $. We have also the topology given by the metric $ d $, which we call the $ \ell^2 $-topology.
	\begin{enumerate}
		\item Show that on $ X $, we have the inclusions
		\begin{equation}
			\text{box topology}\supset \ell^2\text{-topology} \supset \text{uniform topology} \nonumber
		\end{equation}
		\item The set $ \Real^\infty $ of all sequences that are eventually zero is contained in $ X $. Show that the four topologies that $ \Real^\infty $ inherits as a subspace of $ X $ are all distinct.
		\item The set
		\begin{equation}
			H = \prod_{n \in \Zahlen_+}[0,1/n] \nonumber
		\end{equation}
		is contained in $ X $; it is called the \textit{Hilbert cube}. Compare the four topologies that $ H $ inherits as a subspace of $ X $.
	\end{enumerate}
\end{problem}


\end{document}
