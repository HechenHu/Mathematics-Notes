\documentclass{article}
\usepackage[margin=1in]{geometry} 
\usepackage{amsmath,amsthm,amssymb, graphicx, multicol, array}
\usepackage{/Users/HechenHu/Development/NoteTaking/Mathematics-Notes/Customized}

\newenvironment{problem}[2][Problem]{\begin{trivlist}
		\item[\hskip \labelsep {\bfseries #1}\hskip \labelsep {\bfseries #2.}]}{\end{trivlist}}

\begin{document}
	
	\title{Problem Set}
	\author{Hechen Hu}
	\maketitle
	
	\section{Mathematical Analysis}
	\begin{problem}{1}
		Let $ f \in C^{(\infty)}(\Real) $. Show that for $ x \neq 0 $
		\begin{equation}
			\frac{1}{x^{n+1}}f^{(n)}(\frac{1}{x}) = (-1)^n \frac{d^n}{d x^n}(x^{n-1} f (\frac{1}{x})) \nonumber
		\end{equation}
	\end{problem}
\begin{problem}{2}
	Show that the function
	\begin{equation}
		f(x)=\begin{cases}
		&\exp(-\frac{1}{(1+x)^2}-\frac{1}{(1-x)^2}) \text{ for }-1<x<1,\\
		&0\quad \text{ for }1 \leqslant |x|
		\end{cases}\nonumber
	\end{equation}
	is infinitely differentiable on $ \Real $.
\end{problem}
\begin{problem}{3}
	Find $ \lim\limits_{x \to \infty} x[\frac{1}{e}-(\frac{x}{x+1})^x] $.
\end{problem}
\begin{problem}{4}
	Show that if a function $ f $ is defined and differentiable on an open interval $ I $ and $ [a,b]\subset I $, then\\
	\textbf{a)} the function $ f^\prime(x) $ (even if it is not continuous) assumes on $ [a,b] $ all the values between $ f^\prime(a) $ and $ f^\prime (b) $ (\textit{the theorem of Darboux}). \\
	\textbf{b)} if $ f^{\prime\prime} $ also exists in $ (a,b) $, then there is a point $ \xi \in (a,b) $ such that $ f^{\prime}(b)-f^\prime (a)=f^{\prime\prime}(\xi)(b-a) $.
\end{problem}

\begin{problem}{5}
	Let $ x=(x_1,\cdots,x_n) $ and $ \alpha = (\alpha_1,\cdots,\alpha_n) $, where $ x_i >0,\alpha_i>0 $ and $ \sum_{i=1}^{n}\alpha_i=1 $. For any number $ t \neq 0 $ we consider the \textit{mean of order $ t $ of the numbers $ x_1,\cdots,x_n $ with weight $ \alpha_i $}:
	\begin{equation}
		M_t(x,\alpha) = (\sum_{i=1}^{n}\alpha_i x^t_i)^{1/t} \nonumber
	\end{equation} 
	In particular, when $ \alpha_1 = \cdots = \alpha_n = \frac{1}{n} $, we obtain the harmonic, arithmetic, and quadratic means for $ t=-1,1,2 $ respectively.\newpara
	Show that \\
	\textbf{a)} $ \lim\limits_{t \to 0}M_t(x,\alpha)=x_1^{\alpha_1}\cdots x_n^{\alpha_n} $, that is, in the limit one can obtain the geometric mean; \\
	\textbf{b)} $ \lim\limits_{t \to + \infty} M_t(x,\alpha)=\max_{1 \leqslant i \leqslant n}x_i$;\\
	\textbf{c)} $ \lim\limits_{t \to - \infty} M_t(x,\alpha)=\min_{1 \leqslant i \leqslant n}x_i$;\\
	\textbf{d)} $ M_t(x,\alpha) $ is a nondecreasing function of $ t $ on $ \Real $ and is strictly increasing if $ n>1 $ and the numbers $ x_i $ are all nonzero.\\
	
\end{problem}	
	
	
\section{Abstract Algebra}	
\begin{problem}{1}
	Prove the following proposition:
	\begin{Proposition}
		Let $ G $ be a semigroup. $ G $ is a group iff for all $ a,b\in G $ the equations $ ax=b $ and $ ya=b $ have solutions in $ G $.
	\end{Proposition}
\end{problem}
\begin{problem}{2}
	Prove the following theorem:
	\begin{Theorem}[Generalized Commutative Law]
		If $ G $ is a commutative semigroup and $ a_1,\cdots,a_n \in G $, then for any permutation $ i_1,\cdots,i_n $ of $ 1,2,\cdots,n $, $ a_1 a_2 \cdots a_n = a_{i_1}a_{i_2}\cdots a_{i_n} $.
	\end{Theorem}
\end{problem}
	
\end{document}
